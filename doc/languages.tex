% Document Type: LaTeX
% Master File: languages.tex
\documentstyle[titreRV,11pt,times]{book}

\marginparwidth 0pt
\oddsidemargin  1 cm
\evensidemargin  1 cm
\marginparsep 0pt
\topmargin   0pt
\textwidth   14 cm
\textheight  22.5 cm

\newcommand{\ignore}[1]{}
\newenvironment{example}{\begin{quotation}\noindent{\bf Example~:}}{\end{quotation}}
\begin{document}
\title{\Huge The Languages of Thot}
\author{Vincent Quint\\
translated by Ethan Munson}
\date{Version of September 28, 1996}
\maketitle

\pagenumbering{roman}
\setcounter{page}{1}
\tableofcontents
\cleardoublepage

\pagenumbering{arabic}
\setcounter{page}{1}
\chapter{The document model of Thot}

All of the services which Thot provides to the user are based on the
system's internal document representation.  This representation is
itself derived from the document model which underlies Thot.  The
model is presented here, prior to the description of the languages
which permit the generic specification of documents.

\section{The logical structure of documents}

The document model of Thot is primarily designed to allow the user to
operate on those entities which he has in mind when he works on a
document.  The model makes no assumptions about the nature of these entities.
It is essentially these logical entities, such as paragraphs, sections,
chapters, notes, titles, and cross-references which give a document
its logical structure.

Because of this model, the author can divide his document into
chapters, giving each one a title.  The content of these chapters can
be further divided into sections, subsections, etc..  The text is
organized into successive paragraphs, as a function of the content.
In the writing phase, the lines, pages, margins, spacing, fonts,
and character styles are not very important.  In fact, if the system
requires documents to be written in these terms, it gets in the way.
So, Thot's model is primarily based on the logical aspect of
documents.  The creation of a model of this type essentially requires
the definition :

\begin{itemize}
\item
of the entities which can appear in the documents,
\item
and the relations between these entities.
\end{itemize}

The choice of entities to include in the model can be subtle.  Some
documents require chapters, while others only need various levels of
sections.  Certain documents contain appendices, others don't.  In
different documents the same logical entity may go by different names
(e.g. ``Conclusion'' and ``Summary'').  Certain entities which are
absolutely necessary in some documents, such as clauses in a contract
or the address of the recipient in a letter, are useless in most other
cases.

The differences between documents result from more than just the
entities that appear in them, but also from the relationships between
these entities and the ways that they are linked.  In certain
documents, notes are spread throughout the document, for example at
the bottom of the page containing the cross-reference to them, while
in other documents they are collected at the end of each chapter or
even at the end of the work.  As another example, the introduction of
some documents can contain many sections, while in other documents,
the introduction is restricted to be a short sequence of paragraphs.

All of this makes it unlikely that a single model can describe any
document at a relatively high level.  It is obviously tempting to make
up a list of widely used entities, such as chapters, sections,
paragraphs, and titles, and then map all other entities onto the
available choices.  In this way, an introduction can be supported as a
chapter and a contract clause supported as a paragraph or section.
However, in trying to widen the range of usage of certain entities,
their meaning can be lost and the power of the model reduced.  In
addition, while this widening partially solves the problem of choosing
entities, it does not solve the problem of their organization: when a
chapter must be composed of sections, how does one indicate that an
introduction has none when it is merely another chapter?  One solution
is to include introductions in the list of supported entities. But
then, how does one distinguish those introductions which are allowed
to have sections from those which are not.  Perhaps this could be done
by defining two types of introduction. Clearly, this approach risks an
infinite expansion of the list of widely used entities.

\section{Generic and specific structures}

Thus, it is apparently impossible to construct an exhaustive inventory of
all those entities which are necessary and sufficient to describe any
document. It also seems impossible to specify all possible
arrangements of these entities in a document.  This is why Thot uses a
{\em meta-model} instead, which permits the description of numerous
{\em models}, each one describing a {\em class} of documents.

A {\em class} is a set of documents having very similar structure.
Thus, the collection of research reports published by a laboratory
constitutes a class; the set of commercial proposals by the sales
department of a company constitutes another class; the set of articles
published by a journal constitutes a third class.  Clearly, it is not
possible to enumerate every possible document class.  It is also clear
that new document classes must be created to satisfy new needs and
applications.

To give a more rigorous definition of classes, we must introduce the
ideas of {\em generic structure} and {\em specific structure}.  Each
document has a {\em specific structure} which organizes the various
parts which comprise it.  We illustrate this with the help of a simple
example comparing two reports, A and B (see Figure~\ref{structspec}).
The report A contains an introduction followed by three chapters and a
conclusion.  The first chapter contains two sections, the second, three
sections.  That is the {\em specific} structure of document A.
Similarly, the structure of document B is: an introduction, two
chapters, a conclusion; Chapter 1 has three sections while Chapter 2
has four.  The specific structures of these two documents are thus
different. 

\begin{figure}
\begin{verbatim}
        Report A                 Report B
             Introduction              Introduction
             Chapter 1                 Chapter 1
                  Section 1.1               Section 1.1
                  Section 1.2               Section 1.2
             Chapter 2                      Section 1.3
                  Section 2.1          Chapter 2
                  Section 2.2               Section 2.1
                  Section 2.3               Section 2.2
             Chapter 3                      Section 2.3
             Conclusion                     Section 2.4
                                       Conclusion
\end{verbatim}
\caption{Two specific structures}
\label{structspec}
\end{figure}

The {\em generic structure} defines the ways in which specific
structures can be constructed.  It specifies how to generate specific
structures.  The reports A and B, though different, are constructed
in accordance with the same generic structure, which specifies that a
report contains an introduction followed by a variable number of
chapters and a conclusion, with each chapter containing a variable
number of sections.

There is a one-to-one correspondence between a class and a generic
structure: all the documents of a class are constructed in accordance
with the same generic structure.  Hence the definition of the class: a
class is a set of documents whose specific structure is constructed in
accordance with the same generic structure.  A class is characterized
by its generic structure.

Thus, a generic structure can be considered to be a model at the level
which interests us, but only for one class of documents.  When the
definition is limited to a single class of documents, it is possible
to define a model which does a good job of representing the documents
of the class, including the necessary entities and unencumbered by
useless entities.  The description of the organization of the
documents in the class can then be sufficiently precise.

\section{Logical structure and physical structure}

Generic structures only describe the {\em logical} organization of
documents, not their {\em physical} presentation on a screen or on
sheets of paper.  However, for a document to be displayed or printed,
its graphic presentation must be taken into account.

An examination of current printed documents shows that the details of
presentation essentially serve to bring out their logical structure.
Outside of some particular domains, notably advertising, the
presentation is rarely independent of the logical organization of the
text.  Moreover, the art of typography consists of enhancing the
organization of the text being set, without catching the eye of the
reader with overly pronounced effects.  Thus, italic and boldface type
are used to emphasize words or expressions which have greater
significance than the rest of the text: keywords, new ideas,
citations, book titles, etc.  Other effects highlight the organization
of the text: vertical space, margin changes, page breaks, centering,
eventually combined with the changes in the shapes or weight of the
characters. These effects serve to indicate the transitions between
paragraphs, sections, or chapters: an object's level in the logical
structure of the document is shown by the markedness of the effects.

Since the model permits the description of all of the logical
structure of the document, the presentation can be derived from the
model without being submerged in the document itself.  It suffices to
use the logical structure of the document to make the desired changes
in its presentation: changes in type size, type style, spacing, margin,
centering, etc.

Just as one cannot define a unique generic logical structure for all
document classes, one cannot define universal presentation rules which
can be applied to all document classes.  For certain types of
documents the chapter titles will be centered on the page and printed
in large, bold type.  For other documents, the same chapter titles
will be printed in small, italic type and aligned on the left margin.

Therefore, it is necessary to base the presentation specifications for
documents on their class.  Such a specification can be very
fine-grained, because the presentation is expressed in terms of the
entities defined in the generic logical structure of the class.  Thus,
it is possible to specify a different presentation for the chapter
titles and the section titles, and similarly to specify titles for the
sections according to their level in the section hierarchy.  The set
of rules which specify the presentation of all the elements defined in
a generic logical structure is called a {\em generic presentation}.

There are several advantages derived from having a presentation linked
to the generic structure and described by a generic presentation.
Homogeneity is the first.  Since every document in a class corresponds
to the same generic logical structure, a homogenous presentation for
different documents of the same class can be assured by applying the
same generic presentation to all documents of the class.  Homogeneity
of presentation can also be found among the entities of a single
document: every section title will be presented  in the same way, the
first line of every paragraph of the same type will have the same
indentation, etc.

Another advantage of this approach to presentation is that it
facilitates changes to the graphical aspect of documents.  A change to
the generic presentation rules attached to each type of entity will
alter the presentation of the entire document, and will do so
homogenously.  In this case, the internal homogeneity of the class is
no longer assured, but the way to control it is simple.  It suffices
to adopt a single generic presentation for the entire class.

If the presentation of the class does not have to be homogenous, then
the appearance of the document can be adapted to the way it will be
used or to the device used to render it.  This quality is sufficient
to allow the existence of many generic presentations for the same
document class.
\label{presmul}
By applying one or the other of these presentations
to it, the document can be seen under different graphical aspects.  It
must be emphasized that this type of modification of the presentation
is not a change to the document itself (in its specific logical
structure or its content), but only in its appearance at the time of
editing or printing.

\section{Document structures and object structures}

So far, we have only discussed the global structure of documents and
have not considered the contents found in that structure.  We could
limit ourselves to purely textual contents by assuming that a title or
a paragraph contains a simple linear text.  But this model would be
too restrictive.  In fact, certain documents contain not only text,
but also contain tables,  diagrams,  photographs, mathematical
formulas, and program fragments.  The model must permit the
representation of such {\em objects}.

Just as with the whole of the document, the model takes into account
the logical structure of objects of this type.  Some are clearly
structured, others are less so.  Logical structure can be recognized
in mathematical formulas, in tables, and in certain types of
diagrams.  On the other hand, it is difficult to define the structure
of a photograph or of some drawings.  But in any case, it does not
seem possible to define one unique structure which can represent every one
of these types of objects.  The approach taken in the definition of
meta-structure and document classes also applies to objects.  Object
classes can be defined which put together objects of similar type,
constructed from the same generic logical structure.

Thus, a mathematical class can be defined and have a generic logical
structure associated with it.  But even if a single generic structure
can represent a sufficient variety of mathematical formulas, for other
objects with less rigorous structure, multiple classes must be
defined.  As for documents, using multiple classes assures that the
model can describe the full range of objects to be presented.  It also
permits the system to support objects which were not initially
anticipated.  Moreover, this comment applies equally to mathematics:
different classes of formulas can be described depending on the domain
of mathematics being described.

Since objects have the same level of logical representation as
documents, they gain the same advantages.  In particular, it is
possible to define the presentation separately from the objects
themselves and attach it to the class.  Thus, as for documents,
objects of the same type have a uniform presentation and the
presentation of every object in a given class can be changed simply by
changing the generic presentation of the class.  Another advantage of
using this document model is that the system does not bother the user
with the details of presentation, but rather allows the user to
concentrate on the logical aspect of the document and the objects.

It is clear that the documents in a class do not necessarily use the
same classes of objects: one technical report will contain tables while
another report will have no tables but will use mathematical formulas.
The usable object classes are not mentioned in a limiting way in the
generic logical structure of documents.  Rather, they can be chosen
freely from a large set, independent of the document class.

Thus, the object classes will be made commonplace and usable in every
document. The notion of ``object'' can be enlarged to include not only
non-textual elements, but also certain types of textual elements which
can appear in practically every document, whatever their class.  Among
these textual elements, one can mention certain elements defined by
the Scribe and {\LaTeX} formatters as {\em environments}:
enumerations, descriptions, examples, quotations, even paragraphs.

Thus, the document model is not a single, general model describing
every type of document in one place.  Rather, it is a meta-model which
can be used to describe many different models each of which represents
either a class of similar documents or a class of similar objects
which every document can include.

\chapter{The S language}
\section{Document meta-structure}

Since the concept of meta-structure is well suited to the task of
describing documents at a high level of abstraction, this
meta-structure must be precisely defined.  Toward that end this
section first presents the basic elements from which documents and
structured objects are composed and then specifies the ways in which
these basic elements are assembled into structures representing
complete documents and objects.

\subsection{The basic types}

At the lowest level of a document's structure, the first atom
considered is the character.  However, since characters are seldom isolated,
usually appearing as part of a linear sequence, and in order to reduce
the complexity of the document structure, {\em character strings} are
used as atoms and consecutive characters belonging to the same
structural element are grouped in the same character string.

If the structure of a document is not refined to go down to  the level
of words or phrases, the contents of a simple paragraph can be
considered to be a single character string.  On the other hand, the
title of a chapter, the title of the first section of that chapter,
and the text of the first paragraph of that section constitute three
different character strings, because they belong to distinct
structural elements.

If, instead, a very fine-grained representation for the structure of a
document is sought, character strings could be defined to contain only
a single word, or even just a single character.  This is the case, for
example, in programs,  for which one wants to retain a structure very
close to the syntax of the programming language.  In this case, a
Pascal assignment statement initializing a simple variable to zero
would be composed of two structural elements, the identifier of the
variable (a short character string) and the assigned value (a string
of a single character, `0').

The character string is not the only atom necessary for representing
those documents that interest us.  It suffices for purely textual
documents, but as soon as the non-textual objects which we have
considered arise, there must be other atoms; the number of objects
which are to be represented determines the number of types of atoms
that are necessary.

Primitive {\em graphical elements} are used for tables and figures of
different types.  These elements are simple geometric shapes like
horizontal or vertical lines, which are sufficient for tables, or even
oblique lines, arrows, rectangles, and circles for use in figures.
From these elements and character strings, graphical objects and
tables can be constructed.

Photographs, though having very little structure, must still appear in
documents.  They are supported by {\em image} elements, which are
represented as matrices of points.

Finally, mathematical notations require certain elements which are
simultaneously characters and graphical elements, the {\em symbols}.
By way of example, Radicals, integration signs, or even large
parentheses are examples of this type of atom.  The size of each of
these symbols is determined by its environment, that is to say, by the
expression to which it is attached.

To summarize, the primitive elements which are used in the
construction of documents and structured objects are:
\begin{itemize}
  \item character strings,
  \item graphical elements,
  \item images,
  \item and mathematical symbols.
\end{itemize}

\subsection{Constructed elements}

A document is evidently formed from primitive elements.  But the model
of Thot also proposes higher level elements.  Thus, in a document
composed of several chapters, each chapter is an element, and in the
chapters each section is also an element, and so on.  A document is
thus an organized set of elements.

In a document there are different sorts of elements.  Each element has
a {\em type} which indicates the role of the element within the
document as a whole.  Thus, we have, for example, the chapter and
section types.  The document is made up of typed elements: elements of
the type chapter and elements of the type section, among others, but
also character string elements  and graphical elements: the primitive
elements are typed elements just as well.  At the other extreme, the
document itself is also considered to be a typed element.

The important difference between the primitive elements and the other
elements of the document is that the primitive elements are atoms
(they cannot be decomposed), whereas the others, called {\em
constructed elements}, are composed of other elements, which can
either be primitive elements or constructed elements.  A constructed
element of type chapter (or more simply, ``a chapter'') is composed
of sections, which are also constructed elements.  A paragraph, a
constructed element, can be made up of character strings, which are
primitive elements, and of equations, which are constructed elements.

A parallel can be made between the element types and the data types of
typed programming languages, such as Pascal.  The four primitive types
of Thot are comparable to the standard primitive types of Pascal:
integer, character, boolean, and real.  The types of constructed
elements are comparable to the structured types of Pascal which each
programmer defines from the primitive types.

A document is also a constructed element.  This is an important point.
In particular, it allows a document to be treated as part of another
document, and conversely, permits a part of a document to be treated
as a complete document.  Thus, an article presented in a journal is
treated by its author as a document in itself, while the editor of the
journal considers it to be part of an issue.  A table or a figure
appearing in a document can be extracted and treated as a complete
document, for example to prepare transparencies for a conference.

These thoughts about types and constructed elements apply just as well
to objects as they do to documents.  A table is a constructed element
made up of other constructed elements, rows and columns.  A row is
formed of cells, which are also constructed elements which contain
primitive elements (character strings) and/or constructed elements
like equations.

\subsection{Logical structure constructors}

Having defined the primitive elements and the constructed elements, it
is now time to define the types of organization which allow the
building of structures.  For this, we rely on the notion of the {\em
constructor}.  A constructor defines a way of assembling certain
elements in a structure.  It resides at the level of the
meta-structure: it does not describe the existing relations in a given
structure, but rather defines  how elements are assembled to build a
structure that conforms to a model.  In comparison with Pascal, the
constructors correspond to the methods for building structured data
types: arrays, records, sets, and files.

In defining the overall organization of documents, the first two
constructors considered are the aggregate and the list.

\subsubsection{Aggregate and List}

The {\em aggregate} constructor is used to define constructed element
types which are collections of a given number of other elements.
These collections may or may not be ordered.  The elements may be
either constructed or primitive and are specified by their type.  A
report (that is, a constructed element of the report type) has an
aggregate structure.  It is formed from a title, an author's name, an
introduction, a body, and a conclusion, making it a collection of five
element types.  This type of constructor is found in practically every
document, and generally at several levels in a document.  It
corresponds to the {\tt record} of Pascal.

The {\em list} constructor is used to define constructed elements
which are ordered sequences of elements (constructed or primitive)
having the same type.  The minimum and maximum numbers of elements for
the sequence can be specified in the list constructor or the number of
elements can be left unconstrained.  The body of a report is a list of
chapters and is typically required to contain a minimum of two
chapters (is a chapter useful if it is the only one in the report?)
The chapter itself can contain a list of sections, each section
containing a list of paragraphs.  The list constructor corresponds to
a one-dimensional array in Pascal.  In the same way as the aggregate,
the list is a very frequently used constructor in every type of
document.  However, these two constructors are not sufficient to
describe every document structure; thus other constructors supplement
them.

\subsubsection{Choice, Schema, and Unit}
\label{schemasandunits}

The {\em choice} constructor is used to define the structure of an
element type for which one alternative is chosen from several
possibilities.  Thus, a paragraph can be either a simple text
paragraph, or an enumeration, or a citation.  The choice constructor
can be compared to the variant parts of records in Pascal ({\tt case}
in a {\tt record}).

The choice constructor indicates the complete list of possible
options, which can be too restrictive in certain cases, the paragraph
being one such case.  Two constructors, {\em unit} and {\em schema},
address this inconvenience.  They allow more freedom in the choice of
an element type.  If a paragraph is defined by a schema constructor,
it is possible to put in the place of a paragraph a table, an
equation, a drawing or any other object defined by another generic
logical structure.  It is also possible to define a paragraph as a
sequence of units, which could be character strings, symbols, or
images.  The choice constructor alone defines a generic logical
structure that is relatively constrained; in contrast, using units and
schemas, a very open structure can be defined.  The unit and schema
constructors have no equivalents in Pascal.

The {\em schema} constructor represents an object defined by a generic
logical structure chosen freely from among those
available.\footnote{Translator's note: The French name for this
constructor is ``{\em nature}''.  The standard translations of this
word do not work well in English, so a new name for the constructor
was created.}

The {\em unit} constructor represents an element whose type can be
either a primitive type or an element type defined as a unit in the generic
logical structure of the document.\footnote{or in another generic
logical structure used in the document} Such an element may be used in
document objects constructed according to other generic
structures.

Thus, for example, if a cross-reference to a footnote is defined in
the generic logical structure ``Article'' as a unit, a table (an
object defined by another generic structure) can contain
cross-references to footnotes, when they appear in an article.  In
another type of document, a table defined by the same generic structure
can contain other types of elements, depending on the type of document
into which the table is inserted.  All that is needed is to declare, in
the generic structure for tables, that the contents of cells are
units.  In this way, the generic structure of objects is divided up
between different types of documents which are able to adapt
themselves to the environment into which they are inserted.

\subsubsection{Reference and Inclusion}

The {\em reference} is the last constructor.  It is used to define
document elements that are cross-references to other elements, such as
a section, a chapter, a bibliographic citation, or a figure.  The
reference can be compared to the pointer in Pascal, but in contrast to
the pointer, the reference is bi-directional.  The reference can be
used to access both the element being cross-referenced and each of the
elements which make use of the cross-reference.

References can be either {\em internal} or {\em external}.  That is,
they can designate elements which appear in the same document or in
another document.

\label{inclusion}
The {\em inclusion} constructor is a special type of reference.  Like
the reference, it is an internal or external bidirectional link, but
it is not a cross-reference.  This link represents the ``live''
inclusion of the designated element; it accesses the most recent
version of that element and not a ``dead'' copy, fixed in the state
in which it was found at the moment the copy was made.  As soon as an
element is modified, all of its inclusions are automatically brought
up to date.  It must be noted that, in addition to inclusion, the Thot
editor permits the creation of ``dead'' copies.

There are three types of inclusions: inclusions with full expansion,
inclusions with partial expansion, and inclusions without expansion.
During editing, inclusions without expansion are represented on the
screen by the name of the included document (in gray or in a special
color, depending on the type of screen), while inclusions with
expansion (full or partial) are represented by a copy (full or
partial) of the included element (also in gray or a special color).
The on-screen representation of a partial inclusion is a ``skeleton''
image of the included document (see page~\pageref{squelette}).

Inclusion with complete expansion can be used to include parts of the
same document or of other documents.  Thus, it can be either an
internal or an external link.  It can be used to include certain
bibliographic entries of a scientific article in another article, or
to copy part of a mathematical formula into another formula of the
same document, thus assuring that both copies will remain
synchronized.

Inclusion without expansion or with partial expansion is used to
include complete documents.  It is always an external link.  It is
used primarily to divide very large documents into sub-documents that
are easier to manipulate, especially when there are many authors.  So,
a book can include some chapters, where each chapter is a different
document which can be edited separately.  When viewing the book on the
screen, it might be desirable to see only the titles of the chapters
and sections.  This can be achieved using inclusion with partial
expansion.

During printing, inclusions without expansion or with partial
expansion can be represented either as they were shown on the screen
or by a complete (and up-to-date) copy of the included element or document.

The inclusion constructor, whatever its type, respects the generic
structure: only those elements authorized by the generic structure can
be included at a given position in a document.

\subsubsection{Mark pairs}

It is often useful to delimit certain parts of a document
independently from the logical structure.  For example, one might wish
to attach some information (in the form of an attribute, see
section~\ref{attributes}) or a particular treatment to a group of words
or a set of consecutive paragraphs.  {\em Mark pairs} are used to do
this.

Mark pairs are elements which are always paired and are terminals in
the logical structure of the document.  Their position in the
structure of the document is defined in the generic structure.  It is
important to note that when the terminals of a mark pair are {\em
extensions} (see the next section), they can be used quite freely.

\subsubsection{Restrictions and Extensions}
\label{restrictionextensions}

The primitive types and the constructors presented so far permit the
definition of the logical structure of documents and objects in a
rigorous way.  But this definition can be very cumbersome in certain
cases, notably when trying to constrain or extend the authorized
element types in a particular context.  {\em Restrictions} and {\em
extensions} are used to simplify these cases.

A restriction associates with a particular element type $A$, a list of
those element types which elements of type $A$ may not contain, even
if the definition of type $A$ and those of its components authorize
them otherwise.  This simplifies the writing of generic logical
structures and allows limitations to be placed, when necessary, on the
choices offered by the schema and unit constructors.

Extensions are the inverse of restrictions.  They identify a list of
element types whose presence {\em is} permitted, even if its
definition and those of its components do not authorize them
otherwise.

\subsubsection{Summary}

Thus, four constructors are used to construct a document:
\begin{itemize}
  \item the aggregate constructor,
  \item the list constructor,
  \item the choice constructor and its extensions, the unit and schema
constructors,
  \item the reference constructor and its variant, the inclusion.
\end{itemize}

These constructors are also sufficient for objects.  Thus, these
constructors provide a homogenous meta-model which can describe both
the organization of the document as a whole and that of the various
types of objects which it contains.  After presenting the description
language for generic structures, we will present several examples
which illustrate the appropriateness of the model.

The first three constructors (aggregate, list and choice) lead to
tree-like structures for documents and objects, the objects being
simply the subtrees of the tree of a document (or even of other
objects' subtrees).  The reference constructor introduces other,
non-hierarchical, relations which augment those of the tree: when a
paragraph makes reference to a chapter or a section, that relation
leaves the purely tree-like structure.  Moreover,  external reference and
inclusion constructors permit the establishment of links between different
documents, thus creating a hypertext structure.

\subsection{Associated Elements}
\label{elemassoc}

Thanks to the list, aggregate and choice constructors, the organization of the
document is specified rigorously, using constructed and primitive
elements.  But a document is made up of more than just its elements;
it clearly also contains links between them.  There exist elements
whose position in the document's structure is not determinable.  This
is notably the case for figures and notes.  A figure can be designated
at many points in the same document and its place in the physical
document can vary over the life of the document without any effect on
the meaning or clarity of the document.  At one time, it can be placed
at the end of the document along with all other figures.  At another
time, it can appear at the top of the page which follows the first
mention of the figure.  The figures can be dispersed throughout the
document or can be grouped together.  The situation is similar for
notes, which can be printed at the bottom of the page on which they
are mentioned or assembled together at the end of the chapter or even
the end of the work.  Of course, this brings up questions of the
physical position of elements in documents that are broken into pages,
but this reflects the structural instability of these elements.  They
cannot be treated the same way as elements like paragraphs or
sections, whose position in the structure is directly linked to the
semantics of the document.

Those elements whose position in the structure of the document is not
fixed, even though they are definitely part of the document, are
called {\em associated elements}.  Associated elements are themselves
structures, which is to say that their content can be organized
logically by the constructors from primitive and constructed elements.

It can happen that the associated elements are totally disconnected
from the structure of the document, as in a commentary or appraisal of
the entire work.  But more often, the associated elements are linked
to the content of the document by references.  This is generally the
case for notes and figures, among others.

Thus, associated elements introduce a new use for the reference
constructor.  It not only serves to create links between elements of
the principal structure of the document, but also serves to link the
associated elements to the primary structure.

\subsection{Attributes}
\label{attributes}

There remain logical aspects of documents that are not entirely
described by the structure.  Certain types of semantic information,
which are not stated explicitly in the text, must also be taken into
account.  In particular, such information is shown by typographic
effects which do not correspond to a change between structural
elements.  In fact, certain titles are set in bold or italic or are
printed in a different typeface from the rest of the text in order to
mark them as structurally distinct.  But these same effects frequently
appear in the middle of continuous text (e.g. in the interior of a
paragraph).  In this case, there is no change between structural
elements; the effect serves to highlight a word, expression, or
phrase.  The notion of an {\em attribute} is used to express this type
of information.

An attribute is a piece of information attached to a structural
element which augments the type of the element and clarifies its
function in the document.  Keywords, foreign language words, and
titles of other works can all be represented by character strings with
attached attributes.  Attributes may also be attached to constructed
elements.  Thus, an attribute indicating the language can be attached
to a single word or to a large part of a document.

In fact, an attribute can be any piece of information which is linked
to a part of a document and which can be used by agents which work on
the document.  For example, the language in which the document is
written determines the set of characters used by an editor or
formatter.  It also determines the algorithm or hyphenation dictionary
to be used.  The attribute ``keyword'' facilitates the work of an
information retrieval system.  The attribute ``index word'' allows a formatter
to automatically construct an index at the end of the document.

As with the types of constructed elements, the attributes and the
values they can take are defined separately in each generic logical
structure, not in the meta-model, according to the needs of the
document class or the nature of the object.

Many types of attributes are offered: numeric, textual, references,
and enumerations.  {\em Numeric attributes} can take integer values
(negative, positive, or null).  {\em Textual attributes} have as their
values character strings.  {\em Reference attributes} designate an
element of the logical structure.  {\em Enumeration attributes} can
take one value from a limited list of possible values, each value
being a name.

In a generic structure, there is a distinction between {\em global
attributes} and {\em local attributes}.  A global attribute can be
applied to every element type defined in the generic structure where
it is specified.  In contrast, a local attribute can only be applied
to certain types of elements, even only a single type.  The
``language'' attribute presented above is an example of a global
attribute.  An example of a local attribute is the rank of an author
(principal author of the document or secondary author): this attribute
can only be applied sensibly to an element of the ``author'' type.

Attributes can be assigned to the elements which make up the document
in many different ways.  The author can freely and dynamically place
them on any part of the document in order to attach 
supplementary information of his choice.  However, attributes may only
be assigned in accordance with the rules of the generic structure; in
particular, local attributes can only be assigned to those element
types for which they are defined.

In the generic structure, certain local attributes can be made
mandatory for certain element types.  In this case, the Thot editor
automatically associates the attribute with the elements of this type
and it requires the user to provide a value for this attribute.

Attributes can also be automatically assigned, with a given value, by
every application processing the document in order to systematically
add a piece of information to certain predefined elements of the
document.  By way of example, in a report containing a French abstract
and an English abstract, each of the two abstracts is defined as a
sequence of paragraphs.  The first abstract has a value of ``French''
for the ``language'' attribute while the second abstract's
``language'' attribute has a value of ``English''.

In the case of mark pairs, attributes are logically associated with
the pair as a whole, but are actually attached to the first mark.

\subsection{Discussion of the model}

The notions of attribute, constructor, structured element, and
associated element are used in the definition of generic logical
structures of documents and objects.  The problem is to assemble them
to form generic structures.  In fact, many types of elements and
attributes can be found in a variety of generic structures.  Rather
than redefine them for each structure in which they appear, it is best
to share them between structures. The object classes already fill this
sharing function.  If a mathematical class is defined, its formulas
can be used in many different document classes, without redefining the
structure of each class.  This problem arises not only for the objects
considered here; it also arises for the commonplace textual elements
found in many document classes.  This is the reason why the notion of
object is so broad and why paragraphs and enumerations are also
considered to be objects.  These object classes not only permit the
sharing of the structures of elements, but also of the attributes
defined in the generic structures.

Structure, such as that presented here, can appear very rigid, and it
is possible to imagine that a document editing system based on this
model could prove very constraining to the user.  This is, in fact, a
common criticism of syntax-directed editors.  This defect can be
avoided with Thot, primarily for three reasons:
\begin{itemize}
  \item the generic structures are not fixed in the model itself,
  \item the model takes the dynamics of documents into account,
  \item the constructors offer great flexibility.
\end{itemize}

When the generic structure of a document is not predefined, but rather
is constructed specifically for each document class, it can be
carefully adapted to the current needs.  In cases where the generic
structure is inadequate for a particular document of the class, it is
always possible either to create a new class with a generic structure
well suited to the new case or to extend the generic structure of the
existing class to take into account the specifics of the document
which poses the problem.  These two solutions can also be applied to
objects whose structures prove to be poorly designed.

The model is sufficiently flexible to take into account all the phases
of the life of the document.  When a generic structure specifies that
a report must contain a title, an abstract, an introduction, at least
two chapters, and a conclusion, this means only that a report, {\em
upon completion}, will have to contain all of these elements.  When
the author begins writing, none of these elements is present.  The
Thot editor uses this model.  Therefore, it tolerates documents which
do not conform strictly to the generic structure of their class;  it
also considers the generic logical structure to be a way of helping
the user in the construction of a complex document.

In contrast, other applications may reject a document which does not
conform strictly to its generic structure.  This is, for example, what
is done by compilers which refuse to generate code for a program which
is not syntactically correct.  This might also occur when using a
document application for a report which does not have an abstract or
title.

The constructors of the document model bring a great flexibility to the
generic structures.  A choice constructor (and even more, a unit or
schema constructor)  can represent several, very different elements.
The list constructor permits the addition of more elements of the
same type.  Used together, these two constructors permit any series of
elements of different types.  Of course, this flexibility can be
reduced wherever necessary since a generic structure can limit the
choices or the number of elements in a list.

Another difficulty linked to the use of structure in the document
model resides in the choice of the level of the structure.  The
structure of a discussion could be extracted from the text itself via
linguistic analysis.  Some studies are exploring this approach, but
the model of Thot excludes this type of structure.  It only takes into
account the logical structure provided explicitly by the author.

However, the level of structure of the model is not imposed.  Each
generic structure defines its own level of structure, adapted to the
document class or object and to the ways in which it will be
processed.  If it will only be edited and printed, a  relatively
simple structure suffices.  If more specialized processing will be
applied to it, the structure must represent the element types on which
this processing must act.  By way of example, a simple structure is
sufficient for printing formulas, but a more complex structure is
required to perform symbolic or numeric calculations on the
mathematical expressions.  The document model of Thot allows both
types of structure.

\section{The definition language for generic structures}

Generic structures, which form the basis of the document model of
Thot, are specified using a special language.  This definition
language, called S, is described in this section.

Each generic structure, which defines a class of documents or objects,
is specified by a kind of program, written in the S language, which is
called a {\em structure schema}.  Structure schemas are compiled into
tables, called structure tables, which are used by the Thot editor and
which determine its behavior. 

\subsection{Writing Conventions}
\label{metalang}

The grammar of S, like those of the languages P and T presented later,
is described using the meta-language M, derived from the Backus-Naur
Form (BNF).

In this meta-language each rule of the grammar is composed of a
grammar symbol followed by an equals sign (`=') and the right part of
the rule.  The equals sign plays the same role as the traditional
`::=' of BNF: it indicates that the right part defines the symbol of
the left part.  In the right part,
\begin{description}
\item[concatenation] is shown by the juxtaposition of symbols;

\item[character strings] between apostrophes ' represent terminal
symbols, that is, keywords in the language defined.  Keywords are
written here in upper-case letters, but can be written in any
combination of upper and lower-case letters.  For example, the keyword
{\tt DEFPRES} of S can also be written as {\tt defpres} or {\tt
DefPres}.

\item[material between brackets] (`[' and `]') is optional;

\item[material between angle brackets] (`\verb|<|' and `\verb|>|') can be repeated
many times or omitted;

\item[the slash] (`/') indicates an alternative, a choice between the
options separated by the slash character;

\item[the period] marks the end of a rule;

\item[text between braces] (`\{' and `\}') is simply a comment.
\end{description}

The M meta-language also uses the concepts of identifiers, strings,
and integers:
\begin{description}
\item[{\tt NAME}] represents an identifier, a sequence of letters
(upper or lower-case), digits, and underline characters (`\verb|_|'),
beginning with a letter.  Also considered a letter is the sequence of
characters `{\verb|\nnn|}' where the letter {\tt n} represents the ISO
Latin 1 code of the letter in octal.  It is thus possible to use
accented letters in identifiers.  The maximum length of identifiers
is fixed by the compiler.  It is normally 19 characters.

Unlike keywords, upper and lower-case letters are distinct in
identifiers.  Thus, {\tt Title}, {\tt TITLE}, and {\tt title} are
considered different identifiers.

\item[{\tt STRING}] represents (amazingly enough) a string.  This is a
string of characters delimited by apostrophes.  If an apostrophe must
appear in a string, it is doubled.  As with identifiers, strings can
contain characters represented by their octal code (after a
backslash).  As with apostrophes, if a backslash must appear in a
string, it is doubled.

\item[{\tt NUMBER}] represents a positive integer or zero (without a
sign), or said another way, a sequence of decimal digits.
\end{description}

The M language can be used to define itself as follows:
\begin{verbatim}
{ Any text between braces is a comment. }
Grammar      = Rule < Rule > 'END' .
               { The < and > signs indicate zero }
               { or more repetitions. }
               { END marks the end of the grammar. }
Rule         = Ident '=' RightPart '.' .
               { The period indicates the end of a rule }
RightPart    = RtTerminal / RtIntermed .
               { The slash indicates a choice }
RtTerminal   ='NAME' / 'STRING' / 'NUMBER' .
               { Right part of a terminal rule }
RtIntermed   = Possibility < '/' Possibility > .
               { Right part of an intermediate rule }
Possibility  = ElemOpt < ElemOpt > .
ElemOpt      = Element / '[' Element < Element > ']' /
              '<' Element < Element > '>'  .
               { Brackets delimit optional parts }
Element      = Ident / KeyWord .
Ident        = NAME .
               { Identifier, sequence of characters }
KeyWord      = STRING .
               { Character string delimited by apostrophes }
END
\end{verbatim}

\subsection{Extension schemas}
\label{schext}

A structure schema defines the generic logical structure of a class of
documents or objects, independent of the operations which can be
performed on the documents.  However, certain applications may require
particular information to be represented by the structure for the
documents that they operate on.  Thus a document version manager will need to
indicate in the document the parts which belong to one version or
another.  An indexing system will add highly-structured index tables
as well as the links between these tables and the rest of the
document.

Thus, each application needs to extend the generic structure of the
documents on which it operates to introduce new attributes, associated
elements or element types.  These additions are specific to each
application and must be able to be applied to any generic structure:
users will want to manage versions or construct indices for many types
of documents.  Extension schemas fulfill this role: they define
attributes, elements, associated elements, units, etc., but they can
only be used jointly with a structure schema that they complete.
Otherwise, structure schemas can always be used without these
extensions when the corresponding applications are not available.

\subsection{The general organization of structure schemas}

Every structure schema begins with the keyword {\tt STRUCTURE} and
ends with the keyword {\tt END}.  The keyword {\tt STRUCTURE} is
followed by the keyword {\tt EX\-TEN\-SION} in the case where the schema
defines an extension, then by the name of the generic structure which
the schema defines (the name of the document or object class).  The
name of the structure is followed by a semicolon.

In the case of a complete schema (that is, a schema which is not an
extension), the definition of the name of the structure is followed by
the declarations of the default presentation schema, the global
attributes, the parameters, the structure rules, the associated
elements, the units, the skeleton elements and the exceptions.  Only
the definition of the structure rules is required.  Each series of
declarations begins with a keyword: {\tt DEFPRES}, {\tt ATTR}, {\tt
PARAM}, {\tt STRUCT}, {\tt ASSOC}, {\tt UNITS}, {\tt EXPORT} et {\tt
EXCEPT}.

In the case of an extension schema, there are neither parameters nor
skeleton elements and the {\tt STRUCT} section is optional, while that
section is required in a schema that is not an extension.  On the
other hand, extension schemas can contain an {\tt EXTENS} section,
which must not appear in a schema which is not an extension; this
section defines the complements to attach to the rules found in the
schema to which the extension will be added.  The sections {\tt ATTR},
{\tt STRUCT}, {\tt ASSOC}, and {\tt UNITS} define new attributes, new
elements, new associated elements, and new units which add their
definitions to the principal schema.

\begin{verbatim}
     StructSchema ='STRUCTURE' ElemID ';'
                   'DEFPRES' PresID ';'
                 [ 'ATTR' AttrSeq ]
                 [ 'PARAM' RulesSeq ]
                   'STRUCT' RulesSeq
                 [ 'ASSOC' RulesSeq ]
                 [ 'UNITS' RulesSeq ]
                 [ 'EXPORT' SkeletonSeq ]
                 [ 'EXCEPT' ExceptSeq ]
                   'END' .
     ElemID       = NAME .
\end{verbatim}
or
\begin{verbatim}
     ExtensSchema ='STRUCTURE' 'EXTENSION' ElemID ';'
                   'DEFPRES' PresID ';'
                 [ 'ATTR' AttrSeq ]
                 [ 'STRUCT' RulesSeq ]
                 [ 'EXTENS' ExtensRuleSeq ]
                 [ 'ASSOC' RulesSeq ]
                 [ 'UNITS' RulesSeq ]
                 [ 'EXCEPT' ExceptSeq ]
                   'END' .
     ElemID       = NAME .
\end{verbatim}


\subsection{The default presentation}

It was shown on page~\pageref{presmul} that many different
presentations are possible for documents and objects of the same
class.  The structure schema defines a preferred presentation for the
class, called the {\em default presentation}.  Like generic
structures, presentations are described by programs, called {\em
presentation schemas}, which are written in a specific language, P,
presented later in this document (see page~\pageref{langp}).  The name
appearing after the keyword {\tt DEFPRES} is the name of the default
presentation schema.  When a new document is created, the Thot editor
will suggest using this presentation schema, but the user remains free
to choose another if he wishes.

\begin{verbatim}
     PresID = NAME .
\end{verbatim}

\subsection{Global Attributes}
\label{attrglobaux}

If the generic structure includes global attributes of its own, they
are declared after the keyword {\tt ATTR}.  Each global attribute is
defined by its name, followed by an equals sign and the definition of
its type.  The declaration of a global attribute is terminated by a
semi-colon.

For attributes of the numeric, textual, or reference types, the type
is indicated by a keyword, {\tt INTEGER}, {\tt TEXT}, or {\tt
REFERENCE} respectively.

In the case of a reference attribute, the keyword {\tt REFERENCE} is
followed by the type of the referenced element in parentheses.  It can
refer to any type at all, specified by using the keyword {\tt ANY}, or
to a specific type.  In the latter case, the element type designated by
the reference can be defined either in the {\tt STRUCT} section of the
same structure schema (see Section~\ref{elemstruct}) or in the {\tt
STRUCT} section of another structure schema.  When the type is defined
in another schema, the element type is followed by the name of the
structure schema (within parentheses) in which it is defined.  The
name of the designated element type can be preceded by the keyword
{\tt First} or {\tt Second}, but only in the case where the type is
defined as a pair (see section~\ref{paires}).  The keywords indicate
whether the attribute must designate the first mark of the pair or the
second.  If the reference refers to a pair and neither of these two
keywords is present, then the first mark is used.

In the case of an enumeration attribute, the equals sign is followed
by the list of names representing the possible values of the
attribute, the names being separated from each other by commas.  An
enumeration attribute has at least one possible value; the maximum
number of values is defined by the compiler for the S language.

\begin{verbatim}
     AttrSeq   = Attribute < Attribute > .
     Attribute = AttrID '=' AttrType  ';' .
     AttrType  = 'INTEGER' / 'TEXT' /
                 'REFERENCE' '(' RefType ')' /
                 ValueSeq .
     RefType   = 'ANY' / [ FirstSec ] ElemID [ ExtStruct ] .
     FirstSec  = 'First' / 'Second' .
     ExtStruct = '(' ElemID ')' .
     ValueSeq  = AttrVal < ',' AttrVal > .
     AttrID    = NAME .
     AttrVal   = NAME .
\end{verbatim}

There is a predefined global text attribute, the {\em language}, which
is automatically added to every Thot structure schema.  This attribute
allows the editor to perform certain actions, such as hyphenation and
spell-checking, which cannot be performed without knowing the language
in which each part of the document is written.  This attribute can be
used just like any explicitly declared attribute: the system acts as
if every structure schema contains
\begin{verbatim}
ATTR
   Language = TEXT;
\end{verbatim}

\begin{example}
The following specification defines the global enumeration attribute
WordType.

\begin{verbatim}
ATTR
   WordType    = Definition, IndexWord, DocumentTitle;
\end{verbatim}
\end{example}

\subsection{Parameters}
\label{param}

A parameter is a document element which can appear many times in the
document, but always has the same value.  This value can only be
modified in a controlled way by certain applications.  For example, in
an advertising circular, the name of the recipient may appear in the
address part and in the text of the circular.  If the recipient's name
were a parameter, it might only be able to be changed by a
``mail-merge'' application.

Parameters are not needed for every document class, but if the schema
includes parameters they are declared after the keyword {\tt PARAM}.
Each parameter declaration is made in the same way as a structure
element declaration (see page~\pageref{elemstruct}).

During editing, Thot permits the insertion of parameters wherever the
structure schema allows; it also permits the removal of parameters
which are already in the document but does not allow the modification
of the parameter's content in any way.  The content is generated
automatically by the editor during the creation of the parameter,
based on the value of the parameter in the document.

\subsection{Structured elements}
\label{elemstruct}

The rules for defining structured elements are
required{\footnote{except in an extension schema}}: they constitute
the core of a structure schema, since they define the structure of the
different types of elements that occur in a document or object of
the class defined by the schema.

The first structure rule after the keyword {\tt STRUCT} must define
the structure of the class whose name appears in the first instruction
({\tt STRUCTURE}) of the schema.  This is the root rule of the schema,
defining the root of the document tree or object tree.

The remaining rules may be placed in any order, since the language
permits the definition of element types before or after their use, or
even in the same instruction in which they are used.  This last case
allows the definition of recursive structures.

Each rule is composed of a name (the name of the element type whose
structure is being defined) followed by an equals sign and a structure
definition.

If any local attributes are associated with the element type defined
by the rule, they appear between parentheses after the type name.  The
parentheses contain, first, the keyword {\tt ATTR}, then the list of
local attributes, separated by commas.  Each local attribute is
composed of the name of the attribute followed by an equals sign and
the definition of the attribute's type, just as in the definition of
global attributes (see section~\ref{attrglobaux}).  The name of the
attribute can be preceded by an exclamation point to indicate that the
attribute must always be present for this element type.  The same
attribute, identified by its name, can be defined  as a local
attribute for multiple element types.  In this case, the equals sign
and definition of the attribute type need only appear in the first
occurrence of the attribute.  It should be noted that global attributes
cannot also be defined as local attributes.

If any extensions (see section~\ref{restrictionextensions}) are defined for this element type, a plus sign
follows the structure definition and the names of the extension
element types appear between parentheses after the plus.  If there are
multiple extensions, they are separated by commas.  These types can
either be defined in the same schema, defined in other schemas,
or they may be base types identified by the keywords {\tt TEXT}, {\tt
GRAPHICS}, {\tt SYMBOL}, or {\tt PICTURE}.

Restrictions (see section~\ref{restrictionextensions}) are indicated in the same manner as extensions, but they
are introduced by a minus sign and they come after the extensions, or
if there are no extensions, after the structure definition.

If the values of attributes must be attached systematically to this
element type, they are introduced by the keyword {\tt WITH} and
declared in the form of a list of fixed-value attributes.  When such
definitions of fixed attribute values appear, they are always the last
part of the rule.

The rule is terminated by a semicolon.

\begin{verbatim}
  RuleSeq       = Rule < Rule > .
  Rule          = ElemID [ LocAttrSeq ] '=' DefWithAttr ';'.
  LocAttrSeq    = '(' 'ATTR' LocAttr < ';' LocAttr > ')' .
  LocAttr       = [ '!' ] AttrID [ '=' AttrType ] .
  DefWithAttr   = Definition
                  [ '+' '(' ExtensionSeq ')' ]
                  [ '-' '(' RestrictSeq ')' ]
                  [ 'WITH' FixedAttrSeq ] .
  ExtensionSeq  = ExtensionElem < ',' ExtensionElem > .
  ExtensionElem = ElemID / 'TEXT' / 'GRAPHICS' /
                  'SYMBOL' / 'PICTURE' .
  RestrictSeq   = RestrictElem < ',' RestrictElem > .
  RestrictElem  = ElemID / 'TEXT' / 'GRAPHICS' /
                  'SYMBOL' / 'PICTURE' .
\end{verbatim}

The list of fixed-value attributes is composed of a sequence of
attribute-value pairs separated by commas.  Each pair contains the
name of the attribute and the fixed value for this element type, the
two being separated by an equals sign.  If the sign is preceded by a
question mark the given value is only an initial value that may be
modified later rather than a value fixed for all time.  Reference
attributes are an exception to this norm.  They cannot be assigned a
fixed value, but when the name of such an attribute appears this
indicates that this element type must have a valid value for the
attribute.  For the other attribute types, the fixed value is
indicated by an signed integer (numeric attributes), a character
string between apostrophes (textual attributes) or the name of a value
(enumeration attributes).

Fixed-value attributes can either be global (see
section~\ref{attrglobaux}) or local to the element type for which they
are fixed, but they must be declared before they are used.

\begin{verbatim}
    FixedAttrSeq    = FixedAttr < ',' FixedAttr > .
    FixedAttr       = AttrID [ FixedOrModifVal ] .
    FixedOrModifVal = [ '?' ] '=' FixedValue .
    FixedValue      = [ '-' ] NumValue / TextVal / AttrVal .
    NumValue        = NUMBER .
    TextVal         = STRING .
\end{verbatim}

\subsection{Structure definitions}
\label{defstruct}

The structure of an element type can be a simple base type or a
constructed type.

For constructed types, it is frequently the case that similar
structures appear in many places in a document.  For example the
contents of the abstract, of the introduction, and of a section can
have the same structure, that of a sequence of paragraphs.  In this
case, a single, common structure can be defined (the paragraph
sequence in this example), and the schema is written to indicate that
each element type possesses this structure, as follows:
\begin{verbatim}
     Abstract           = Paragraph_sequence;
     Introduction       = Paragraph_sequence;
     Section_contents   = Paragraph_sequence;
\end{verbatim}
The equals sign means ``has the same structure as''.

If the element type defined is a simple base type, this is indicated
by one of the keywords {\tt TEXT}, {\tt GRAPHICS}, {\tt SYMBOL}, or
{\tt PICTURE}.  If some local attributes must be associated with a base
type, the keyword of the base type is followed by the declaration of
the local attributes using the syntax presented in
section~\ref{elemstruct}.

In the case of an open choice, the type is indicated by the
keyword {\tt UNIT} for units or the keyword {\tt NATURE} for objects
having arbitrary structure.

A unit represents one of the two following categories:
\begin{itemize}
  \item a base type: text, graphical element, symbol, image,

  \item an element whose type is chosen from among the types defined
as units in the {\tt UNITS} section of the document's structure schema.
It can also be chosen from among the types defined as units in the
{\tt UNITS} section of the structure schemas that defines the
ancestors of the element to which the rule is applied (see pages~\pageref{schemasandunits} and \pageref{uniteslog}).
\end{itemize}

Before the creation of an element defined as a unit, the Thot editor
asks the user to choose between the categories of elements.

Thus, the contents of a paragraph can be specified as a sequence of
units, which will permit the inclusion in the paragraphs of character
strings, symbols, and various elements, such as cross-references, if
these are defined as units.

A schema object (keyword {\tt NATURE}) represents an object defined by
a structure schema freely chosen from among the available schemas;
in the case the element type is defined by the first rule (the root rule)
of the chosen schema.

If the element type defined is a constructed type, the list, aggregate,
choice, and reference constructors are used.  In this case the
definition begins with a keyword identifying the constructor.  This
keyword is followed by a syntax specific to each constructor.

The local attribute definitions appear after the name of the 
element type being defined, if this element type has local attributes.
The syntax for local attribute definitions is presented in section~\ref{elemstruct}.

\begin{verbatim}
   Definition = BaseType [ LocAttrSeq ] / Constr / Element .
   BaseType   = 'TEXT' / 'GRAPHICS' / 'SYMBOL' / 'PICTURE' /
                'UNIT' / 'NATURE' .
   Element    = ElemID [ ExtOrDef ] .
   ExtOrDef   = 'EXTERN' / 'INCLUDED' / 
                [ LocAttrSeq ] '=' Definition .
   Constr     = 'LIST' [ '[' min '..' max ']' ] 'OF'
                       '(' DefWithAttr ')' /
                'BEGIN' DefOptSeq 'END' /
                'AGGREGATE' DefOptSeq 'END' /
                'CASE' 'OF' DefSeq 'END' /
                'REFERENCE' '(' RefType ')' /
                'PAIR' .
\end{verbatim}

\subsubsection{List}

The list constructor permits the definition of an element type
composed of a list of elements, all of the same type.  A list
definition begins with the {\tt LIST} keyword followed by an optional
range, the keyword {\tt OF}, and the definition, between parentheses,
of the element type which must compose the list.  The optional range
is composed of the minimum and maximum number of elements for the list
separated by two periods and enclosed by brackets.  If the range is
not present, the number of list elements is unconstrained.  When only
one of the two bounds of the range is unconstrained, it is represented
by a star ('\verb|*|') character.\footnote{Even when both bounds are
unconstrained, they can be specified by {\tt [\verb|*|..\verb|*|]}, but
it is simpler not to specify any bound.}

\begin{verbatim}
               'LIST' [ '[' min '..' max ']' ]
               'OF' '(' DefWithAttr ')'
     min     = Integer / '*' .
     max     = Integer / '*' .
     Integer = NUMBER .
\end{verbatim}

Before the document is edited, the Thot editor creates the minimum
number of elements for the list.  If no minimum was given, it creates
a single element.  If a maximum number of elements is given and that
number is attained, the editor refuses to create new elements for the
list.

\begin{example}
The following two instructions define the body of a document as a
sequence of at least two chapters and the contents of a section as a
sequence of paragraphs.  A single paragraph can be the entire contents
of a section.

\begin{verbatim}
Body             = LIST [2..*] OF (Chapter);
Section_contents = LIST OF (Paragraph);
\end{verbatim}
\end{example}

\subsubsection{Aggregate}

The aggregate constructor is used to define an element type as a
collection of sub-elements, each having a fixed type.  The collection
may be ordered or unordered.  The elements composing the collection
are called {\em components}.  In the definition of an aggregate, a
keyword indicates whether or not the aggregate is ordered: {\tt BEGIN}
for an ordered aggregate, {\tt AGGREGATE} for an unordered aggregate.
This keyword is followed by the list of component type definitions which is
terminated by the {\tt END} keyword.  The component type definitions
are separated by commas.

Before creating an aggregate, the Thot editor creates all the
aggregate's components in the order they appear in the structure
schema, even for unordered aggregates.  However, unlike ordered
aggregates, the components of an unordered aggregate may be rearranged
using operations of the Thot editor.  The exceptions to the rule are
any components whose name was preceded by a question mark character
('\verb|?|').  These components, which are optional, can be created by
explicit request, possibly at the time the aggregate is created, but
they are not created automatically {\em prior} to the creation of the
aggregate.

\begin{verbatim}
                 'BEGIN' DefOptSeq 'END'
     DefOptSeq = DefOpt ';' < DefOpt ';' > .
     DefOpt    = [ '?' ] DefWithAttr .
\end{verbatim}

\begin{example}
In a bilingual document, each paragraph has an English version and
French version.  In certain cases, the translator wants to add a
marginal note, but this note is present in very few paragraphs.  Thus,
it must not be created systematically for every paragraph.  A
bilingual paragraph of this type is declared:

\begin{verbatim}
Bilingual_paragraph = BEGIN
                      French_paragraph  = TEXT;
                      English_paragraph = TEXT;
                      ? Note            = TEXT;
                      END;
\end{verbatim}
\end{example}

\subsubsection{Choice}

The choice constructor permits the definition of an element type which
is chosen from among a set of possible types.  The keywords {\tt CASE}
and {\tt OF} are followed by a list of definitions of possible types,
which are separated by semicolons and terminated by the {\tt END} keyword.

\begin{verbatim}
               'CASE' 'OF' DefSeq 'END'
     DefSeq = DefWithAttr ';' < DefWithAttr ';' > .
\end{verbatim}

Before the creation of an element defined as a choice, the Thot editor
presents the list of possible types for the element to the user.  The
user has only to select the element type that he wants to create from
this list.

The order of the type declarations is important.  It
determines the order of the list presented to the user before the
creation of the element.  Also, when a Choice element is being created
automatically, the first type in the list is used.  In fact, using the
Thot editor, when an empty Choice element is selected,
it is possible to select this element and to enter its text from keyboard.
In this case, the editor uses the first element type which can contain
an atom of the character string type.

The two special cases of the choice constructor, the {\em schema} and
the {\em unit} are discussed in sections~\ref{defstruct}
and~\ref{uniteslog}.

\begin{example}
It is common in documents to treat a variety of objects as if they
were ordinary paragraphs.  Thus, a ``Paragraph'' might actually be
composed of a block of text (an ordinary paragraph), or a mathematical
formula whose structure is defined by another structure schema named
Math, or a table, also defined by another structure schema.  Here is a
definition of such a paragraph:

\begin{verbatim}
Paragraph = CASE OF
              Simple_text = TEXT;
              Formula     = Math;
              Table_para  = Table;
              END;
\end{verbatim}
\end{example}

\subsubsection{Reference}
\label{references}

Like all elements in Thot, references are typed.  An element type
defined as a reference is a cross-reference to an element of some
other given type.  The keyword {\tt REFERENCE} is followed by the name
of a type enclosed in parentheses.  When the type which is being
cross-referenced is defined in another structure schema, the type name
is itself followed by the name of the external structure schema in
which it is defined.

When the designated element type is a mark pair (see
section~\ref{paires}), it can be preceded by a {\tt FIRST} or {\tt
SECOND} keyword.  These keywords indicate whether the reference points
to the first or second mark of the pair.  If the reference points to a
pair and neither of these two keywords is present, the reference is
considered to point to the first mark of the pair.

There is an exception to the principle of typed references:  it is
possible to define a reference which designates an element of any
type, which can either be in the same document or another document.
In this case, it suffices to put the keyword {\tt ANY} in the
parentheses which indicate the referenced element type.

\begin{verbatim}
             'REFERENCE' '(' RefType ')'
   RefType = 'ANY' / [ FirstSec ] ElemID [ ExtStruct ] .
\end{verbatim}

When defining an inclusion, the {\tt REFERENCE} keyword is not used.
Inclusions with complete expansion are not declared as such in the
structure schemas, since any element defined in a structure schema can
be replaced by an element of the same type.  Instead, inclusions
without expansion or with partial expansion must be declared
explicitly whenever they will include a complete object ( and not a
part of an object).  In this case, the object type to be included
(that is, the name of its structure schema) is followed by a keyword:
{\tt EXTERN} for inclusion without expansion and {\tt INCLUDED} for
partial expansion.

Before creating a cross-reference or an inclusion, the Thot editor
asks the user to choose, from the document images displayed, the
referenced or included element.

\begin{example}
If the types Note and Section are defined in the
Article structure schema, it is possible to define, in the same
structure schema, a reference to a note and a reference to a section
in this manner:

\begin{verbatim}
Ref_note    = REFERENCE (Note);
Ref_section = REFERENCE (Section);
\end{verbatim}

It is also possible to define the generic structure of a collection of
articles, which include (with partial expansion) objects of the 
Article class and which possess an introduction which may include
cross-references to sections of the included articles.  In the
Collection structure schema, the definitions are:

\begin{verbatim}
Collection = BEGIN
          Collection_title = TEXT;
          Introduction = LIST OF (Elem = CASE OF
                                           TEXT;
                                           Ref_sect;
                                           END);
          Body = LIST OF (Article INCLUDED);
          END;
Ref_sect = REFERENCE (Section (Article));
\end{verbatim}

Here we define a Folder document class which has a title
and includes documents of different types, particularly Folders:

\begin{verbatim}
Folder = BEGIN
          Folder_title    = TEXT;
          Folder_contents = LIST OF (Document);
          END;

Document = CASE OF
              Article EXTERN;
              Collection EXTERN;
              Folder EXTERN;
              END;
\end{verbatim}

Under this definition, Folder represents either an aggregate which
contains a folder title and the list of included documents or an
included folder.  To resolve this ambiguity, in the P language, the
placement of a star character in front of the type name (here, Folder)
indicates an included document.
\end{example}

\subsubsection{Mark pairs}
\label{paires}

Like other elements, mark pairs are typed.  The two marks of the pair
have the same type, but there exist two predefined subtypes which
apply to all mark pairs: the first mark of the pair (called {\tt First}
in the P and T languages) and the second mark (called {\tt Second}). 

In the S language, a mark pair is noted simply by the {\tt PAIR}
keyword.

In the Thot editor, marks are always moved or destroyed together.  The
two marks of a pair have the same identifier, unique within the
document, which permits intertwining mark pairs without risk of
ambiguity.

\subsection{Imports}

Because of schema constructors, it is possible, before editing a
document, to use classes defined by other structure schemas whenever
they are needed.  It is also possible to assign specific document
classes to certain element types.  In this case, these classes are
simply designated by their name.  In fact, if a type name is not
defined in the structure schema, it is assumed that it specifies a
structure defined by another structure schema.

\begin{example}

If the types Math and Table don't appear in the left part
of a structure rule in the schema, the following two rules indicate
that a formula has the structure of an object defined by the structure
schema Math and that a table element has the structure of an
object defined by the Table schema.

\begin{verbatim}
Formula = Math;
Table_elem = Table;
\end{verbatim}
\end{example}

\subsection{Extension rules}

The {\tt EXTENS} section, which can only appear in an extension
schema, defines complements to the rules in the primary schema (i.e.
the structure schema to which the extension schema will be applied).
More precisely, this section permits the addition to an existing type
of local attributes, extensions, restrictions and fixed-value
attributes.

These additions can be applied to the root rule of the primary schema,
designated by the keyword {\tt Root}, or to any other explicitly named
rule.

Extension rules are separated from each other by a semicolon and each
extension rule has the same syntax as a structure rule (see
section~\ref{elemstruct}), but the part which defines the constructor
is absent.

\begin{verbatim}
     ExtenRuleSeq = ExtensRule ';' < ExtensRule ';' > .
     ExtensRule =      RootOrElem [ LocAttrSeq ]
                        [ '+' '(' ExtensionSeq ')' ]
                        [ '-' '(' RestrictSeq ')' ]
                        [ 'WITH' FixedAttrSeq ] .
     RootOrElem =       'Root' / ElemID .
\end{verbatim}

\subsection{Associated elements}

If associated elements are necessary, they must be declared in a
specific section of the structure schema, introduced by the keyword
{\tt ASSOC}.  Each associated element type is specified like any other
structured element.  However, these types must not appear in any other
element types of the schema, except in {\tt REFERENCE} rules.

\subsection{Units}
\label{uniteslog}

The {\tt UNITS} section of the structure schema contains the
declarations of the element types which can be used in the external
objects making up parts of the document or in objects of the class
defined by the schema.  As with associated elements, these
element types are defined just like other structured element types.
They can be used in the other element types of the schema, but they can
also be used in any other rule of the schema.

\begin{example}
If references to notes are declared as units:
\begin{verbatim}
UNITS
   Ref_note = REFERENCE (Note);
\end{verbatim}
then it is possible to use references to notes in a cell of a table,
even when {\tt Table} is an external structure schema.  The {\tt
Table} schema must declare a cell to be a sequence of units, which can
then be base element types (text, for example) or references to notes
in the document.
\begin{verbatim}
Cell = LIST OF (UNITS);
\end{verbatim}
\end{example}

\subsection{Skeleton elements}
\label{squelette}

When editing a document which contains or must contain external
references to several other documents, it may be necessary to load a
large number of documents, simply to see the parts designated by the
external references of the document while editing, or to access the
source of included elements.  In this case, the external documents are
not modified and it is only necessary to see the elements of these
documents which could be referenced.  Because of this, the editor will
suggest that the documents be loaded in ``skeleton'' form.  This form
contains only the elements of the document explicitly mentioned in the
{\tt EXPORT} section of their structure schema and, for these
elements, only the part of the contents specified in that section.
This form has the advantage of being very compact, thus requiring very
few resources from the editor.  This is also the skeleton form which
constitutes the expanded form of inclusions with partial expansion
(see page~\pageref{inclusion}).

Skeleton elements must be declared explicitly in the {\tt EXPORT}
section of the structure schema that defines them.  This section
begins with the keyword {\tt EXPORT} followed by a comma-separated
list of the element types which must appear in the skeleton form and
ending with a semicolon.  These types must have been previously
declared in the schema.

For each skeleton element type, the part of the contents which is
loaded by the editor, and therefore displayable, can be specified by
putting the keyword {\tt WITH} and the name of the contained element
type to be loaded after the name of the skeleton element type.  In
this case only that named element, among all the elements contained in
the exportable element type, will be loaded.  If the {\tt WITH} is
absent, the entire contents of the skeleton element will be loaded
by the editor.  If instead, it is better that the skeleton form not
load the contents of a particular element type, the keyword
{\tt WITH} must be followed by the word {\tt Nothing}.

\begin{verbatim}
                [ 'EXPORT' SkeletonSeq ]

     SkeletonSeq = SkelElem < ',' SkelElem > ';' .
     SkelElem    = ElemID [ 'WITH' Contents ] .
     Contents    = 'Nothing' / ElemID [ ExtStruct ] .
\end{verbatim}

\begin{example}
Suppose that, in documents of the article class, the element types
Article\_title, Figure, Section, Paragraph, and Biblio should appear
in the skeleton form in order to make it easier to create external
references to them from other documents.  When loading an article in
its skeleton form, all of these element types will be loaded except
for paragraphs, but only the article title will be loaded in its
entirety.  For figures, the legend will be loaded, while for sections,
the title will be loaded, and for bibliographic entries, only the
title that they contain will be loaded.  Note that bibliographic
elements are defined in another structure schema, RefBib.  To produce
this result, the following declarations should be placed in the Article
structure schema:

\begin{verbatim}
EXPORT
   Article_title,
   Figure With Legend,
   Section With Section_title,
   Paragraph With Nothing,
   Biblio With Biblio_title(RefBib);
\end{verbatim}
\end{example}

\subsection{Exceptions}

The behavior of the Thot editor and the actions that it performs are
determined by the structure schemas.  These actions are applied to
all document and object types in accordance with their generic
structure.  For certain object types, such as tables and graphics,
these actions are not sufficient or are poorly adapted and some
special actions must be added to or substituted for certain standard
actions.  These special actions are called {\em exceptions}.

Exceptions only inhibit or modify
certain standard actions, but they can be used freely in every
structure schema.

Each structure schema can contain a section defining exceptions.  It
begins with the
keyword {\tt EXCEPT} and is composed of a sequence of exception
declarations, separated by semicolons.  Each declaration of an
exception begins with the name of an element type or attribute
followed by a colon.  This indicates the element type or
attribute to which the following exceptions apply.  When the given
element type name is a mark pair (see section~\ref{paires}), and only
in this case, the type name can be preceded by the keyword {\tt First}
or {\tt Second}, to indicate if the exceptions which follow are
associated with the first mark of the pair or the second.  In the
absence of this keyword, the first mark is used.

When placed in an extension schema (see section~\ref{schext}), the
keyword {\tt EXTERN} indicates that the type name which follows is
found in the principal schema (the schema being extended by the
extension schema).  The exceptions are indicated by a name.  They
are separated by semicolons.

\begin{verbatim}
                  [ 'EXCEPT' ExceptSeq ]

     ExceptSeq     = Except ';' < Except ';' > .
     Except        = [ 'EXTERN' ] [ FirstSec ] ExcTypeOrAttr
                     ':' ExcValSeq .
     ExcTypeOrAttr = ElemID / AttrID .
     ExcValSeq     = ExcValue < ',' ExcValue > .
     ExcValue      ='NoCut' / 'NoCreate' /
                    'NoHMove' / 'NoVMove' / 'NoMove' /
                    'NoHResize' / 'NoVResize' / 'NoResize' /
                    'NewWidth' / 'NewHeight' /
                    'NewHPos' / 'NewVPos' /
                    'Invisible' / 'NoSelect' /
                    'Hidden' / 'ActiveRef' /
                    'ImportLine' / 'ImportParagraph' /
                    'NoPaginate' / 'ParagraphBreak' /
                    'HighlightChildren' / 'ExtendedSelection' .
\end{verbatim}

The following are the available exceptions:

\begin{description}
\item[{\tt NoCut}]: This exception can only be applied to element
types.  Elements of a type to which this exception is applied cannot
be destroyed by the editor.  

\item[{\tt NoCreate}]: This exception can only be applied to element
types.  Elements of a type to which this exception is applied cannot
be created by ordinary commands for creating new elements.  These
elements are usually created by special actions associated with other
exceptions. 

\item[{\tt NoHMove}]: This exception can only be applied to element
types.  Elements of a type to which this exception is applied cannot
be moved horizontally with the mouse.

\item[{\tt NoVMove}]: This exception can only be applied to element
types.  Elements of a type to which this exception is applied cannot
be moved vertically with the mouse.

\item[{\tt NoMove}]: This exception can only be applied to element
types.  Elements of a type to which this exception is applied cannot
be moved in any direction with the mouse.

\item[{\tt NoHResize}]: This exception can only be applied to element
types.  Elements of a type to which this exception is applied cannot
be resized horizontally with the mouse.

\item[{\tt NoVResize}]: This exception can only be applied to element
types.  Elements of a type to which this exception is applied cannot
be resized vertically with the mouse.

\item[{\tt NoResize}]: This exception can only be applied to element
types.  Elements of a type to which this exception is applied cannot
be resized in any direction with the mouse.

\item[{\tt NoSelect}]: This exception can only be applied to element
types.  Elements of a type to which this exception is applied cannot
be selected directly with the mouse, but they can be selected by other
methods provided by the editor.

\item[{\tt NewWidth}]: This exception can only be applied to numeric
attributes.  If the width of an element which has this attribute is
modified with the mouse, the value of the new width will be assigned
to the attribute.

\item[{\tt NewHeight}]: This exception can only be applied to numeric
attributes.  If the height of an element which has this attribute is
modified with the mouse, the value of the new height will be assigned
to the attribute.

\item[{\tt NewHPos}]: This exception can only be applied to numeric
attributes.  If the horizontal position of an element which has this
attribute is modified with the mouse, the value of the new horizontal
position will be assigned to the attribute.

\item[{\tt NewVPos}]: This exception can only be applied to numeric
attributes.  If the vertical position of an element which has this
attribute is modified with the mouse, the value of the new vertical
position will be assigned to the attribute.

\item[{\tt Invisible}]:  This exception can only be applied to
attributes, but can be applied to all attribute types.  It indicates
that the attribute must not be seen by the user and that its value
must not be changed directly.  This exception is usually used when
another exception manipulates the value of an attribute.

\item[{\tt Hidden}]:  This exception can only be applied to element
types.  It indicates that elements of this type, although present in
the document's structure, must not be shown to the user of the editor.
In particular, the creation menus must not propose this type and the
selection message must not pick it.

\item[{\tt ActiveRef}]: This exception can only be applied to
attributes of the reference type.  It indicates that when the user of
the editor makes a double click on an element which possesses a
reference attribute having this exception, the element designated by
the reference attribute will be selected.

\item[{\tt ImportLine}]: This exception can only be applied to element
types.  It indicates that elements of this type should receive the
content of imported text files.  An element is created for each line
of the imported file.  A structure schema cannot contain several
exceptions {\tt ImportLine} and, if it contains one, it should not
contain any exception {\tt ImportParagraph}.

\item[{\tt ImportParagraph}]: This exception can only be applied to element
types.  It indicates that elements of this type should receive the
content of imported text files.  An element is created for each paragraph
of the imported file.  A paragraph is a sequence of lines without any
empty line.  A structure schema cannot contain several exceptions
{\tt ImportParagraph} and, if it contains one, it should not contain any
exception {\tt ImportLine}.

\item[{\tt NoPaginate}]: This exception can only be applied to the root
element, i.e. the name that appear after the keyword {\tt STRUCTURE} at the
beginning of the structure schema.  It indicates that the editor should not
allow the user to paginate documents of that type.

\item[{\tt ParagraphBreak}]: This exception can only be applied to element
types.  When the caret is within an element of a type to which this exception
is applied, it is that element that will be split when the user hits the
Return key.

\item[{\tt HighlightChildren}]: This exception can only be applied to element
types.  Elements of a type to which this exception is applied are not
highlighted themselves when they are selected, but all their children are
highlighted instead.

\item[{\tt ExtendedSelection}]: This exception can only be applied to element
types.  The selection extension command (middle button of the mouse) only add
the clicked element (if it has that exception) to the current selection,
without selecting other elements between the current selection and the
clicked element.

\end{description}

\begin{example}
Consider a structure schema for object-style graphics which defines
the Graphic\_object element type with the associated Height and Weight numeric
attributes.  Suppose that we want documents of this class to have the
following qualities:
\begin{itemize}

\item Whenever the width or height of an object is changed using the
mouse, the new values are stored in the object's Width and
Height attributes.

\item The user should not be able to change the values of the
Width and Height attributes via the Attributes menu of
the Thot editor.
\end{itemize}
The following exceptions will produce this effect.
\begin{verbatim}
STRUCT
...
   Objet_graphique (ATTR Height = Integer; Width = Integer)
       = GRAPHICS with Height ?= 10, Width ?= 10;
...
EXCEPT
   Height: NewHeight, Invisible;
   Width: NewWidth, Invisible;
\end{verbatim}
\end{example}

\section{Some examples}

In order to illustrate the principles of the document model and the
syntax of the S language, this section presents two examples of
structure schemas.  One defines a class of documents, the other
defines a class of objects.

\subsection{A class of documents: articles}

This example shows a possible structure for articles published in a
journal.  Text between braces is comments.

\begin{verbatim}
STRUCTURE Article;  { This schema defines the Article class }
DEFPRES ArticleP;   { The default presentation schema is
                      ArticleP }
ATTR                { Global attribute definitions }
   WordType = Definition, IndexWord, DocumentTitle;
   { A single global attribute is defined, with three values }
STRUCT              { Definition of the generic structure }
   Article = BEGIN  { The Article class has an aggregate
                      structure }
             Title = BEGIN   { The title is an aggregate }
                     French_title = 
                         Text WITH Language='Fran\347ais';
                     English_title =
                         Text WITH Language='English';
                     END;
             Authors = 
               LIST OF (Author
                 (ATTR Author_type=principal,secondary)
                 { The Author type has a local attribute }
                 = BEGIN
                   Author_name = Text;
                   Info = Paragraphs ;
                   { Paragraphs is defined later }
                   Address    = Text;
                   END
                 );
             Keywords = Text;
             { The journal's editor introduces the article
               with a short introduction, in French and
               in English }
             Introduction = 
                 BEGIN
                 French_intr  = Paragraphs WITH
                                Language='Fran\347ais';
                 English_intr = Paragraphs WITH
                                Language='English';
                 END;
             Body = Sections; { Sections are defined later }
		   { Appendixes are only created on demand }
           ? Appendices = 
                 LIST OF (Appendix =
                          BEGIN
                          Appendix_Title    = Text;
                          Appendix_Contents = Paragraphs;
                          END
                         );
             END;      { End of the Article aggregate }

    Sections = LIST [2..*] OF (
                 Section = { At least 2 sections }
                 BEGIN
                 Section_title   = Text;
                 Section_contents =
                   BEGIN
                   Paragraphs;
                   Sections; { Sections at a lower level }
                   END;
                 END
                 );

    Paragraphs = LIST OF (Paragraph = CASE OF
                               Enumeration = 
                                   LIST [2..*] OF
                                       (Item = Paragraphs);
                               Isolated_formula = Formula;
                               LIST OF (UNIT);
                               END
                          );

ASSOC         { Associated elements definitions }

   Figure = BEGIN
            Figure_legend  = Text;
            Illustration   = NATURE;
            END;

   Biblio_citation = CASE OF
                        Ref_Article =
                           BEGIN
                           Authors_Bib   = Text;
                           Article_Title = Text;
                           Journal       = Text;
                           Page_Numbers  = Text;
                           Date          = Text;
                           END;
                        Ref_Livre =
                           BEGIN
                           Authors_Bib; { Defined above }
                           Book_Title   = Text;
                           Editor       = Text;
                           Date;        { Defined above }
                           END;
                       END;

   Note =  Paragraphs - (Ref_note);

UNITS      { Elements which can be used in objects }

   Ref_note    = REFERENCE (Note);
   Ref_biblio  = REFERENCE (Biblio_citation);
   Ref_figure  = REFERENCE (Figure);
   Ref_formula = REFERENCE (Isolated_formula);

EXPORT     { Skeleton elements }

   Title,
   Figure with Figure_legend,
   Section With Section_title;

END           { End of the structure schema }
\end{verbatim}

This schema is very complete since it defines both paragraphs and
bibliographic citations.  These element types could just as well be
defined in other structure schemas, as is the case with the {\tt
Formula} class.  All sorts of other elements can be inserted into an
article, since a paragraph can contain any type of unit.  Similarly,
figures can be any class of document or object that the user chooses.

Generally, an article doesn't contain appendices, but it is possible
to add them on explicit request:  this is the effect of the question
mark before the word Appendices.

The Figure, Biblio\_citation and Note elements are associated
elements.  Thus, they are only used in {\tt REFERENCE} statements.

Various types of cross-references can be put in paragraphs.  They can
also be placed the objects which are part of the article, since the
cross-references are defined as units ({\tt UNITS}).

There is a single restriction to prevent the creation of
Ref\_note elements within notes.

It is worth noting that the S language permits the definition of
recursive structures like sections: a section can contain other
sections (which are thus at the next lower level of the document
tree).  Paragraphs are also recursive elements, since a paragraph can
contain an enumeration in which each element ({\tt Item}) is composed
of paragraphs.

\subsection{A class of objects: mathematical formulas}

The example below defines the {\tt Formula} class which is used in
Article documents.  This class represents mathematical formulas  with
a rather simple structure, but sufficient to produce a correct rendition
on the screen or printer.  To support more elaborate operations
(formal or numeric calculations), a finer structure should be defined.
This class doesn't use any other class and doesn't define any
associated elements or units.

\begin{verbatim}
STRUCTURE Formula;
DEFPRES FormulaP;

ATTR
   String_type = Function_name, Variable_name;

STRUCT
   Formula      = Expression;
   Expression   = LIST OF (Construction);
   Construction = CASE OF
                  TEXT;         { Simple character string }
                  Index    = Expression;
                  Exponent = Expression;
                  Fraction =
                        BEGIN
                        Numerator  = Expression;
                        Denominator = Expression;
                        END;
                  Root = 
                        BEGIN
                      ? Order = TEXT;
                        Root_Contents = Expression;
                        END;
                  Integral =
                        BEGIN
                        Integration_Symbol = SYMBOL;
                        Lower_Bound        = Expression;
                        Upper_Bound        = Expression;
                        END;
                  Triple =
                        BEGIN
                        Princ_Expression = Expression;
                        Lower_Expression = Expression;
                        Upper_Expression = Expression;
                        END;
                  Column = LIST [2..*] OF 
                              (Element = Expression);
                  Parentheses_Block =
                        BEGIN
                        Opening  = SYMBOL;
                        Contents = Expression;
                        Closing  = SYMBOL;
                        END;
                  END;       { End of Choice Constructor }
END                          { End of Structure Schema }
\end{verbatim}

This schema defines a single global attribute which allows functions
and variables to be distinguished.  In the presentation schema, this
attribute can be used to choose between roman (for functions) and
italic characters (for variables).

A formula's structure is that of a mathematical expression, which is
itself a sequence of mathematical constructions.  A mathematical
construction can be either a simple character string, an index, an
exponent, a fraction, a root, etc.  Each of these mathematical
constructions has a sensible structure which generally includes one or
more expressions, thus making the formula class's structure definition
recursive.

In most cases, the roots which appear in the formulas are square roots
and their order (2) is not specified.  This is why the Order
component is marked optional by a question mark.  When explicitly
requested, it is possible to add an order to a root, for example for
cube roots (order = 3).

An integral is formed by an integration symbol, chosen by the user
(simple integral, double, curvilinear, etc.), and two bounds.  A
more fine-grained schema would add components for the integrand and
the integration variable.  Similarly, the Block\_Parentheses
construction leaves the choice of opening and closing symbols to the
user.  They can be brackets, braces, parentheses, etc.

\chapter{The P Language}

\section{Document presentation}

Because of the model adopted for Thot, the presentation of documents
is clearly separated from their structure and content.  After having
presented the logical structure of documents, we now detail the
principles implemented for their presentation.  The concept of {\em
presentation} encompasses what typographers call the page layout, the
composition, and the model of the document.  It is the set of
operations which display the document on the screen or print it on
paper.  Like logical structure, document presentation is defined
generically with the help of a language, called P.

\subsection{Two levels of presentation}

The link between structure and presentation is clear: the logical
organization of a document is used to carry out its presentation,
since the purpose of the presentation is to make evident the
organization of the document.  But the presentation is equally
dependent on the device used to render the document.  Certain
presentation effects, notably changes of font or character set, cannot
be performed on all printers or on all screens.  This is why Thot uses
a two-level approach, where the presentation is first described in
abstract terms, without taking into account each particular device,
and then the presentation is realized within the constraints of a
given device.

Thus, presentation is only described as a function of the structure of
the documents and the image that would be produced on an idealized
device.  For this reason, presentation descriptions do not refer to
any device characteristics: they describe {\em abstract presentations}
which can be concretized on different devices.

A presentation description also defines a {\em generic presentation},
since it describes the appearance of a class of documents or objects.
This generic presentation must also be applied to document and object
instances, each conforming to its generic logical structure, but with
all the allowances that were called to mind above: missing elements,
constructed elements with other logical structures, etc.

In order to preserve the homogeneity between documents and objects,
presentation is described with a single set of tools which support the
layout of a large document as well as the composition of objects like
a graphical figure or mathematical formula.  This unity of
presentation description tools contrasts with the traditional
approach, which focuses more on documents than objects and thus is
based on the usual typographic conventions, such as the placement of
margins, indentations, vertical spaces, line lengths, justification,
font changes, etc.

\subsection{Boxes}

To assure the homogeneity of tools, all presentation in Thot, for
documents as well as for the objects which they contain, is based on
the notion of the {\em box}, such as was implemented in {\TeX}.

Corresponding to each element of the document is a box,  which is the
rectangle enclosing the element on the display device (screen or sheet
of paper);  the outline of this rectangle is normally not visible.
The sides of the box are parallel to the sides of the screen or the
sheet of paper.  By way of example, a box is associated with a
character string, a line of text, a page, a paragraph, a title, a
mathematical formula, or a table cell.

Whatever element it corresponds to, each box possesses four sides and
four axes, which we designate as follows (see figure~\ref{boite}):

\begin{description}
\item[ {\tt  Top} ]~: the upper side,
\item[ {\tt  Bottom} ]~: the lower side,
\item[ {\tt  Left} ]~: the left side,
\item[ {\tt  Right} ]~: the right side,
\item[ {\tt  VMiddle} ]~: the vertical axis passing through the center
of the box,
\item[ {\tt  HMiddle} ]~: the horizontal axis passing through the center
of the box,
\item[ {\tt  VRef} ]~: the vertical reference axis,
\item[ {\tt  HRef} ]~: the horizontal reference axis.
\end{description}

\begin{figure}
\begin{center}
\setlength{\unitlength}{1 mm}
\begin{picture}(105,55)
\thicklines
\put(26,4){\framebox(72,42)}
\thinlines
\put(19,46){\makebox(0,0)[r]{Top}}
\multiput(20,46)(2,0){3}{\line(1,0){1}}
\put(19,25){\makebox(0,0)[r]{HMiddle}}
\multiput(20,25)(2,0){42}{\line(1,0){1}}
\put(19,13){\makebox(0,0)[r]{HRef}}
\multiput(20,13)(2,0){42}{\line(1,0){1}}
\put(19,4){\makebox(0,0)[r]{Bottom}}
\multiput(20,4)(2,0){3}{\line(1,0){1}}
\put(26,54){\makebox(0,0)[t]{Left}}
\multiput(26,46)(0,2){2}{\line(0,1){1}}
\put(62,54){\makebox(0,0)[t]{VMiddle}}
\multiput(62,0)(0,2){25}{\line(0,1){1}}
\put(85,54){\makebox(0,0)[t]{VRef}}
\multiput(85,0)(0,2){25}{\line(0,1){1}}
\put(98,54){\makebox(0,0)[t]{Right}}
\multiput(98,46)(0,2){2}{\line(0,1){1}}
\end{picture}
\end{center}
\caption{The sides and axes of boxes}
\label{boite}
\end{figure}

The principal role of boxes is to set the extent and position of the
images of the different elements of a document with respect to each
other on the reproduction device.  This is done by defining relations
between the boxes of different elements which give relative extents and
positions to these boxes.

There are three types of boxes:
\begin{itemize}
  \item boxes corresponding to structural elements of the document,
  \item presentation boxes,
  \item page layout boxes.
\end{itemize}

{\bf Boxes corresponding to structural elements of the document} are
those which linked to each of the elements (base or structured) of
the logical structure of the document.  Such a box contains all the
contents of the element to which it corresponds.  These boxes form a
tree-like structure, identical to that of the structural elements to
which they correspond.  This tree expresses the inclusion relationships
between the boxes: a box includes all the boxes of its subtree.  On
the other hand, there are no predefined rules for the relative
positions of the included boxes.  If they are at the same level, they
can overlap, be contiguous, or be disjoint.  The rules expressed in
the generic presentation specify their relative positions.

{\bf Presentation boxes} represent elements which are not found in the
logical structure of the document but which are added to meet the
needs of presentation.  These boxes are linked to the elements of the
logical structure that are best suited to bringing them out (???).
For example, they are used to add the character string ``Summary:''
before the summary in the presentation of a report or to represent the
fraction bar in a formula, or also to make the title of a field in a
form appear.  These elements have no role in the logical structure of
the document: the presence of a Summary element in the document does
not require the creation of another structural object to hold the word
``Summary''. Similarly, if a Fraction element contains both a
Numerator element and a Denominator element, the fraction bar has no
purpose structurally.  On the other hand, these elements of the
presentation are important for the reader of the reproduced document
or for the user of an editor.  This is why they must appear in the
document's image.  It is the generic presentation which specifies the
presentation boxes to add by indicating their content (a base element
for which the value is specified) and the position that they must take
in the tree of boxes.  During editing, these boxes cannot be modified
by the user.

{\bf Page layout boxes} are boxes created implicitly by the page
layout rules.  These rules indicate how the contents of a structured
element must be broken into lines and pages.  In contrast to
presentation boxes, these line and page boxes do not depend on the
logical structure of the document, but rather on the physical
constraints of the reproduction services: character size, height and
width of the window on the screen or of the sheet of paper.

\subsection{Views and visibility}

One of the operations that one might wish to perform on a document is
to view it is different ways.  For this reason, it is possible to
define several {\em views} for the same document, or better yet, for
all documents of the same class.  A view is not a different
presentation of the document, but rather a filter which only allows
the display of certain parts of the document.  For example, it might
be desirable to see only the titles of chapters and sections in order
to be able to move rapidly through the document.  Such a view could be
called a ``table of contents''.  It might also be desirable to see
only the mathematical formulas of a document in order to avoid being
distracted by the non-mathematical aspects of the document.  A
``mathematics'' view could provide this service.

Views, like presentation, are based on the generic logical structure.
Each document class, and each generic presentation, can be provided
with views which are particularly useful for that class or
presentation.  For each view, the {\em visibility} of elements is
defined, indicated whether or not the elements must be presented to
the user.  The visibility is calculated  as a function of the type of
the elements or their hierarchical position in the structure of the
document.  Thus, for a table of contents, all the ``Chapter Title''
and ``Section Title'' elements are made visible.  However, the
hierarchical level could be used to make the section titles invisible
below a certain threshold level.  By varying this threshold, the
granularity of the view can be varied.  In the ``mathematics'' view,
only Formula elements would be made visible, no matter what their
hierarchical level.

Because views are especially useful for producing a synthetic image of
the document, it is necessary to adapt the presentation of the
elements to the view in which they appear.  For example, it is
inappropriate to have a page break before every chapter title in the
table of contents.  Thus, generic presentations take into account the
possible views and permit each element type's presentation to vary
according the view in which its image appears.
\label{lesvues}

Views are also used, when editing documents, to display the associated
elements.  So, in addition to the primary view of the document, there
can be a ``notes'' view and a ``figures'' view which contain,
respectively, the associated elements of the Note and Figure types.
In this way, it is possible to see simultaneously the text which
refers to these elements and the elements themselves, even if they
will be separated when printed.

\subsection{Pages}

Presentation schemas can be defined which display the document as a
long scroll, without page breaks.  This type of schema is particularly
well-suited to the initial phase of work on a document, where jumps
from page to page would hinder composing and reading the document on a
screen.  In this case, the associated elements (such as notes), which
are normally displayed in the page footer, are presented in a separate
window.  But, once the document is written, it may be desirable to
display the document on the screen in the same manner in which it will
be printed.  So, the presentation schema must define pages.

The P language permits the specification of the dimensions of pages as
well as their composition.  It is possible to generate running titles,
page numbers, zones at the bottom of the page for notes, etc.  The
editor follows this model and inserts page break marks in the document
which are used during printing, insuring that the pages on paper are
the same as on the screen.

Once a document has been edited with a presentation schema defining
pages, it contains page marks.  But it is always possible to edit the
document using a schema without pages.  In this case, the page marks
are simply ignored by the editor.  They are considered again as soon
as a schema with pages is used.  Thus, the user is free to choose
between schemas with and without pages.

Thot treats the page break, rather than the page itself, as a box.
This page break box contains all the elements of one page's footer, a
rule marking the edge of this page, and all the elements of the next
page's header.  The elements of the header and footer can be running
titles, page number, associated elements (notes, for example), etc.
All these elements, as well as their content and graphical appearance,
are defined by the generic presentation.

\subsection{Numbering}

Many elements are numbered in documents: pages, chapters, sections,
formulas, theorems, notes, figures, bibliographic references,
exercises, examples, lemmas, etc.  Because Thot has a notion of
logical structure, all of these numbers (with the exception of pages)
are redundant with information implicit in the logical structure of
the document.  Such numbers are simply a way to make the structure of
the document more visible.  So, they are part of the document's
presentation and are calculated by the editor from the logical
structure.  The structure does not contain numbers as such; it only
defines relative structural positions between elements, which serve as
ordering relations on these elements.

If the structure schema defines the body of a document as a sequence
of at least two chapters:
\begin{verbatim}
Body = LIST [2..*] OF Chapter ;
\end{verbatim}
the sequence defined by the list constructor is ordered and each
chapter can be assigned a number based on its rank in the Body list.
Therefore, all elements contained in lists a the structure of a
document can be numbered, but they are not the only ones.  The tree
structure induced by the aggregate, list, and choice constructors
(excluding references) defines a total order on the elements of the
document's primary structure.  So, it is possible to define a
numbering which uses this order, filtering elements according to their
type so that only certain element types are taken into account in the
numbering.  In this way, it possible to number all the theorems and
lemmas of a chapter in the same sequence of numbers, even when they
are not part of the same list constructor and appear at different
levels of the document's tree.  By changing the filter, they can be
numbered separately: one sequence of numbers for theorems, another for
the lemmas.

Associated elements pose a special problem, since they are not part of
the document's primary structure, but are attached only by references,
which violate the total order of the document.  Then, these associated
elements are frequently numbered, precisely because the number is an
effective way to visualize the reference.  In order to resolve this
problem, Thot implicitly defines a list constructor for each type
of associated element, gathering together (and ordering) these
elements.  Thus, the associated elements can be numbered by type.

Since they are calculated from the document's logical structure and
only for the needs of the presentation, numbers are presentation
elements, described by presentation boxes, just like the fraction bar
or the word ``Summary''.  Nevertheless, numbers differ from these
other boxes because their content varies from instance to instance,
even though they are of the same type, whereas all fraction bars are
horizontal lines and the same word ``Summary'' appears at the head of
every document's summary.

\subsection{Presentation parameters}

The principal parameters which determine document presentation are the
{\em positions} and {\em dimensions} of boxes, the {\em font}, the
{\em style}, the {\em size}, the {\em underlining} and the {\em color}
of their content.  From these parameters, and some others of less
importance, it is possible to represent the usual typographic
parameters for the textual parts of the document.  These same
parameters can be used to describe the geometry of the non-textual
elements, even though they are two-dimensional elements unlike the
text, which is linear.

As we have already  seen, the positions of the boxes always respect
the rule of enclosure: a box in the tree encloses all the boxes of the
next lower level which are attached to it.  The positional parameters
permit the specification of the position of each box in relation to
the enclosing box or to its sibling boxes (boxes directly attached to
the same enclosing box in the tree of boxes).

The presentation parameters also provide control over the dimensions
of the boxes.  The dimensions of a box can depend either on its
content or on its context (its sibling boxes and the enclosing box).
Each dimension (height or width) can be defined independently of the
other.

Because of the position and dimension parameters, it is possible to do
the same things that are normally done in typography by changing
margins, line lengths, and vertical or horizontal skips.  This
approach can also align or center elements and groups of elements.

In contrast to the position and dimension parameters, the font, style,
size, underlining, and color do not concern the box itself (the
rectangle delimiting the element), but its content.  These parameters
indicate the typographic attributes which must be applied to the text
contained in the box, and by extension, to all base elements.

For text, the font parameter is used to change the family of
characters (Times, Helvetica, Courier, etc.); the style is used to
obtain italic or roman, bold or light characters;  the size determines
the point size of the characters; underlining defines the type and
thickness of the lines drawn above, below, or through the characters.

For graphics, the line style parameter can be either solid, dotted, or
dashed;  the line thickness parameter controls the width of the lines;
the fill pattern parameter determines how closed geometric figures
must be filled.

While some of the parameters which determine the appearance of a box's
contents make sense only for one content type (text or graphic), other
parameters apply to all content types: these are the color parameters.
These indicate the color of lines and the background color.

\section{Presentation description language}
\label{langp}

A generic presentation defines the values of presentation parameters
(or the way to calculate those values) for a generic structure, or
more precisely, for all the element types and all the global and local
attributes defined in that generic structure.  This definition of the
presentation parameters is made with the P language.  A program
written in this language, that is a generic presentation expressed in
P, is call a {\em presentation schema}.  This section describes the
syntax and semantics of the language, using the same meta-language as
was used for the definition of the S language (see
page~\pageref{metalang}).

Recall that it is possible to write many different presentation
schemas for the same class of documents or objects.  This allows users
to choose for a document the graphical appearance  which best suits
their type of work or their personal taste.

\subsection{The organization of a presentation schema}

A presentation schema begins with the word {\tt PRESENTATION} and ends
with the word {\tt END}.  The word {\tt PRESENTATION} is followed by
the name of the generic structure to which the presentation will be
applied.  This name must be the same as that which follows the
keyword {\tt STRUCTURE} in the structure schema associated with the
presentation schema.

After this declaration of the name of the structure, the following
sections appear (in order):

\begin{itemize}
\item
Declarations
   \begin{itemize}
   \item
   views,
   \item
   printed views,
   \item
   counters,
   \item
   presentation constants,
   \item
   variables,
   \end{itemize}
\item
default presentation rules,
\item
presentation box and page layout box definitions,
\item
presentation rules for structured elements,
\item
presentation rules for attributes,
\item
rules for transmitting values to attributes of included documents.
\end{itemize}

Each of these sections is introduced by a keyword which is followed by
a sequence of declarations.  Every section is optional.

\begin{verbatim}
     SchemaPres ='PRESENTATION' ElemID ';'
               [ 'VIEWS' ViewSeq ]
               [ 'PRINT' PrintViewSeq ]
               [ 'COUNTERS' CounterSeq ]
               [ 'CONST' ConstSeq ]
               [ 'VAR' VarSeq ]
               [ 'DEFAULT' ViewRuleSeq ]
               [ 'BOXES' BoxSeq ]
               [ 'RULES' PresentSeq ]
               [ 'ATTRIBUTES' PresAttrSeq ]
               [ 'TRANSMIT' TransmitSeq ]
                 'END' .
     ElemID  = NAME .
\end{verbatim}

\subsection{Views}

Each of the possible views must be declared in the presentation
schema.  As has already been described on page~\pageref{lesvues}, the
presentation rules for an element type can vary according to the view
in which the element appears.  The name of the view is used to
designate the view to which the presentation rules apply (see the {\tt
IN} instruction, page~\pageref{instrin}).  The definition of the
view's contents are dispersed throughout the presentation rules
attached to the different element types and attributes.  The {\tt
VIEWS} section is simply a sequence of view names separated by commas
and terminated by a semi-colon.

One of the view names (and only one) can be followed by the keyword
{\tt EXPORT}.  This keyword identifies the view which presents the
members of the document class in skeleton form (see
section~\ref{squelette}).  The graphical appearance  and the content
of this view is defined just as with other views, but it is useless
to specify presentation rules concerning this view for the elements
which are not loaded in the skeleton form.

It is not necessary to declare any views; in this case there is a
single unnamed view.  If many views are declared, the first view
listed is considered the principal view.  The principal view is the
one to which all rules that are not preceded by an indication of a
view will apply (see the instruction {\tt IN},
page~\pageref{instrin}).

The principal view is the the one which the editor presents on the
screen when the user asks to create or edit a document.  Thus, it
makes sense to put the most frequently used view at the head of the
list.  But if the structure schema contains skeleton elements (see
section~\ref{squelette}) and is loaded in its skeleton form, the view
whose name is followed by the keyword {\tt EXPORT} will be opened and no
other views can be opened.

\begin{verbatim}
                     'VIEWS' ViewSeq
     ViewSeq         = ViewDeclaration
                      < ',' ViewDeclaration > ';' .
     ViewDeclaration = ViewID [ 'EXPORT' ] .
     ViewID          = NAME .
\end{verbatim}

\begin{example}
When editing a report, it might be useful have views of the table of
contents and of the mathematical formulas, in addition to the
principal view which shows the document in its entirety.  To achieve
this, a presentation schema for the Report class would have the
following {\tt VIEWS} section:

\begin{verbatim}
VIEWS
     Full_text, Table_of_contents, Formulas;
\end{verbatim}

The contents of these views are specified in the presentation rules of
the schema.
\end{example}

\subsection{Print Views}

When editing a document, each view is presented in a different
window.  In addition to the views specified by the {\tt VIEWS}
instruction, the user can display the associated elements with one
window for each type of associated element.

When printing a document, it is possible to print any number of views,
chosen from among all the views which the editor can display (views in
the strict sense or associated elements).  Print views, as well as the
order in which they must be printed, are indicated by the {\tt PRINT}
instruction.  It appears after the {\tt VIEWS} instruction and is
formed of the keyword {\tt PRINT} followed by the ordered list of
print view names.  The print view names are separated by commas and
followed by a semi-colon.  A print view name is either a view name
declared in the {\tt VIEWS} instruction or the name of an associated
element type (with an ``s'' added to the end).  The associated element
must have been declared in the {\tt ASSOC} section of the structure
schema.

\begin{verbatim}
                    'PRINT' PrintViewSeq
     PrintViewSeq = PrintView < ',' PrintView > ';' .
     PrintView    = ViewID / ElemID .
\end{verbatim}

If the {\tt PRINT} instruction is absent, the printing program will
print only the principal view (the first view specified by the {\tt
VIEWS} instruction or the single, unnamed view when there is no {\tt
VIEWS} instruction).

\begin{example}
Consider a Report presentation using the view declarations from the
preceding example.  Suppose we want to print the full text and table
of contents views, but not the Formulas view, which is only useful
when editing.  In addition, suppose that we also want to print the
bibliographic citations, which are associated elements (of type {\tt
Citation}).  A sensible printing order would be to print the full text
then the bibliography and finally the table of contents.  To obtain
this result when printing, the presentation schema would say:

\begin{verbatim}
PRINT
     Full_text, Citations, Table_of_contents;
\end{verbatim}
\end{example}

\subsection{Counters}

A presentation has a {\em counter} for each type of number in the
presentation.  All counters, and therefore all types of numbers, used
in the schema must be declared after the {\tt COUNTERS} keyword.

Each counter declaration is composed of a name identifying the
counter followed by a colon and the counting function to be applied to
the counter.  The counter declaration ends with a semi-colon.

The counting function indicates how the counter values will be
calculated.  Three types of counting functions are available.  The first
type is used to count the elements of a list or aggregate: it assigns
to the counter the rank of the element in the list or aggregate.  More
precisely, the function 
\begin{verbatim}
RANK OF ElemID [ LevelAsc ] [ INIT AttrID ]
        [ 'REINIT' AttrID ]
\end{verbatim}
indicates that when an element creates, by a creation rule (see the
{\tt Create} instructions, page~\pageref{creation}), a presentation
box containing  the counter value, this value is the rank of the
creating element, if it is of type {\tt ElemID}, otherwise the rank of
the first element of type {\tt ElemID} which encloses the creating
element in the logical structure of the document.

The type name can be preceded by a star in the special case where the
structure schema defines an element of whose {\tt ElemID} is the same
as that of an inclusion without expansion or with partial expansion
see section~\ref{references}.  To resolve this ambiguity, the {\tt
ElemID} alone refers to the type defined in the structure schema while
the {\tt ElemID} preceded by a star refers to the included type.

The type name {\tt ElemID} can be followed by an integer.  That number
represents the relative level, among the ancestors of the creating element,
of the element whose rank is asked.  If that relative level $n$ is
unsigned, the $n^th$ element of type {\tt ElemID} encountered when
travelling the logical structure from the root to the creating element
is taken into account.  If the relative level is negative, the logical
structure is travelled in the other direction, from the creating element
to the root.

The function can end with the keyword {\tt INIT} followed by the name
of a numeric attribute (and only a numeric attribute).  Then, the rank
of the first element of the list or aggregate is considered to be the
value of this attribute, rather than the default value of 1, and the
rank of the other elements is shifted accordingly.  The attribute
which determines the initial value is searched on the element itself
and on its ancestors.

The function can end with the keyword {\tt REINIT} followed by the name
of a numeric attribute (and only a numeric attribute).  Then, if an
element to be counted has this attribute, the counter value for this
element is the attribute value and the following elements are numbered
starting from this value.

When the {\tt RANK} function is written
\begin{verbatim}
RANK OF Page [ ViewID ] [ INIT AttrID ]
\end{verbatim}
({\tt Page} is a keyword of the P language), the counter takes as its
value the number of the page on which the element which creates the
presentation box containing the number appears.  This is done as if
the pages of the document form a list for each view.  The counter only
takes into account the pages of the relevant view, that is the view
displaying the presentation box whose contents take the value of the
number.  However, if the keyword {\tt Page} is followed by the name of a
view (between parentheses), it is the pages of that view that are
taken into account.  As in the preceding form, the {\tt RANK} function
applied to pages can end with the {\tt INIT} keyword followed the name
of a numeric attribute which sets the value of the first page's
number.  This attribute must be a local attribute of the document
itself, and not of one of its components.

The second counting function is used to count the occurrences of a
certain element type in a specified context.  The instruction
\begin{verbatim}
SET n ON Type1 ADD m ON Type2 [ INIT AttrID ]
\end{verbatim}
says that when the document is traversed from beginning to end (in the
order induced by the logical structure), the counter is assigned the
value {\tt n} each time an element of type {\tt Type1} is encountered, no
matter what the current value of the counter, and the value {\tt m} is
added to the current value of the counter each time an element of type
{\tt Type2} is encountered.

As with the {\tt RANK} function, the type names can be preceded by a
star to resolve the ambiguity of included elements.

If the function ends with the keyword {\tt INIT} followed by the name
of an attribute and if the document possesses this attribute, the
value of this attribute is used in place of {\tt n}.  The attribute
must be numeric.  It is searched on the element itself and on its ancestors.

This function can also be used with the {\tt Page} keyword in the
place of {\tt Type1} or {\tt Type2}.  In the first case, the counter
is reinitialized on each page with the value {\tt n}, while in the
second case, it is incremented by {\tt m} on each page.  As with the
preceding counting function, the word {\tt Page} can be followed by a
name between parentheses.  In this case, the name specifies a view
whose pages are taken into account.

The definition of a counter can contain several {\tt SET} functions
and several {\tt ADD} functions, each with a different value.  The
total number of counting functions must not be greater than 6.

The third counting function is used to count the elements of a
certain type encountered when travelling from the creating element
to the root of the logical structure.  The creating element is
included if it is of that type.  That function is written
\begin{verbatim}
RLEVEL OF Type
\end{verbatim}
where {\tt Type} represents the type of the elements to be counted.

The formal definition of counter declarations is:

\begin{verbatim}
                    'COUNTERS' CounterSeq
     CounterSeq   = Counter < Counter > .
     Counter      = CounterID ':' CounterFunc ';' .
     CounterID    = NAME .
     CounterFunc  = 'RANK' 'OF' TypeOrPage [ SLevelAsc ]
                    [ 'INIT' AttrID ] [ 'REINIT' AttrID ] /
                    SetFunction < SetFunction >
                    AddFunction < AddFunction >
                    [ 'INIT' AttrID ] /
                    'RLEVEL' 'OF' ElemID .
     SLevelAsc    = [ '-' ] LevelAsc .
     LevelAsc     =  NUMBER .
     SetFunction  = 'SET' CounterValue 'ON' TypeOrPage .
     AddFunction  = 'ADD' CounterValue 'ON' TypeOrPage .
     TypeOrPage   = 'Page' [ '(' ViewID ')' ] / 
                    [ '*' ] ElemID .
     CounterValue = NUMBER .
\end{verbatim}

\begin{example}
If the body of a chapter is defined as a sequence of sections in the
structure schema:

\begin{verbatim}
Chapter_body = LIST OF (Section = 
                            BEGIN
                            Section_Title = Text;
                            Section_Body  = Paragraphs;
                            END
                         );
\end{verbatim}
the section counter is declared:

\begin{verbatim}
SectionCtr : RANK OF Section;
\end{verbatim}
and the display of the section number before the section title is
obtained by a {\tt CreateBefore} rule (see page~\pageref{creation})
attached the {\tt Section\_Title} type, which creates a presentation
box whose content is the value of the {\tt SectionCtr} counter (see
the {\tt Content} instruction, page~\pageref{content}).

In order to number the formulas separately within each chapter, the
formula counter is declared:

\begin{verbatim}
FormulaCtr : SET 0 ON Chapter ADD 1 ON Formula;
\end{verbatim}
and the display of the formula number in the right margin, alongside
each formula, is obtained by a {\tt CreateAfter} instruction attached
to the {\tt Formula} type, which creates a presentation box whose
content is the value of the {\tt FormulaCtr} counter.

To number the page chapter by chapter, with the first page of each
chapter having the number 1, the counter definition would be
\begin{verbatim}
ChapterPageCtr : SET 0 ON Chapter ADD 1 ON Page;
\end{verbatim}
If there is also a chapter counter
\begin{verbatim}
ChapterCtr : RANK OF Chapter;
\end{verbatim}
the content (see page~\pageref{content}) of a presentation box created
at the top of each page could be defined as:
\begin{verbatim}
Content : (VALUE(ChapterCtr, URoman) TEXT '-'
           VALUE(ChapterPageCtr, Arabic));
\end{verbatim}
Thus, the presentation box contains the number of the chapter in
upper-case roman
numerals followed by a hyphen and the number of the page within the
chapter in arabic numerals.
\end{example}

\begin{example}
To count tables and figures together in a document of the chapter
type, a counter could be defined using:
\begin{verbatim}
CommonCtr : SET 0 ON Chapter ADD 1 ON Table
            ADD 1 ON Figure;
\end{verbatim}
\end{example}

\subsection{Presentation constants}
\label{constpres}

Presentation constants are used in the definition of the content of
presentation boxes.  This content is used in variable definitions (see
page~\pageref{variables}) and in the {\tt Content} rule (see
page~\pageref{content}).  The only presentation constants which can be
used are character strings, mathematical symbols, graphical elements,
and images, that is to say, base elements.

Constants can be defined directly in the variables or presentation
boxes ({\tt Content} rule) which use them.  But it is only necessary them
to declare once, in the constant declaration section, even though they
are used in many variables or boxes.  Thus, each declared constant has
a name, which allows it to be designated whenever it is used, a type
(one of the four base types) and a value (a character string or a
single character for mathematical symbols and graphical elements).

The constant declarations appear after the keyword {\tt CONST}.  Each
declaration is composed of the name of the constant, an equals sign, a
keyword representing its type ({\tt Text}, {\tt Symbol}, {\tt
Graphics} or {\tt Picture}) and the string representing its value.
A semi-colon terminates each declaration.

In the case of a character string, the keyword {\tt Text} can be
followed by the name of an alphabet (for example, {\tt Greek} or {\tt
Latin}) in which the constant's text should be expressed.  If the
alphabet name is absent, the Latin alphabet is used.  When the
alphabet name is present, only the first letter of the alphabet name
is interpreted.  Thus, the words {\tt Greek} and {\tt Grec} designate
the same alphabet.  In current versions of Thot, only the Greek and
Latin alphabets are available.

\begin{verbatim}
                 'CONST' ConstSeq
     ConstSeq   = Const < Const > .
     Const      = ConstID '=' ConstType ConstValue ';' .
     ConstID    = NAME .
     ConstType  ='Text' [ Alphabet ] / 'Symbol' /
                 'Graphics' / 'Picture' .
     ConstValue = STRING .
     Alphabet   = NAME .
\end{verbatim}

For character strings in the Latin alphabet (ISO-Latin 1 character
set), characters having codes higher than 127 (decimal) are
represented by their code in octal.

In the case of a symbol or graphical element, the value only contains
a single character, between apostrophes, which indicates the form of
the element which must be drawn in the box whose content is the
constant.  The symbol or graphical element takes the dimensions of the
box, which are determined by the {\tt Height} and {\tt Width} rules.
The table of codes for the symbols and graphical elements can be found
on page~\pageref{codesymbole}.

\begin{example}
The constants ``Summary:'' and fraction bar, which were described
earlier, are declared:

\begin{verbatim}
CONST
     SummaryConst = Text 'Summary:';
     Bar          = Graphics 'h';
\end{verbatim}
\end{example}

\subsection{Variables}
\label{variables}

Variables permit the definition of computed content for presentation
boxes.  A variable is associated with a presentation box by a {\tt
Content} rule; but before being used in a {\tt Content} rule, a
variable can be defined in the {\tt VAR} section.  It is also possible
to define a variable at the time of its use in a {\tt Content} rule,
as can be done with a constant.

A variable has a name and a value which is a character string
resulting from the concatenation of the values of a sequence of
functions.  Each variable declaration is composed of the variable name
followed by a colon and the sequence of functions which produces its
value, separated by spaces.  Each declaration is terminated by a
semi-colon.

\begin{verbatim}
                     'VAR' VarSeq
     VarSeq       = Variable < Variable > .
     Variable       = VarID ':' FunctionSeq ';' .
     VarID       = NAME .
     FunctionSeq = Function < Function > .
\end{verbatim}

Several functions are available.  The first two return, in the form of a
character string, the current date.  {\tt DATE} returns the date in
English, while {\tt FDATE} returns the date in french.


Two other functions, {\tt DocName} and {\tt DirName}, return the document
name and the directory where the document is stored.

Function {\tt ElemName} returns the type of the element which created
the presentation box whose contents are the variable.

Another function simply returns the value of a presentation constant.
For any constant declared in the {\tt CONST} section, it is sufficient
to give the name of the constant.  Otherwise, the type and value of
the constant must be given, using the same form as in a constant
declaration (see page~\pageref{constpres}). If the constant is not
of type text, (types {\tt Symbol}, {\tt Graphics} or {\tt Picture}),
it must be alone in the variable definition; only constants of type
{\tt Text} can be mixed with other functions.

It is also possible to obtain the value of an attribute, simply by
mentioning the attribute's name.  The value of this function is the
value of the attribute for the element which created the presentation
box whose contents are the variable.  If the creating element does not
have the indicated attribute, the value is an empty string.  In the
case of a numeric attribute, the attribute is translated into a
decimal number in arabic numerals.  If another form is desired, the
{\tt VALUE} function must be used.

The last available function returns, as a character string, the value
of a counter, an attribute or a page number. This value can be
presented in different styles.  The keyword {\tt VALUE} is followed
(between parentheses) by the name of the counter, the name of the
attribute, or the keyword {\tt PageNumber} and the desired style, the
two parameters being separated by a comma.  The style is a keyword
which indicates whether the value should be presented in arabic
numerals ({\tt Arabic}), lower-case roman numerals ({\tt LRoman}),
or upper-case roman numerals ({\tt URoman}), or by an upper-case
letter ({\tt Uppercase}) or lower-case letter ({\tt Lowercase}).

For a page counter, the keyword {\tt PageNumber} can be followed,
between parentheses, by the name of the view from which to obtain the
page number.  By default, the first view declared in the {\tt VIEWS}
section is  used.  The value obtained is the number of the page on
which is found the element that is using the variable in a {\tt
Content} rule.

For an ordinary counter, the name of the counter can be preceded by
the keyword {\tt MaxRangeVal} or {\tt MinRangeVal}.  These keywords
mean that the value returned by the function is the maximum (minimum
resp.) value taken by the counter in the whole document, not the
value for the element concerned by the function.

\begin{verbatim}
     Function =     'DATE' / 'FDATE' /
                    'DocName' / 'DirName' /
		    'ElemName' / 'AttributeName' /
                     ConstID / ConstType ConstValue /
                     AttrID /
                    'VALUE' '(' PageAttrCtr ','
                                CounterStyle ')' .
     PageAttrCtr =  'PageNumber' [ '(' ViewID ')' ] /
                     [ MinMax ] CounterID / AttrID .
     CounterStyle = 'Arabic' / 'LRoman' / 'URoman' /
                    'Uppercase' / 'Lowercase' .
     MinMax =       'MaxRangeVal' / 'MinRangeVal' .
\end{verbatim}

\begin{example}
To make today's date appear at the top of the first page of a report,
a {\tt CREATE} rule (see page~\pageref{creation}) associated with the
Report\_Title element type generates a presentation box whose content
(specified by the {\tt Content} rule of that presentation box) is the variable:
\begin{verbatim}
VAR
     Todays_date : TEXT 'Version of ' DATE;
\end{verbatim}

To produce, before each section title, the chapter number (in
upper-case roman numerals) followed by the section number (in arabic
numerals), two counters must be defined:

\begin{verbatim}
COUNTERS
     ChapterCtr : RANK OF Chapter;
     SectionCtr : RANK OF Section;
\end{verbatim}
and the Section\_Title element must create a presentation box whose
content is the variable
\begin{verbatim}
VAR
     SectionNum : VALUE (ChapterCtr, URoman) TEXT '-'
                  VALUE (SectionCtr, Arabic);
\end{verbatim}

In order to make the page number on which each section begins appear
in the table of contents view next to the section title, each
Section\_Title element must create a presentation box, visible only in
the table of contents view, whose content is the variable:

\begin{verbatim}
VAR
     TitlePageNume :
           VALUE (PageNumber(Full_text), Arabic);
\end{verbatim}
\end{example}

\subsection{Default presentation rules}
\label{reglesdefaut}

In order to avoid having to specify, for each element type defined in
the structure schema, values for every one of the numerous
presentation parameters, the presentation schema allows the definition
of a set of default presentation rules.  These rules apply to all the
boxes of the elements defined in the structure schema and to the
presentation boxes and page layout boxes defined in the presentation
schema.  Only rules which differ from these default need to be
specified in other sections of the presentation schema.

For the primary view, the default rules can define every presentation
parameter, but not the presentation functions (see
page~\pageref{fonctpres}) or the linebreaking conditions (the {\tt
NoBreak1}, {\tt NoBreak2}, and {\tt Gather} rules, see
page~\pageref{condcoupure}).

In a presentation schema, the default presentation rules section is
optional; in this case, the {\tt DEFAULT} keyword is also absent and
the following rules are considered to be the default rules:

\begin{verbatim}
   Visibility: Enclosing =;
   VertRef: * . Left;
   HorizRef: Enclosed . HRef;
   Height: Enclosed . Height;
   Width: Enclosed . Width;
   VertPos: Top = Previous . Bottom;
   HorizPos: Left = Enclosing . Left;
   Size: Enclosing =;
   Style: Enclosing =;
   Font: Enclosing =;
   Underline: Enclosing =;
   Thickness: Enclosing =;
   Indent: Enclosing =;
   LineSpacing: Enclosing =;
   Adjust: Enclosing =;
   Justify: Enclosing =;
   Hyphenate: Enclosing =;
   PageBreak: Yes;
   LineBreak: Yes;
   InLine: Yes;
   Depth: 0;
   LineStyle: Enclosing =;
   LineWeight: Enclosing =;
   FillPattern: Enclosing =;
   Background: Enclosing =;
   Foreground: Enclosing =;
\end{verbatim}

If other values are desired for the default rules, they must be
defined explicitly in the default rules section.  In fact, it is only
necessary to define those default rules which differ from the ones
above, since the rules above will be used whenever a rule is not
explicitly named.

Default rules for views other than the primary  view can also be
specified.  Otherwise, the default rules for the primary views are
applied to the other views.

Default rules are expressed in the same way as explicit rules for
document elements.  Their syntax is described on
page~\pageref{reglepres}.

\subsection{Presentation and page layout boxes}

The presentation process uses elements which are not part of the
logical structure of the document, such as pages (which are the page
layout boxes) or alternatively, rules, numbers, or words introducing
certain parts of the document, such as ``Summary'', ``Appendices'',
``Bibliography'', etc. (which are presentation boxes).

After the word {\tt BOXES}, each presentation or page layout box is
defined by its name and a sequence of presentation rules which
indicate how they must be displayed.  These rules are the same as
those which define the boxes associated with element of the logical
structure of the document, with a single exception, the {\tt Content}
rule (see page~\pageref{content}) which is used only to specify the
content of presentation boxes.  The content of boxes associated with
elements of the document structure is defined in each document or
object and thus is not specified in the presentation schema, which
applies to all documents or objects of a class.

Among the rules which define a presentation box, certain ones can
refer to another presentation box (for example, in their positional
rules).  If the designated box is defined after the box which
designates it, a {\tt FORWARD} instruction followed by the name of the
designated box must appear before the designation.

\begin{verbatim}
              'BOXES' BoxSeq
     BoxSeq = Box < Box > .
     Box    ='FORWARD' BoxID ';' /
              BoxID ':' ViewRuleSeq .
     BoxID  = NAME .
\end{verbatim}

\subsection{Presentation of structured elements}

After the words {\tt RULES}, the presentation schema gives the
presentation rules that apply to the elements whose types are defined
in the structure schema.  Only those rules which differ from the
default must be specified in the {\tt RULES} section only (see
page~\pageref{reglesdefaut}).

The rule definitions for each element type are composed of the name of
the element type (as specified in the structure schema) followed by a
colon and the set of rules specific to that type.

The type name can be preceded by a star in the special case where the
structure schema defines an inclusion without expansion (or with
partial expansion) of a type with the same name as an element of
defined in the structure schema (see section~\ref{references}).

In the case where the element is a mark pair (see
section~\ref{paires}), but only in this case, the type name can be
preceded by the keywords {\tt First} or {\tt Second}.  These keywords
indicate whether the rules that follow apply to the first or second
mark of the pair.

\begin{verbatim}
                   'RULES' PresentSeq
     PresentSeq = Present < Present > .
     Present      = [ '*' ] [ FirstSec ] ElemID ':'
                    ViewRuleSeq .
     FirstSec     = 'First' / 'Second' .
\end{verbatim}

A presentation schema can define presentation rules for base elements,
which are defined implicitly in the structure schemas.  In the English
version of the presentation schema compiler, the base type names are
the same as in the S language, but they are terminated by the {\tt
\_UNIT} suffix: {\tt TEXT\_UNIT}, {\tt PICTURE\_UNIT}, {\tt
SYMBOL\_UNIT}, {\tt GRAPHICS\_UNIT}.  In the French version of the
compiler, the base types are: {\tt TEXTE}, {\tt IMAGE}, {\tt SYMBOLE},
{\tt GRAPHIQUE}.  Whatever version of the compiler is used, the base
type names are written in upper-case letters.

\subsection{Logical attribute presentation}
\label{presattributes}

After the keyword {\tt ATTRIBUTES}, all attributes which are to have
some effect on the presentation of the element to which they are
attached must be mentioned, along with the corresponding presentation
rules.  This is true for both global attributes (which can be attached
to all element types) and local attributes (which can only be attached
to certain element types).

Also mentioned in this section are attributes which imply an effect on
elements in the subtree of the element to which they are attached.
The presentation of these descendants  can be modified as a function
of the value of the attribute which they inherit, just as if it was
attached to them directly.

The specification for each attribute includes the attribute's name,
followed by an optional value specification and, after a colon, a
set of rules.  The set of rules must contain at least one rule.

When there is no value specification, the rules are applied to all
elements which carry the attribute, no matter what their value.  When
the rules must only apply when the attribute has certain values, these
values must be specified.  Thus, the same attribute can appear in the
{\tt ATTRIBUTES} section several times, with each appearance having a
different value specification.  However, reference attributes never
have a value specification and, as a result, can only appear once in
the {\tt ATTRIBUTES} section.

To specify that the presentation rules apply to some of the
descendants of the element having the attribute,
the name of the affected element type is given,
between parentheses, after the attribute name.  This way, the
presentation rules for the attribute will be applied to the
element having the attribute, if it is of the given type, and to all of its descendants of the given type.
In the case where this type is a mark pair (see
section~\ref{paires}), but only in this case, the type name can be
preceded by the keywords {\tt First} or {\tt Second}.  These keywords
indicate whether the rules that follow apply to the first or second
mark of the pair.
If the rule must apply to several different element
types, the specification must be repeated for each element type.

The specification of values for which the presentation rules will be
applied varies according to the type of the attribute:
\begin{description}
\item{numeric attribute:} If the rules are to apply for one value of
the attribute, then the attribute name is followed by an equals sign
and this value.  If the rules are to apply for all values less than
(or greater than) a threshold value, non-inclusive, the attribute name
followed by a '\verb|<|' sign (or a '\verb|>|' sign, respectively) and
the threshold value.  If the rules must apply to a range of values,
the attribute name is followed by the word '{\tt IN}' and the two
bounds of the range, enclosed in brackets and separated by two periods
('{\tt ..}').  In the case of ranges, the values of the bounds are
included in the range.

The threshold value in the comparisons can be the value of an
attribute attached to an ancestor element.  In this case, the
attribute name is given instead of a constant value.

It is also possible to write rules which apply only when a comparison
between two different attributes of the element's ancestors is true.
In this case, the first attribute name is followed by a comparison
keyword and the name of the second attribute.  The comparison keywords
are {\tt EQUAL} (simple equality), {\tt LESS} (non-inclusive less
than), and {\tt GREATER} (non-inclusive greater than).

\item{text attribute:}  If the rules are to apply for one value of
the attribute, then the attribute name is followed by an equals sign
and this value.

\item{reference attribute:}  There is never a value specification; the
rules apply no matter what element is designated by the attribute.

\item{enumerated attribute:}  If the rules are to apply for one value of
the attribute, then the attribute name is followed by an equals sign
and this value.

\end{description}

The order in which the rules associated with a numeric attribute are
defined is important.  When multiple sets of rules can be applied, the
first set declared is the one used.

Rules for attributes have priority over both default rules and rules
associated with element types.  The attribute rules apply to the
element to which the attribute is attached.  It is the rules which
apply to the surrounding elements (and especially to the descendants)
which determine the effect of the attribute rules on the environment (
and especially on the terminal elements of the structure).

\begin{verbatim}
                    'ATTRIBUTES' PresAttrSeq
     PresAttrSeq  = PresAttr < PresAttr > .
     PresAttr     = AttrID [ '(' [ FirstSec ] ElemID ')' ]
                    [ AttrRelation ] ':' ViewRuleSeq .
     AttrID       = NAME .
     AttrRelation ='=' AttrVal /
                    '>' [ '-' ] MinValue /
                    '<' [ '-' ] MaxValue /
                    'IN' '[' [ '-' ] LowerBound '..'
                    [ '-' ] UpperBound ']' /
                    'GREATER' AttrID /
                    'EQUAL' AttrID /
                    'LESS' AttrID .
     AttrVal      = [ '-' ] EqualNum / EqualText /
                    AttrValue .
     MinValue     = NUMBER .
     MaxValue     = NUMBER .
     LowerBound   = NUMBER .
     UpperBound   = NUMBER.
     EqualNum     = NUMBER .
     EqualText    = STRING .
     AttrValue    = NAME .
\end{verbatim}

In presentation rules associated with a numeric attribute (and only in
such rules), the attribute name can be used in place of a numeric
value.  In this case, the value of the attribute is used in the
application of the rule.  Thus, the attribute can represent a relation
between the size of two boxes, the height and width of a box, the
height of an area where page breaks are prohibited, the distance
between two boxes, the position of the reference axis of a box, the
interline spacing,  the indentation of the first line, the visibility,
the depth of (???), or the character set.

The the presentation rules associated with reference attributes, it is
possible to use the element designated by the attribute as a reference
box in a positional or extent rule.  This element is represented in the
position or extent rule by the keyword {\tt Referred} (see
sections~\ref{position} and~\ref{dimension}).

\begin{example}
In all structure schemas, there is a global Language attribute
defined as follows:

\begin{verbatim}
ATTR
     Language = TEXT;
\end{verbatim}
The following rules would make French text be displayed in roman
characters and English text be displayed in italics:

\begin{verbatim}
ATTRIBUTES
     Language = 'French' :
              Style : Roman;
     Language = 'English' :
              Style : Italics;
\end{verbatim}
Using these rules, when the user puts the Language attribute with the
value 'English' on the summary of a document, every character string
(terminal elements) contained in the summary are displayed in italics.
The {\tt Style} rule is described on page~\pageref{style}.
\end{example}

\begin{example}
A numeric attribute representing the importance of the part of the
document to which it is attached can be defined:
\begin{verbatim}
ATTR
     Importance = INTEGER;
\end{verbatim}
In the presentation schema, the importance of an element is reflected
in the choice of character size, using the following rules.

\begin{verbatim}
ATTRIBUTES
     Importance < 2 :
              Size : 1;
     Importance IN [2..4] :
              Size : Importance;
     Importance = 10 :
              Size : 5;
     Importance > 4 :
              Size : 4;
\end{verbatim}
Thus, the character size corresponds to the value of the Importance
attribute; its value is

\begin{itemize}
  \item the value of the Importance attribute when the value is
between 2 and 4 (inclusive), 

  \item 1, when the value of the Importance attribute is less than 2,

  \item 4, when the value of the Importance attribute is greater than
4,

  \item 5, when the value of the Importance attribute is 10.
\end{itemize} 
The last case (value 5) must be defined before the case which handles
all Importance values greater than 4, because the two rules are not
disjoint and the first one defined will have priority.  Otherwise,
when the Importance attribute has value 10, the font size will be 4.
\end{example}

\begin{example}
Suppose the structure defines a list element which can have an
attribute defining the type of list (numbered or not):
\begin{verbatim}
STRUCT
    list (ATTR list_type = enumeration, dash)
         = LIST OF (list_item = TEXT);
\end{verbatim}
Then, the presentation schema could use the attribute placed on the
list element to put either a dash or a number before the each element
of the list:
\begin{verbatim}
ATTRIBUTES
   list_type (list_item) = enumeration :
        CreateBefore (NumberBox);
   list_type (list_item) = dash :
        CreateBefore (DashBox);
\end{verbatim}
\end{example}

\begin{example}
Suppose that two attributes are defined in the structure schema.  The
first is a numeric global attribute called ``version''.  The other is
a local attribute defined on the root of the document called
``Document\_version'':

\begin{verbatim}
STRUCTURE Document
ATTR
    version = INTEGER;
STRUCT
    Document (ATTR Document_version = INTEGER) =
        BEGIN
        SomeElement ;
        ...
        SomeOtherElement ;
        END ;
...
\end{verbatim}
These attributes can be used in the presentation schema to place
change bars in the margin next to elements whose version attribute
has a value equal to the Document\_version attribute of the root and
to place a star in margin of elements whose version attribute is less
than the value of the root's Document\_version attribute:
\begin{verbatim}
ATTRIBUTES
    version EQUAL Document_version :
        CreateBefore (ChangeBarBox) ;
    version LESS Document_version :
        CreateBefore (StarBox) ;
\end{verbatim}
\end{example}

\subsection{Value transmission rules}

The last section of a presentation schema, which is optional, serves
to defines the way in which a document transmits certain values to its
sub-documents.  A sub-document is an document included without
expansion or with partial expansion (see page~\pageref{inclusion}).
The primary document can transmit to its sub-documents the values of
certain counters or the textual content of certain of its elements, as
a function of their type.

The sub-documents receive these values in attributes which must be
defined in their structure schema as local attributes of the root
element.  The types of these attributes must correspond to the type of
the value which they receive:  numeric attributes for receiving the
value of a counter, textual attributes for receiving the content of
an element.

In the structure schema of the primary document, there appears at the
end, after the {\tt TRANSMIT} keyword, a sequence of transmission
rules.  Each rule begins with the name of the counter to transmit or of the
element type whose textual content will be transmitted.  This name is
followed by the keyword {\tt To} and the name of the attribute of the
sub-document to which the value is transmitted.  The sub-document class
is indicated between parentheses after the name of the attribute.  The
transmission rule ends with a semicolon.

\begin{verbatim}
     TransmitSeq   =  Transmit < Transmit > .
     Transmit      =  TypeOrCounter 'To' ExternAttr
                      '(' ElemID ')' ';' .
     TypeOrCounter =  CounterID / ElemID .
     ExternAttr    =  NAME .
\end{verbatim}

\begin{example}
Consider a Book document class which includes instances of the Chapter
document class.  These classes might have the following schemas:

\begin{verbatim}
STRUCTURE Book
STRUCT
   Book = BEGIN
          Title = Text;
          Body  = LIST OF (Chapter INCLUDED);
          END;
   ...

STRUCTURE Chapter
STRUCT
   Chapter (ATTR FirstPageNum = Integer;
                 ChapterNum = Integer;
                 CurrentTitle   = Text) =
           BEGIN
           ChapterTitle = Text;
           ...
           END;
   ...
\end{verbatim}

Then the presentation schema for books could define chapter and page
counters.  The following transmission rules in the book presentation
schema would transmit values for the three attributes defined at the
root of each chapter sub-document.

\begin{verbatim}
PRESENTATION Book;
VIEWS
   Full_text;
COUNTERS
   ChapterCtr: Rank of Chapter;
   PageCtr: Rank of Page(Full_text);
...
TRANSMIT
   PageCtr TO FirstPageNum(Chapter);
   ChapterCtr TO ChapterNum(Chapter);
   Title TO CurrentTitle(Chapter);
END
\end{verbatim}

Thus, each chapter included in a book can number its pages as a
function of the number of pages preceding it in the book, can make
the chapter's number appear before the number of each of its sections,
or can place the title of the book at the top of each page.
\end{example}

\subsection{Presentation rules}

Whether defining the appearance of a presentation or page layout box,
an element type, or an attribute value, the set of presentation rules
that apply is always defined in the same way.

Normally, a set of presentation rules is placed between the keywords
{\tt BEGIN} and {\tt END}, the keyword {\tt END} being followed by a
semicolon.  The first section of this block defines the rules that
apply to the primary view, if the default rules (see
page~\pageref{reglesdefaut}) are not completely suitable.  Next comes
the rules which apply to specific other views, with a rule sequence
for each view for which the default rules are not satisfactory.  If
the default rules are suitable for the non-primary views, there will
not be any specific rules for these views.  If there is only one rule
which applies to all views then the keywords {\tt BEGIN} and {\tt END}
need not appear.

For each view, it is only necessary to specify those rules which
differ from the default rules for the view, so that for certain views
(or even all views), there may be no specific rules.

\label{instrin}
The specific rules for a non-primary view are introduced by the {\tt
IN} keyword, followed by the view name.  The rules for that view
follow, delimited by the keywords {\tt BEGIN} and {\tt END}, or
without these two keywords when there is only one rule.

{\bf Note:} the view name which follows the {\tt IN} keyword must not
be the name of the primary view, since the rules for that view are
found before the rules for the other views.

Within each block concerning a view, other blocks can appear, delimited
by the same keywords {\tt BEGIN} and {\tt END}.  Each of these blocks
gathers the presentation rules that apply, for a given view, only when
a given condition is satisfied.  Each block is preceded by a condition
introduced by the {\tt IF} keyword.  If such a conditional block
contains only one rule, the keywords {\tt BEGIN} and {\tt END} can
be omitted.

Although the syntax allows any presentation rule to appear in
a conditional block, only creation rules (see section~\ref{creation})
are allowed after any condition; other rules are allowed only after
conditions {\tt Within} and ElemID.  In addition, the following
rules cannot
be conditional: {\tt PageBreak, LineBreak, Inline, Gather}.

For a given view, the rules that apply without any condition must
appear before the first conditional block.  If some rules apply only
when none of the specified condition holds, they are grouped in a
block preceded by the keyword {\tt Otherwise}, and that block must
appear after the last conditionnal block concerning the same view.

\begin{verbatim}
     ViewRuleSeq  = 'BEGIN' < RulesAndCond > < ViewRules >
                    'END' ';' /
                    ViewRules / CondRules / Rule .
     RulesAndCond = CondRules / Rule .
     ViewRules    = 'IN' ViewID CondRuleSeq .
     CondRuleSeq  = 'BEGIN' < RulesAndCond > 'END' ';' /
                    CondRules / Rule .
     CondRules    = CondRule < CondRule >
                    [ 'Otherwise' RuleSeq ] .
     CondRule     = 'IF' ConditionSeq RuleSeq .
     RulesSeq     = 'BEGIN' Rule < Rule > 'END' ';' /
                    Rule .
\end{verbatim}

\begin{example}
The following rules for a report's title make the title visible in the
primary view and invisible in the table of contents and in the formula
views (the {\tt Visibility} rule is presented on page~\pageref{visib}).
\begin{verbatim}
Title : BEGIN
        Visibility : 1;
        ...    {Other rules for the primary view}
        IN Table_of_contents
           Visibility : 0;
        IN Formulas
           Visibility : 0;
        END;
\end{verbatim}
\end{example}

\subsection{Conditions applying to presentation rules}

Many conditions can be applied to presentation rules.  Conditions allow
certain presentation rules to apply only in certain cases.  These
conditions can be based on the structural position of the element.  They
can be based on whether the element has references, and what type
of references, whether the element has attributes, whether the element
is empty or not.  They can also be based on the value of a counter.

It is possible to specify several conditions which must
all be true for the rules to apply.

A set of conditions is specified by the {\tt IF} keyword.  This
keyword is followed by the sequence of conditions, separated by the
{\tt AND} keyword.  Each condition is specified by a keyword which
defines the condition type.  In some cases, the keyword is followed
by other data, which specify the condition more precisely.

An elementary condition can be
negative; it is then preceded by the {\tt NOT} keyword.

When the presentation rule(s) controlled by the condition apply to a
reference element or a reference attribute, an elementary condition
can also apply to element referred by this reference.  The {\tt Target}
keyword is used for that purpose.  It must appear before the keyword
defining the condition type.

\begin{verbatim}
     CondRule     = 'IF' ConditionSeq RuleSeq .
     ConditionSeq  = Condition < 'AND' Condition > .
     Condition     = [ 'NOT' ] [ 'Target' ] ConditionElem .
     ConditionElem ='First' / 'Last' /
                     [ 'Immediately' ] 'Within' [ NumParent ]
                                       ElemID [ ExtStruct ] /
                     ElemID /
                    'Referred' / 'FirstRef' / 'LastRef' /
                    'ExternalRef' / 'InternalRef' / 'CopyRef' /
                    'AnyAttributes' / 'FirstAttr' / 'LastAttr' /
                    'UserPage' / 'StartPage' / 'ComputedPage' /
                    'Empty' /
                    '(' [ MinMax ] CounterName CounterCond ')' /
                     CondPage '(' CounterID ')' .
     NumParent     = [ GreaterLess ] NParent .
     GreaterLess   = '>' / '<' .
     NParent       = NUMBER.
     ExtStruct     = '(' ElemID ')' .
     CounterCond   ='<' MaxCtrVal / '>' MinCtrVal /
                    '=' EqCtrVal / 
                    'IN' '[' ['-'] MinCtrBound '.' '.'
                     ['-'] MaxCtrBound ']' .
     PageCond      ='Even' / 'Odd' / 'One' .
     MaxCtrVal     = NUMBER .
     MinCtrVal     = NUMBER .
     EqCtrVal      = NUMBER .
     MaxCtrBound   = NUMBER .
     MinCtrBound   = NUMBER .
\end{verbatim}

\subsubsection{Conditions based on the logical position of the element}

The condition can be on the position of the element in the document's
logical structure tree.  It is possible to test whether the element is
the first ({\tt First}) or last ({\tt Last}) among its siblings or if
it is not the first ({\tt NOT First}) or not the last ({\tt NOT
Last}). These conditions can be associated only with creation rules
(see section~\ref{creation}).

It is also possible to test if the element is contained in an element
of a given type ({\tt Within}) or if it is not ({\tt NOT Within}).
The type is indicated after the keyword {\tt Within}.
If that element type is defined in a structure schema which is
not the one which corresponds to the presentation schema, the type name
of this element must be followed, between parentheses, by the name of
the structure schema which defines it.

If the keyword
{\tt Within} is preceded by {\tt Immediately}, the condition is
satisfied only if the {\em parent} element has the type indicated.
If the word {\tt Immediately} is missing, the condition is satisfied
if any {\em ancestor} has the type indicated.

An integer $n$ can appear between the keyword {\tt Within} and the
type.  It specifies the number of ancestors of the indicated type that must
be present for the condition to be satisfied.  If the keyword
{\tt Immediately} is also present, the $n$ immediate ancestors of the
element must have the indicated type.  The integer $n$ must be positive
or zero.  It can be preceded by {\tt <} or {\tt >} to indicate a maximum
or minimum number of ancestors.  If these symbols are missing, the
condition is satisfied only if it exists exactly $n$ ancestors.  When
this number is missing, it is equivalent to > 0.

If the condition applies to presentation rules associated with an attribute,
in the {\tt ATTRIBUTEs} section of the presentation schema, the condition
can be simply an element name. Presentation rules are then executed only
if the attribute is attached to an element of that type. The keyword
{\tt NOT} before the element name indicates that the presentation rules
must be executed only if the element is not of the type indicated.

\subsubsection{Conditions on references}

References may be taken into account in conditions, which can be based
on the fact that the element, or one of its ancestors, is designated
by a at least one reference ({\tt Referred}) or by none ({\tt NOT
Referred}).

If the element or attribute to which the condition is
attached is a reference, the condition can be based on the fact that
it acts as the first reference to the designated element ({\tt FirstRef}),
or as the last ({\tt LastRef}), or as a reference to an element located in
another document ({\tt ExternalRef}) or in the same document
({\tt InternalRef}).

The condition can also be based on the fact that the element is an
inclusion (see page~\pageref{inclusion}).  This is noted ({\tt CopyRef}).

Like all conditions, conditions
on references can be inverted by the {\tt NOT} keyword.
These conditions can be associated only with creation rules
(see section~\ref{creation}).

\subsubsection{Conditions on logical attributes}

The condition can be based on the presence or absence
of attributes associated with the element, no matter what
the attributes or their values.  The {\tt AnyAttributes} keyword
expresses this condition.

If the condition appears in the presentation rules of an attribute, the
{\tt FirstAttr} and {\tt LastAttr} keywords can be used to indicate
that the rules must only be applied if this attribute is the first
attribute for the element or if it is the last
(respectively).  These conditions can also be inverted by the {\tt
NOT} keyword.
These conditions can be associated only with creation rules
(see section~\ref{creation}).

It is also possible to apply certain presentation rules only
when the element being processed or one of its ancestors has a certain
attribute, perhaps with a certain value.  This can be done in the
{\tt ATTRIBUTES} section (see section~\ref{presattributes}).

\subsubsection{Conditions on page breaks}

The page break base type (and only this type) can use the following
conditions:\\
{\tt ComputedPage}, {\tt StartPage}, and {\tt UserPage}.  The
{\tt ComputedPage} condition indicates that
the presentation rule(s) should apply if the page break was created
automatically by Thot;  the {\tt StartPage} condition is true if the
page break is generated before the element by the {\tt Page} rule;
and the {\tt UserPage} condition applies if the page break
was inserted by the user.

These conditions can be associated only with creation rules
(see section~\ref{creation}).

\subsubsection{Conditions on the element's content}

The condition can be based on whether or not the element is empty.  An
element which has no children or whose leaves are all empty is
considered to be empty itself.  This condition is expressed by the
{\tt Empty} keyword, optionally preceded by the {\tt NOT} keyword.
This condition can be associated only with creation rules
(see section~\ref{creation}).

\subsubsection{Conditions on counters}

Presentation rules can apply when the counter's value is one, is even or odd, is
equal, greater than or less than a given value or falls in a range of
values.  This is particularly useful for creating header and
footer boxes.
These conditions can be associated only with creation rules
(see section~\ref{creation}).

To compare the value of a
counter to a given value, a comparison is given between parentheses.
The comparison is composed of the counter name followed by an equals,
greater than, or less than sign and the value to which the counter
will be compared.  A test for whether or not a counter's value falls in a
range also appears within parentheses.  In this case, the counter name
is followed by the {\tt IN} keyword and the range definition within
brackets.  The {\tt Even}, {\tt Odd} and {\tt One} are used to test
a counter's value and are followed by the counter name between
parentheses.

The list of possible conditions on counters is:

\begin{description}
\item[ {\tt Even (Counter):} ]the box is created only if the counter
has an even value.
\item[ {\tt Odd (Counter):} ]the box is created only if the counter
has an odd value.
\item[ {\tt One (Counter):} ]the box is created only the counter's
value is 1.
\item[ {\tt NOT One (Counter):} ]the box is created, unless the
counter's value is 1.
\item[ {\tt (Counter < Value):} ]the box is created only if the
counter's value is less than Value.
\item[ {\tt (Counter > Value):} ]the box is created only if the
counter's value is greater than Value.
\item[ {\tt (Counter = Value):} ]the box is created only if the
counter's value is equal to Value.
\item[ {\tt NOT (Counter = Value):} ]the is created only if the
counter's value is different than Value.
\item[ {\tt (Counter IN [MinValue..MaxValue]):} ]the box is created
only if the counter's value falls in the range bounded by MinValue and
MaxValue (inclusive).
\item[ {\tt NOT (Counter IN [MinValue..MaxValue]):} ]the box is
created only if the value of the counter does not fall in the range bounded
by MinValue and MaxValue (inclusive).
\end{description}

{\bf Note:} the {\tt NOT Even} and {\tt NOT Odd} conditions are
syntactically correct but can be expressed more simply by {\tt Odd}
and {\tt Even}, respectively.

\subsection{A presentation rule}
\label{reglepres}

A presentation rule defines either a presentation parameter or
presentation function.  The parameters are:
\begin{itemize}
\item the position of the vertical and horizontal reference axes of
the box,
\item the position of the box in relation to other boxes,
\item the height or width of the box,
\item the characteristics of the lines contained in the box: interline
spacing, indentation of the first line, justification, hyphenation,
\item the conditions for breaking the box across pages,
\item the characteristics of the characters contained in the box:
size, font, style, underlining,
\item the depth of the box among overlapping boxes (often called
stacking order),
\item the characteristics of graphic elements contained in the box:
style and thickness of lines, fill pattern for closed objects,
\item the colors in text, graphics, images, and symbols contained in
the box are displayed or printed,
\item for presentation boxes only, the contents of the box.
\end{itemize}
The presentation functions are:
\label{fonctpres}
\begin{itemize}
\item the creation of a presentation box
\item the line-breaking or page-breaking style,
\item the copying of another box.
\end{itemize}

For each box and each view, every presentation parameter is defined
once and only once, either explicitly or by the default rules (see
page~\pageref{reglesdefaut}).  In contrast, presentation functions are
not obligatory and can appear many times for the same element.  for
example an element can create many presentation boxes.  Another
element may not use any presentation functions.

Each rule defining a presentation parameter begins with a keyword
followed by a colon. the keyword indicates the parameter which is the
subject of the rule.  After the keyword and the colon, the remainder
of the rule varies.  All rules are terminated by a semicolon.

\begin{verbatim}
     Rule     = PresParam ';' / PresFunc ';' .
     PresParam ='VertRef' ':' PositionHoriz /
                'HorizRef' ':' PositionVert /
                'VertPos' ':' VPos /
                'HorizPos' ':' HPos /
                'Height' ':' Dimension /
                'Width' ':' Dimension /
                'LineSpacing' ':' DistanceInherit /
                'Indent' ':' DistanceInherit /
                'Adjust' ':' AdjustInherit /
                'Justify' ':' BoolInherit /
                'Hyphenate' ':' BoolInherit /
                'PageBreak' ':' Boolean /
                'LineBreak' ':' Boolean /
                'InLine' ':' Boolean /
                'NoBreak1' ':' AbsDist /
                'NoBreak2' ':' AbsDist /
                'Gather' ':' Boolean /
                'Visibility' ':' NumberInherit /
                'Size'  ':' SizeInherit /
                'Font' ':' NameInherit /
                'Style' ':' StyleInherit /
                'Underline' ':' UnderLineInherit /
                'Thickness' ':' ThicknessInherit /
                'Depth' ':' NumberInherit /
                'LineStyle' ':' LineStyleInherit /
                'LineWeight' ':' DistanceInherit /
                'FillPattern' ':' NameInherit /
                'Background' ':' NameInherit /
                'Foreground' ':' NameInherit .
                'Content' ':' VarConst .
     PresFunc = Creation '(' BoxID ')' /
                'Line' /
                'NoLine' /
                'Page' '(' BoxID ')' /
                'Copy' '(' BoxTypeToCopy ')' .
\end{verbatim}

\subsection{Box axes}

The position of the middle axes {\tt VMiddle} and {\tt HMiddle} in
relation to their box is always calculated automatically as a function
of the height and width of the box and is not specified by the
presentation rules.  In the presentation schema, these middle axes are
used only to position their box with respect to another by specifying
the distance between the middle axis and an axis or a side of another
box (see the relative position, page~\pageref{position}).

The reference axes of a box are also used to position their box in
relation to another, but in contrast to the middle axes, the
presentation schema must make their position explicit, either in
relation to a side or the middle axis of the box itself, or in
relation to an axis of an enclosed box.

Only boxes of base elements have predefined reference axes.  For
character string boxes, the horizontal reference axis is the baseline
of the characters (the line which passes immediately under the
upper-case letters, ignoring the letter Q) and the vertical reference
axis is at the left edge of the first character of the string.

The positions of a box's reference axes are defined by the {\tt
VertRef} and {\tt HorizRef} rules which specify the distance between
the reference axis and an axis or parallel side of the same box or of
an enclosed box (see page~\pageref{distance} for how to specify this
distance).

\begin{verbatim}
               'VertRef' ':' PositionHoriz
               'HorizRef' ':' PositionVert
\end{verbatim}

\begin{example}
If, in the structure schema for mathematical formulas, the fraction
element is defined by

\begin{verbatim}
Fraction = BEGIN
           Numerator   = Expression;
           Denominator = Expression;
           END;
\end{verbatim}
then the horizontal reference axis of the fraction can be positioned
on top of the denominator by the rule:

\begin{verbatim}
Fraction :
     BEGIN
     HorizRef : Enclosed Denominator . Top;
     ...
     END;
\end{verbatim}
To put the horizontal reference axis of a column at its middle:

\begin{verbatim}
Column :
     BEGIN
     HorizRef : * . HMiddle;
     ...
     END;
\end{verbatim}
\end{example}

\subsection{Distance units}
\label{unites}

Some distances and dimensions appear in many rules of a presentation
schema, especially in position rules ({\tt VertPos, HorizPos}), in
extent rules for boxes ({\tt Height, Width}), in rules defining lines
({\tt LineSpacing, Indent}), in rules controlling pagination ({\tt
NoBreak1, NoBreak2}) and in rules specifying the thickness of strokes
({\tt LineWeight}).

In all these rules, the distance or extent can be expressed
\begin{itemize}
   \item either in relative units, which depend on the size of the
         characters in the current font: height of the element's font or
         height of the letter 'x',
   \item or in absolute units: centimeter, millimeter, inch, typographer's
         point, pica or pixel.
\end{itemize}
Units can be chosen freely.  Thus, it is possible to use relative units in
one rule, centimeters in the next rule, and typographer's points in
another.

Absolute units are used to set rigid rules for the appearance of
documents.  In contrast, relative units allow changes of scale.  The
editor lets the value of relative units be changed dynamically.  Such
changes affect every box using relative units simultaneously and in
the same proportion.  Changing the value of the relative units affects
the size of the characters and graphical elements, and the size of the
boxes and the distances between them.

\label{distance}
A distance or extent is specified by a number, which may be followed by
one or more spaces and a units keyword.  When there is no units
keyword, the number specifies the number of relative units, where a
relative unit is the height of a character in the current font (an
em).  When the number is followed by a units keyword, the keyword
indicates the type of absolute units:
\begin{itemize}
   \item {\tt em} : height of the element's font,
   \item {\tt ex} : height of the letter 'x',
   \item {\tt cm} : centimeter,
   \item {\tt mm} : millimeter,
   \item {\tt in} : inch (1 in = 2.54 cm),
   \item {\tt pt} : point (1 pt = 1/72 in),
   \item {\tt pc} : pica (1 pc = 12 pt),
   \item {\tt px} : pixel.
\end{itemize}

Whatever the chosen unit, relative or absolute, the number is not
necessarily an integer and may be expressed in fixed point notation
(using the American convention of a period to express the decimal
point).

If the distance appears in a presentation rule for a numeric
attribute, the number can be replaced by the name of an attribute.  In
this case, the value of the attribute is used.  Obviously, the
attribute name cannot be followed by a decimal point and a fractional
part, but it can be followed a units keyword.  However, the choice of
units is limited to em, ex, pt and px.

\begin{verbatim}
     Distance      = [ Sign ] AbsDist .
     Sign          ='+' / '-' .
     AbsDist       = IntegerOrAttr [ '.' DecimalPart ]
                     [ Unit ].
     IntegerOrAttr = IntegerPart / AttrID .
     IntegerPart   = NUMBER .
     DecimalPart   = NUMBER .
     Unit          ='em' / 'ex' / 'cm' / 'mm' / 'in' / 'pt' /
                    'pc' / 'px' / '%' .
\end{verbatim}

\begin{example}
The following rules specify that a box has a height of 10.5
centimeters and a width of 5.3 ems:
\begin{verbatim}
Height : 10.5 cm;
Width : 5.3;
\end{verbatim}
\end{example}

\subsection{Relative positions}
\label{position}

The positioning of boxes uses the eight axes and sides, the sides
generally being used to define the juxtapositioning (vertical or
horizontal) of boxes, the middle axes being used to define centering,
and the reference axes being used for alignment.

Two rules allow a box to placed relative to other boxes.  The {\tt
VertPos} rule positions the box vertically.  The {\tt HorizPos} rule
positions the box horizontally.  It is possible that a box's position
could be entirely determined by other boxes positioned relative to it.
In this case, the position is implicit and the word {\tt nil} can be
used to specify that no position rule is needed.  Otherwise, an
explicit rule must be given by indicating the axis or side which
defines the position of the box, followed by an equals sign and the
distance between between this axis or side and a parallel axis or side
of another box, called the reference box.  The box for which the rule
is written will be positioned relative to the reference box.

\begin{verbatim}
                'VertPos' ':' VPos
                'HorizPos' ':' HPos
     HPos     = 'nil' / VertAxis '=' HorizPosition
                [ 'UserSpecified' ].
     VPos     = 'nil' / HorizAxis '=' VertPosition
                [ 'UserSpecified' ].
     VertAxis  = 'Left' / 'VMiddle' / 'VRef' / 'Right' .
     HorizAxis = 'Top' / 'HMiddle' / 'HRef' / 'Bottom' .
\end{verbatim}

The reference box is an adjacent box: enclosing, enclosed or adjacent.
When a rule is associated with a reference type attribute (and only in
this case), it can be a box of the element designated by the
attribute.  The reference box can be either a presentation box
previously defined in the {\tt BOXES} section of the schema and
created by a creation function, or the box associated with a
structured element.

The structural position of the reference box (relative to the box for
which the rule is being written) is indicated by a keyword: {\tt
Enclosing}, {\tt Enclosed}, or, for sibling boxes, {\tt Previous} or
{\tt Next}.  The reference attributes, or presentation boxes created
by a reference attribute, the {\tt Referred} keyword may be used to
designate the element which the reference points to.  The keyword {\tt
Creator} can be used in rules for presentation boxes to designate the
box of the element which created the presentation box.  Finally, the {\tt Root}
keyword can be used to designate the root of the document.

When the keyword is ambiguous, it is followed by a name of a type or
presentation box which resolves the ambiguity (the {\tt Creator} and
{\tt Root} keywords are never ambiguous).  If this name is not given,
then the first box encountered is used as the reference box. It is
also possible to use just the name of a type or presentation box
without an initial keyword.  In this case, a sibling having that name
will be used.  If the name is preceded by the keyword {\tt NOT}, then
the reference box will be the first box whose type is not the named
one.  In place of the box or type name, the keywords {\tt AnyElem} and
{\tt AnyBox} can be used, representing respectively, any structured
element box and any presentation box.  A type name may be preceded by
a star in order to resolve the ambiguity in the special case where the
structure schema defines an inclusion without expansion (or with
partial expansion) of the same type as an element of the scheme (see
section~\ref{references}).  For mark pairs (and only for mark pairs,
see section~\ref{paires}), the type name {\em must} be preceded by the
{\tt First} or {\tt Second} keyword, which indicates which of the two
marks of the pair  should be used as the reference box.

The star character ('{\tt *}') used alone designates the box to which
the rule applies (in this case, it is obviously useless to specify the
type of the reference box).

The keywords {\tt Enclosing} and {\tt Enclosed} can be used no matter
what constructor defines the type to which the rule applies.  When
applied to the element which represents the entire document, {\tt
Enclosing} designates the window or page in which the document's image
is displayed for the view to which the rule applies.  A box or type
name without a keyword is used for aggregate elements and designates
another element of the same aggregate.  It can also be used to
designate a presentation or page layout box.  The keywords {\tt
Previous} and {\tt Next} are primarily used to denote list elements,
but can also be used to denote elements of an aggregate.

In the position rule, the structural position relative to the
reference box is followed, after a period, by the name of an axis or
side.  The rule specifies its node's position as being some distance
from this axis or side of the reference box.  If this distance is
zero, then the distance does not appear in the rule.  Otherwise, it
does appear as a positive or negative number (the sign is required for
negative numbers). The sign takes into account the orientation of the
coordinate axes: for top to bottom for the vertical axis and from left
to right for the horizontal axis.  Thus, a negative distance in a
vertical position indicates that the side or axis specified in the rule
is above the side or axis of the reference box. 

The distance can be followed by the {\tt UserSpecified} keyword (even
if the distance is nil and does not appear, the {\tt UserSpecified}
keyword can be used).  It indicates that when the element to which the
rule applies is being created, the editor will ask the user to specify
the distance himself, using the mouse.  In this case, the distance
specified in the rule is a default distance which is suggested to the
user but can be modified.  The {\tt UserSpecified} keyword can be used
either in the vertical position rule, the horizontal position rule, or
both.

\begin{verbatim}
     VertPosition  = Reference '.' HorizAxis [ Distance ] .
     HorizPosition = Reference '.' VertAxis [ Distance ] .
     Reference     ='Enclosing' [ BoxTypeNot ] /
                    'Enclosed' [ BoxTypeNot ] /
                    'Previous' [ BoxTypeNot ] /
                    'Next' [ BoxTypeNot ] /
                    'Referred' [ BoxTypeNot ] /
                    'Creator' /
                    'Root' /
                    '*' /
                     BoxOrType .
     BoxOrType     = BoxID /
                     [ '*' ] [ FirstSec ] ElemID /
                    'AnyElem' / 'AnyBox' .
     BoxTypeNot    = [ 'NOT' ] BoxOrType .
\end{verbatim}

\label{expos1}
\begin{example}
If a report is defined by by the following structure schema:

\begin{verbatim}
Report = BEGIN
          Title  = Text;
          Summary = Text;
          Keywords = Text;
          ...
          END;
\end{verbatim}
then the presentation schema could contain the rules:

\label{exemplerapp}
\begin{verbatim}
Report : BEGIN
          VertPos  : Top = Enclosing . Top;
          HorizPos : Left = Enclosing . Left;
          ...
          END;
\end{verbatim}
These rules place the report in the upper left corner of the enclosing
box, which is the window in which the document is being edited.

\begin{verbatim}
Title :   BEGIN
          VertPos  : Top = Enclosing . Top + 1;
          HorizPos : VMiddle = Enclosing . VMiddle;
          ...
          END;
\end{verbatim}
The top of the title is one line (a line has the height of the
characters of the title) from the top of the report, which is also the
top of the editing window.  The title is centered horizontally in the
window (see figure~\ref{posdim}).

\begin{verbatim}
Summary :  BEGIN
          VertPos  : Top = Title . Bottom + 1.5;
          HorizPos : Left = Enclosing . Left + 2 cm;
          ...
          END;
\end{verbatim}
The top of the summary is place a line and a half below the bottom of
the title and is shifted two centimeters from the side of the window.
\end{example}

\label{expos2}
\begin{example}
Suppose there is a Design logical structure which contains graphical elements:

\begin{verbatim}
Design = LIST OF (ElemGraph = GRAPHICS);
\end{verbatim}

The following rules allow the user to freely choose the position of
each element when it is created:
\begin{verbatim}
ElemGraph =
   BEGIN
   VertPos : Top = Enclosing . Top + 1 cm UserSpecified;
   HorizPos: Left = Enclosing . Left UserSpecified;
   ...
   END;
\end{verbatim}
Thus, when a graphical element is created, its default placement is at
the left of the window and 1 cm from the top, but the user can move it
immediately, simply by moving the mouse.

\end{example}

\subsection{Box extents}
\label{dimension}

The extents (height and width) of each box are defined by the two
rules {\tt Height} and {\tt Width}.  There are three types of extents:
fixed, relative, and elastic.

\subsubsection{Fixed extents}

A fixed dimension sets the height or width of the box independently of
all other boxes.  It is expressed in distance units (see
page~\pageref{unites}).  The extent can be followed by the {\tt
UserSpecified} keyword which indicates that when the element to which
the rule applies is being created, the editor will ask the user to
specify the extent himself, using the mouse.  In this case, the extent
specified in the rule is a default extent which is suggested to the
user but can be modified.  The {\tt UserSpecified} keyword can be used
either in the {\tt Height} rule, the {\tt Width} rule, or both.

A fixed extent rule can be ended by the {\tt Min} keyword, which
signifies that the indicated value is a minimum, and that, if the
contents of the box require it, a larger extent is possible.

\begin{verbatim}
                'Height' ':' Dimension
                'Width' ':' Dimension
     Dimension = AbsDist [ 'UserSpecified' ]  [ 'Min' ] /
                 ...
\end{verbatim}

\begin{example}
Continuing with the example of  page~\pageref{expos2}, it is possible
to allow the user to choose the size of each graphical element as it
is created:

\begin{verbatim}
ElemGraph : BEGIN
            Width : 2 cm UserSpecified;
            Height : 1 cm UserSpecified;
            ...
            END;
\end{verbatim}

Thus, when a graphical element is create, it is drawn by default with
a width of 2 cm and a height of 1 cm, but the user is free to resize
it immediately with the mouse.

\begin{verbatim}
Summary :  BEGIN
           Height : 5 cm Min;
           ...
           END;
Keywords : BEGIN
           VertPos : Top = Summary . Bottom;
           ...
           END;
\end{verbatim}
\end{example}

\subsubsection{Relative extents}

A relative extent determines the extent as a function of the extent of
another box, just as a relative position places a box in relation to
another.  The reference box in an extent rule is designated using the
same syntax as is used in a relative position rule.  It is followed by
a period and a {\tt Height} or {\tt Width} keyword, depending on the
extent being referred to.  Next comes the relation between the
extent being defined and the extent of the reference box.  This
relation can be either a percentage or a difference.

A percentage is indicated by a star (the multiplication symbol)
followed by the numeric percentage value (which may be greater than or
less than 100) and the percent (`\%') character.  A difference
is simply indicated by a signed difference.

If the rule appears in the presentation rules of a numeric attribute,
the percentage value can be replaced by the name of the attribute.
This attribute is then used as a percentage.  The attribute can also
be used as part of a difference (see page~\pageref{distance}).

Just as with a fixed extent, a relative extent rule can end with the
{\tt Min} keyword, which signifies that the extent is a minimum and
that, if the contents of the box require it, a larger extent is
possible.

A special case of relative extent rules is:
\begin{verbatim}
Height : Enclosed . Height;
\end{verbatim}
or
\begin{verbatim}
Width  : Enclosed . Width;
\end{verbatim}
which specifies that the box has a height (or width) such that it
encloses all the boxes which it contains.

{\bf Note:} character strings (type {\tt TEXT\_UNIT}) generally must
use the sum of the widths of the characters which compose them as
their width, which is expressed by the rule:
\begin{verbatim}
TEXT_UNIT :
   Width  : Enclosed . Width;
\end{verbatim}
If this rule is not the default {\tt Width} rule, it must be given
explicitly in the {\tt RULES} section which defines the presentation
rules of the logical elements.

\begin{verbatim}
                  'Height' ':' Extent
                  'Width' ':' Extent
     Extent      = Reference '.' HeightWidth [ Relation ]
                   [ 'Min' ] / ...
     HeightWidth ='Height' / 'Width' .
     Relation    ='*' ExtentAttr '%' / Distance .
     ExtentAttr  = ExtentVal / AttrID .
     ExtentVal   = NUMBER .
\end{verbatim}

\begin{example}
Completing the example of page~\pageref{expos1}, it is possible to
specify that the report takes its width from the editing window and
its height from the size of its contents (this can obviously be
greater than that of the window):

\begin{verbatim}
Report :  BEGIN
          Width : Enclosing . Width;
          Height : Enclosed . Height;
          ...
          END;
\end{verbatim}
Then, the following rules make the title occupy 60\% of the width of
the report (which is that of the window) and is broken into centered
lines of this width (see the {\tt Line} rule, page~\pageref{regleline}).

\begin{verbatim}
Title :   BEGIN
          Width : Enclosing . Width * 60%;
          Height : Enclosed . Height;
          Line;
          Adjust : VMiddle;
          ...
          END;
\end{verbatim}
The summary occupy the entire width of the window, with the exception
of a 2 cm margin reserved by the horizontal position rule:

\begin{verbatim}
Summary : BEGIN
          Width : Enclosing . Width - 2 cm;
          Height : Enclosed . Height;
          ...
          END;
\end{verbatim}
This set of rules, plus the position rules given on
page~\pageref{exemplerapp}, produce the layout of boxes shown in
figure~\ref{posdim}.

\begin{figure}
\begin{center}
\setlength{\unitlength}{1 mm}
\begin{picture}(100,54)
\thicklines
\multiput(0,0)(0,2){2}{\line(0,1){1}}
\multiput(15,0)(0,2){2}{\line(0,1){1}}
\multiput(100,0)(0,2){2}{\line(0,1){1}}
\put(0,4){\line(0,1){25}}
\put(15,4){\line(0,1){7}}
\put(15,11){\line(1,0){85}}
\put(15,0){\makebox(85,11){Summary}}
\put(100,4){\line(0,1){25}}
\put(20,26){\line(0,1){3}}
\put(80,26){\line(0,1){3}}
\put(20,26){\line(1,0){60}}
\multiput(0,29)(0,2){6}{\line(0,1){1}}
\multiput(20,29)(0,2){6}{\line(0,1){1}}
\multiput(80,29)(0,2){6}{\line(0,1){1}}
\multiput(100,29)(0,2){6}{\line(0,1){1}}
\put(20,41){\line(0,1){3}}
\put(80,41){\line(0,1){3}}
\put(20,44){\line(1,0){60}}
\put(20,34){\makebox(60,10){Title}}
\put(0,41){\line(0,1){13}}
\put(100,41){\line(0,1){13}}
\put(0,54){\line(1,0){100}}
\put(3,51){\makebox(0,0)[tl]{Window and Report}}
\thinlines
\put(7.5,5.5){\vector(-1,0){7}}
\put(7.5,5.5){\vector(1,0){7}}
\put(7.5,7){\makebox(0,0)[b]{2 cm}}
\put(50,18.5){\vector(0,1){7}}
\put(50,18.5){\vector(0,-1){7}}
\put(51.5,18.5){\makebox(0,0)[l]{1.5 lines}}
\put(50,29){\vector(-1,0){29.5}}
\put(50,29){\vector(1,0){29.5}}
\put(50,30.5){\makebox(0,0)[b]{60\%}}
\put(10,35){\vector(-1,0){9.5}}
\put(10,35){\vector(1,0){9.5}}
\put(10,36.5){\makebox(0,0)[b]{20\%}}
\put(90,35){\vector(-1,0){9.5}}
\put(90,35){\vector(1,0){9.5}}
\put(90,36.5){\makebox(0,0)[b]{20\%}}
\put(50,49){\vector(0,1){4.5}}
\put(50,49){\vector(0,-1){4.5}}
\put(51.5,49){\makebox(0,0)[l]{1 line}}
\end{picture}
\end{center}
\caption{Box position and extent}
\label{posdim}
\end{figure}
\end{example}

\subsubsection{Elastic extents}

The last type of extent is the elastic extent.  Either one or both
extents can be elastic.  A box has an elastic extent when two opposite
sides are linked by distance constraints to two sides or axes of other
boxes.

One of the sides of the elastic box is linked by a position rule ({\tt
VertPos} or {\tt HorizPos}) to a neighboring box.  The other side is
link to another box by a {\tt Height} or {\tt Width} rule, which takes
the same form as the position rule.  For the elastic box itself, the
notions of sides (left or right, top or bottom) are fuzzy, since the
movement of either one of the two reference boxes can, for example,
make the left side of the elastic box move to the right of its right
side.  This is not important.  The only requirement is that the two
sides of the elastic box used in the position and extent rule are
opposite sides of the box.

\begin{verbatim}
                  'Height' ':' Extent
                  'Width' ':' Extent
     Extent   = HPos / VPos / ...
\end{verbatim}

\begin{example}
Suppose we want to draw an elastic arrow or line between the middle of
the bottom side of box A and the upper left corner of box B.  To do
this, we would define a graphics box whose upper left corner coincides
with the middle of the bottom side of A (a position rule) and whose
lower right corner coincides with with the upper left corner of B
(dimension rules):

\begin{verbatim}
LinkedBox :
   BEGIN
   VertPos : Top = A .Bottom;
   HorizPos : Left = A . VMiddle;
   Height : Bottom = B . Top;
   Width : Right = B . Left;
   END;
\end{verbatim}
\end{example}

\begin{example}
The element SectionTitle creates a presentation box called SectionNum
which contains the number of the section.  Suppose we want to align
the SectionNum and SectionTitle horizontally, have the SectionNum take
its width from its contents (the section number), have the
SectionTitle box begin 0.5 cm to the right of the SectionNum box and
end at the right edge of its enclosing box.  This would make the
SectionTitle box elastic, since its width is defined by the position
of its left and right sides.  The following rules produce this effect:

\begin{verbatim}
SectionNum :
   BEGIN
   HorizPos : Left = Enclosing . Left;
   Width : Enclosed . Width;
   ...
   END;

SectionTitle :
   BEGIN
   HorizPos : Left = SectionNum . Right + 0.5 cm;
   Width : Right = Enclosing . Right;
   ...
   END;
\end{verbatim}
\end{example}

\subsection{Inheritance}

A presentation parameter can be defined by reference to the same
parameter of another box in the tree of boxes.  These structural links
are expressed by kinship.  The reference box can be that of the
element immediately above in the structure ({\tt Enclosing}), two
levels above ({\tt GrandFather}), immediately below ({\tt Enclosed})
or immediately before ({\tt Previous}).  In the case of a presentation
box, and only in that case, the reference box may be the element which
created the presentation box ({\tt Creator}).

Kinship is expressed in terms of the logical structure of the
document and not in terms of the tree of boxes.  The presentation box
cannot transmit any of their parameters by inheritance; only
structured element boxes can do so.  As an example, consider an
element B which follows an element A in the logical structure.  The
element B creates a presentation box P in front of itself, using the
{\tt CreateBefore} rule (see the creation rules,
page~\pageref{creation}).  If element B's box inherits its character
style using the {\tt Previous} kinship operation, it gets its
character style from A's box, not from P's box.  Inheritance works
differently for positions and extents, which can refer to presentation
boxes.

The inherited parameter value can be the same as that of the reference
box.  This is indicated by an equals sign.  However, for numeric
parameters, a different value can be obtained by adding or subtracting
a number from the reference box's parameter value.  Addition is
indicated by a plus sign before the number, while subtraction is
specified with a minus sign.  The value of a
parameter can also be given a maximum (if the sign is a plus) or
minimum (if the sign is a minus).

If the rule is being applied to a numeric attribute, the number to add
or subtract can be replaced by the attribute name.  The value of a
maximum or minimum may also be replaced by an attribute name.  In
these cases, the value of the attribute is used.

\begin{verbatim}
  Inheritance   = Kinship  InheritedValue .
  Kinship       ='Enclosing' / 'GrandFather' / 'Enclosed' /
                 'Previous' / 'Creator' .
  InheritedValue ='+' PosIntAttr [ 'Max' maximumA ] /
                 '-' NegIntAttr [ 'Min' minimumA ] /
                 '=' .
  PosIntAttr    = PosInt / AttrID .
  PosInt        = NUMBER .
  NegIntAttr    = NegInt / AttrID .
  NegInt        = NUMBER .
  maximumA      = maximum / AttrID .
  maximum       = NUMBER .
  minimumA      = minimum / AttrID .
  minimum       = NUMBER .
\end{verbatim}

The parameters which can be obtained by inheritance are justification,
hyphenation, interline spacing, character font (font family), font
style, font size, visibility, indentation, underlining, alignment of
text, stacking order of objects, the style and thickness of lines,
fill pattern and the colors of lines and characters.

\subsection{Line breaking}
\label{regleline}

The {\tt Line} rule specifies that the contents of the box should be
broken into lines: the boxes included in the box to which this rule is
attached are displayed one after the other, from left to right, with
their horizontal reference axes aligned so that they form a series of
lines.  The length of these lines is equal to the width of the box to
which the {\tt Line} rule is attached.

When an included box overflows the current line, it is either carried
forward to the next line, cur, or left the way it is.  The {\tt
LineBreak} rule (see page~\pageref{condcoupure}) is used to allow or
prevent the breaking of included boxes.  If the included box is not
breakable but is longer than the space remaining on the line, it is
left as is.  When a character string box is breakable, the line is
broken between words or, if necessary, by hyphenating a word (see
page~\pageref{reglehyphenate}).  When a compound box is breakable, the
box is transparent in regard to line breaking.  The boxes included in
the compound box are treated just like included boxes which have the
{\tt LineBreak} rule.  Thus, it is possible to traverse a complete
subtree of boxes to line break the text leaves of a complex structure.

The relative position rules of the included boxes are ignored, since the
boxes will be placed according to the line breaking rules.

The {\tt Line} rule does not have a parameter.  The characteristics of
the lines that will be constructed are determined by the {\tt
LineSpacing}, {\tt Indent}, {\tt Adjust}, {\tt Justify}, and {\tt
Hyphenate} rules.  Moreover, the {\tt Inline} rule (see
page~\pageref{regleinline}) permits the exclusion of certain elements
from the line breaking process.

When the {\tt Line} rule appears in the rules sequence of a
non-primary view, it applies only to that view, but when the {\tt
Line} rule appears in the rules sequence of the primary view, it also
applies to the other views by default, except for those views which
explicitly invoke the {\tt NoLine} rule.  Thus, the {\tt NoLine} rule
can be used in a non-primary view to override the primary view's {\tt
Line} rule.  The {\tt NoLine} rule must not be used with the primary
view because the absence of the {\tt Line} rule has the same effect.
Like the {\tt Line} rule, the {\tt NoLine} rule does not take any
parameters.

\begin{verbatim}
                 'Line'
                 'NoLine'
\end{verbatim}

\subsubsection{Line spacing}
\label{interligne}

The {\tt LineSpacing} rule defines the line spacing to be used in the
line breaking process.  The line spacing is the distance between the
baselines (horizontal reference axis) of the successive lines produced
by the {\tt Line} rule.  The value of the line spacing can be
specified as a constant or by inheritance.  It is expressed in any of
the available distance units (see page~\pageref{unites}).

Inheritance allows the value to be obtained from a relative in the
structure tree, either without change (an equals sign appears after
the inheritance keyword), with a positive difference (a plus sign), or
a negative difference (a minus sign).  When the rule uses a
difference, the value of the difference  follows the sign and is
expressed as a distance (see page~\pageref{unites}).

\begin{verbatim}
                      'LineSpacing' ':' DistOrInherit
     DistOrInherit =   Kinship InheritedDist / Distance .
     InheritedDist = '=' / '+' AbsDist / '-' AbsDist .
\end{verbatim}

When the line spacing value (or its difference from another element)
is expressed in relative units, it changes with the size of the
characters.  Thus, when a larger font is chosen for a part of the
document, the line spacing of that part expands proportionally.  In
contrast, when the line spacing value is expressed in absolute units
(centimeters, inches, typographer's points), it is independent of the
characters, which permits the maintenance of a consistent line
spacing, whatever the character font.  Either approach can be taken,
depending on the desired effect.

\subsubsection{First line indentation}

The {\tt Indent} rule is used to specify the indentation of the first
line of the elements broken into lines by the {\tt Line} function.
The indentation determines how far the first line of the element is
shifted with respect to the other lines of the same element.  It can
be specified as a constant or by inheritance.  The constant value is
a positive integer (shifted to the right; the sign is optional), a
negative integer (shifted to the left) or zero (no shift).  All the
units described on page~\pageref{unites} can be used.

Indentation can be defined for any box, regardless of whether the box
is line broken, and transmitted by inheritance to elements that are
line broken.  The size of the indentation is specified in the same
manner as the line spacing (see section~\ref{interligne}).

\begin{verbatim}
                 'Indent' ':' DistOrInherit
\end{verbatim}

\subsubsection{Alignment}

The alignment style of the lines constructed during line breaking is
defined by the {\tt Adjust} rule.  The alignment value can be a
constant or inherited.  A constant value is specified by a keyword:
\begin{itemize}
   \item {\tt Left} : at the left edge,
   \item {\tt Right} : at the right edge,
   \item {\tt VMiddle} : centered
   \item {\tt LeftWithDots} : at the left edge with a dotted line
filling out the last line up to the right edge of the line breaking box.
\end{itemize}

An inherited value can only be the same as that of the reference box
and is specified by a kinship keyword followed by an equals sign.

\begin{verbatim}
                  'Adjust' ':' AlignOrInherit
     AlignOrInherit = Kinship '=' / Alignment .
     Alignment      = 'Left' / 'Right' / 'VMiddle' /
                      'LeftWithDots' .
\end{verbatim}

\subsubsection{Justification}

The {\tt Justify} rule indicates whether the lines contained in the
box and produced by a {\tt Line} rule should be extended horizontally
to occupy the entire width of their enclosing box.  The first and last
lines are treated specially: the position of the beginning of the
first line is fixed by the {\tt Indent} rule and last line is not
extended.  The justification parameter defined by this rule takes a
boolean value, which can be a constant or inherited.  A constant
boolean value is expressed by either the {\tt Yes} or the {\tt No}
keyword.  An inherited value can only be the same as that of the
reference box and is specified by a kinship keyword followed by an
equals sign.

\begin{verbatim}
                   'Justify' ':' BoolInherit
     BoolInherit = Boolean / Kinship '=' .
     Boolean      ='Yes' / 'No' .
\end{verbatim}

When the lines are justified, the alignment parameter specified in the
{\tt Adjust} rule has no influence, other than on the last line
produced.  This occurs because, when the other are extended to the
limits of the box, the alignment style is no longer perceptible.

\begin{example}
An important use of inheritance is to vary the characteristics of
lines for an element type (for example, Paragraph) according to the
enclosing environment (for example, Summary or Section), and thus
obtain different line breaking styles for the same elements when they
appear in different environments.  The following rules specify that
paragraphs inherit their alignment, justification, and line spacing:
\begin{verbatim}
Paragraph :
   BEGIN
   Justify : Enclosing = ;
   LineSpacing : Enclosing = ;
   Adjust : Enclosing =;
   Line;
   END;
\end{verbatim}
If the alignment, justification, and line spacing of the Section
and Summary  elements is fixed:
\begin{verbatim}
Section :
   BEGIN
   Adjust : Left;
   Justify : Yes;
   LineSpacing : 1;
   END;
Summary :
   BEGIN
   Adjust : VMiddle;
   Justify : No;
   LineSpacing : 1.3;
   END;
\end{verbatim}
then the paragraphs appearing in sections are justified with a simple
line spacing while those appearing in summaries are centered and not
justified and have a larger line spacing.  These are nevertheless the
very same type of paragraph defined in the logical structure schema.
\end{example}

\subsubsection{Hyphenation}
\label{reglehyphenate}

The {\tt Hyphenate} rule indicates whether or not words should be
broken by hyphenation at the end of lines.  It affects the lines
produced by the {\tt Line} rule and contained in the box carrying the
{\tt Hyphenate} rule.

The hyphenation parameter takes a boolean value, which can be either
constant or inherited.  A constant boolean value is expressed by
either the {\tt Yes} or the {\tt No} keyword.  An inherited value can
only be the same as that of the reference box and is specified by a
kinship keyword followed by an equals sign.

\begin{verbatim}
                   'Hyphenate' ':' BoolInherit
     BoolInherit = Boolean / Kinship '=' .
     Boolean      ='Yes' / 'No' .
\end{verbatim}

\subsubsection{Avoiding line breaking}
\label{regleinline}

The {\tt InLine} rule is used to specify that a box that would
otherwise participate in line breaking asked for by the {\tt Line}
rule of an enclosing box, instead avoids the line breaking process and
positions itself  according to the {\tt HorizPos} and {\tt VertPos}
rules that apply to it.  When the {\tt InLine} rule applies to a box
which would not be line broken, it has no effect.

The rule is expressed by the {\tt InLine} keyword followed by a colon
and the keyword {\tt Yes}, if the box should participate in line
breaking, or the keyword {\tt No}, if it should not.  This is the only
form possible: this rule cannot be inherited.  Moreover, it can only
appear in the rules of the primary view and applies to all views
defined in the presentation schema.

\begin{verbatim}
               'InLine' ':' Boolean .
     Boolean = 'Yes' / 'No' .
\end{verbatim}

\begin{example}
Suppose the structure schema defines a logical attribute called {\tt
New} which is used to identify the passages in a document which were
recently modified.  It would be nice to have the presentation schema
make a bar appear in the left margin next to each passage having the
{\tt New} attribute.  A new passage can be an entire element, such as
a paragraph or section, or it can be some words in the middle of a
paragraph.  To produce the desired effect, the {\tt New} attribute is
given a creation rule which generates a {\tt VerticalBar} presentation
box.

When the {\tt New} attribute is attached to a character string which
is inside a line broken element (inside a paragraph, for example), the
bar is one of the elements which participates in line breaking and it
is placed normally in the current line, at the end of the character
string which has the attribute.  To avoid this, the {\tt InLine} rule
is used in the following way:
 
\begin{verbatim}
BOXES
  VerticalBar:
     BEGIN
     Content: Graphics 'l';
     HorizPos: Left = Root . Left;
     VertPos: Top = Creator . Top;
     Height: Bottom = Creator . Bottom;
     Width: 1 pt;
     InLine: No;
     ...
     END;
...
ATTRIBUTES
  Nouveau:
     BEGIN
     CreateAfter(VerticalBar);
     END;
\end{verbatim}
\end{example}

\subsection{Page breaking and line breaking conditions}
\label{condcoupure}

Pages are constructed by the editor in accordance with the model
specified by a {\tt Page} rule (see page~\pageref{page}).  The page
model describes only the composition of the pages but does not give
any rules for breaking different element types across pages.  Now, it
is possible that certain elements must not be cut by page breaks,
while others can be cut anywhere.  The {\tt PageBreak}, {\tt
NoBreak1}, and {\tt NoBreak2} rules are used to specify the conditions
under which each element type can be cut.

The {\tt PageBreak} rule is used to indicate whether or not the box
can be cut during the construction of pages.  If cutting is
authorized, the box can be cut, with one part appearing at the bottom
of a page and the other part appearing at the top of the next page.
The rule is formed by the {\tt PageBreak} keyword followed by a colon
and a constant boolean value ({\tt Yes} or {\tt No}).  This is the only
form possible: this rule cannot be inherited.  Moreover, it can only
appear in the rules of the primary view and applies to all views
defined in the presentation schema.

Whether objects can be cut by line breaks can be controlled in a
similar way using the {\tt LineBreak} rule.  This rule allows the
specification of whether or not the box can be cut during the
construction of lines.  If cutting is authorized, the box can be cut,
with one part appearing at the end of a line and the other part
appearing at the beginning of the next line.  The rule is formed by
the {\tt LineBreak} keyword followed by a colon and a constant boolean
value ({\tt Yes} or {\tt No}).  This is the only form possible: this
rule cannot be inherited.  Moreover, it can only appear in the rules
of the primary view and applies to all views defined in the
presentation schema.

\begin{verbatim}
               'PageBreak' ':' Boolean .
               'LineBreak' ':' Boolean .
     Boolean = 'Yes' / 'No' .
\end{verbatim}

When a box can be cut by a page break, it is possible that a page
break will fall an inappropriate spot, creating, for example, a widow
or orphan, or separating the title of a section from the first
paragraph of the section.  The {\tt NoBreak1} and {\tt NoBreak2} rules
are used to avoid this.  They specify that the box of the element type
to which they apply cannot be cut within a certain zone above ({\tt
NoBreak1} rule) or below ({\tt NoBreak2} rule).  These two rules
specify the height of the zones in which page breaks are prohibited.

The {\tt NoBreak1} and {\tt NoBreak2} rules give the height of the
zone in which page breaking is prohibited.  The height is given as a
constant value using any of the available units, absolute or relative
(see page~\pageref{unites}).  The value may not be inherited.

\begin{verbatim}
                   'NoBreak1' ':' AbsDist .
                   'NoBreak2' ':' AbsDist .
\end{verbatim}

\begin{example}
The following rules  prevent widows and orphans in a paragraph:
\begin{verbatim}
Paragraph :
   BEGIN
   NoBreak1 : 2;
   NoBreak2 : 2;
   END;
\end{verbatim}
This rule prevents a section title from becoming separated from the
first paragraph of the section by prohibiting page breaks at the
beginning of the section rule:
\begin{verbatim}
Section :
   NoBreak1 : 1.5 cm;
\end{verbatim}

Finally, this rule prevents a figure from being page broken in any way:
\begin{verbatim}
Figure :
   PageBreak : No;
\end{verbatim}
\end{example}

The Thot editor constructs the document images displayed on the screen
dynamically.  As the user moves in the document or makes the document
scroll in a window, the editor constructs the image to be displayed in
little bits, filling the gaps which are produced in the course of
these operations.  It stops filling in the image when an element
reaches the edge of the window in which the gap appears.  If the
appearance of the document is complex, it is possible that the image
in incomplete, even though the edge of the window was reached.  For
example, an element might need to be presented to the side of the last
element displayed, but its image was not constructed.  The user will
not know whether the element is really absent or if its image has
simply not been constructed.

The {\tt Gather} rule is used to remedy this problem.  When the rule
{\tt Gather : Yes;} is associated with an element type, the image of
such elements is constructed as a block by the editor: it is never
split up.

The {\tt Gather} rule may not appear in the default rules (see
page~\pageref{reglesdefaut}).  Elements which do not have the {\tt
Gather} rule are considered susceptible to being split up during
display.  Thus, it is not necessary to use the {\tt Gather : No;}
form.  This rule must be used prudently and only for those elements
which truly need it.  If used incorrectly, it can pointlessly increase
the size of the image constructed by the editor and lead to excessive
memory consumption by the editor.

Like the {\tt PageBreak} and {\tt LineBreak} rules, the {\tt Gather}
rule can only appear in rules of the primary view and applies to all
views defined in the presentation schema.

\begin{verbatim}
                   'Gather' ':' Boolean .
\end{verbatim}

\subsection{Visibility}
\label{visib}

The visibility parameter is used to control which elements should or
should not be displayed, based on context.  An element can have
different visibilities in different views.  If an element's visibility
is zero for a view, that element is not displayed in that view and
does not occupy any space (its extents are zero).

Visibility takes non-negative integer values (positive or zero).  If
values greater than 1 are used, they allow the user to choose a degree
of visibility and, thus, to see only those boxes whose visibility
parameter exceeds a certain threshold.  This gives the user control
over the granularity of the displayed images.

The visibility parameter can be defined as a constant or by
inheritance.  If defined by inheritance, it cannot be based on the
value of the next or previous box.  Visibility can only be inherited
from above.

If it is a numeric attribute's presentation rule, the visibility can
be specified by the attribute's name, in which case the value of the
attribute is used.

\begin{verbatim}
                  'Visibility' ':' NumberInherit
     NumberInherit = Integer / AttrID / Inheritance .
     Integer      = NUMBER .
\end{verbatim}

\begin{example}
Suppose that only {\tt Formula} elements should be displayed in the
{\tt MathView} view.  Then, the default rules should include:
\begin{verbatim}
DEFAULT
     IN MathView Visibility:0;
\end{verbatim}
which makes all elements invisible in the {\tt MathView} view.
However, the {\tt Formula} element also has a {\tt Visibility} rule:

\begin{verbatim}
Formula :
     IN MathView Visibility:5;
\end{verbatim}
which makes formulas, and only formulas, visible.
\end{example}

\subsection{Character style parameters}

Four parameters are used to determine which characters are used to
display text.  They are size, font, style, and underlining.

\subsubsection{Character size}

The size parameter has two effects.  First, it is used to specify the
actual size and distance units for boxes defined in relative units
(see page~\pageref{unites}).  Second, it defines the size of the
characters contained in the box.

As a distance or length, the size can be expressed in abstract or
absolute units.  It can also be inherited.  If it is not inherited, it
is expressed simply as an integer followed by the {\tt pt} keyword,
which indicates that the size is expressed in typographer's points.
The absence of the {\tt pt} keyword indicates that it is in abstract
units in which the value 1 represents the smallest size while the
value 16 is the largest size.  The relationship between these abstract
sizes and the real character sizes is controlled by a table which can
be modified statically or even dynamically during the execution of the
Thot editor.

If it is a numeric attribute's presentation rule, the value of the
size parameter can be specified by the attribute's name, in which case
the value of the attribute is used.

{\bf Note:} the only unit available for  defining an absolute size is
the typographer's point.  Centimeters and inches may not be used.

If the size is inherited, the rule must specify the relative from
which to inherit and any difference from that relative's value.  The
difference can be expressed in either typographer's points or in
abstract units.  The maximum or minimum size can also be specified,
but without specifying the type of unit: it is the same as was
specified for the difference.

In a numeric attribute's presentation rule, the difference in size can
be indicated by the attribute's name, which means that the attribute's
value should be used as the difference.  The attribute can also be
used as the minimum or maximum size.

\begin{verbatim}
                    'Size' ':' SizeInherit
     SizeInherit   = SizeAttr [ 'pt' ] /
                     Kinship InheritedSize .
     InheritedSize ='+' SizeAttr [ 'pt' ]
                     [ 'Max' MaxSizeAttr ] /
                    '-' SizeAttr [ 'pt' ]
                     [ 'Min' MinSizeAttr ] /
                    '=' .
     SizeAttr       = Size / AttrID .
     Size        = NUMBER .
     MaxSizeAttr    = MaxSize / AttrID .
     MaxSize     = NUMBER .
     MinSizeAttr    = MinSize / AttrID .
     MinSize     = NUMBER .
\end{verbatim}

\begin{example}
The rule 
\begin{verbatim}
Size : Enclosing - 2 pt Min 7;
\end{verbatim}
states that the character size is 2 points less than that of the
enclosing box, but that it may not be less than 7 points, whatever the
enclosing box's value.

The following rules make the text of a report be displayed with
medium-sized characters (for example, size 5), while the title is
displayed with larger characters and the summary is displayed with
smaller characters:

\begin{verbatim}
Report :
     Size : 5;
Title :
     Size : Enclosing + 2;
Summary :
     Size : Enclosing - 1;
\end{verbatim}
Thus, the character sizes in the entire document can be changed by
changing the size parameter of the Report element, while preserving
the relationships between the sizes of the different elements.
\end{example}

\subsubsection{Font and character style}
\label{style}

The {\tt Font} rule determines the font family to be used to display
the characters contained in the box, while the {\tt Style} rule
determines their style.  Thot recognizes three character fonts (Times,
Helvetica, and Courier) and six styles: Roman, Italics, Bold,
BoldItalics, Oblique, and BoldOblique.  

The font family and style can specified by a named constant or can be
inherited.  For the name of the font family only the first character
is used.

Only identical inheritance is allowed: the box takes the same font or
style as the box from which it inherits.  This is indicated by an
equals sign after the kinship specification.

\begin{example}
To specify that the summary uses the font family of the rest of the
document, but in the italic style, the following rules are used:

\begin{verbatim}
Summary :
   BEGIN
   Font : Enclosing =;
   Style : Italics;
   END;
\end{verbatim}
\end{example}

\subsubsection{Underlining}
\label{underline}

The {\tt Underline} rule is used to specify if the characters
contained in a box should have lines drawn on or near them.  There are
four underlining styles: {\tt Underlined}, {\tt Overlined}, {\tt
CrossedOut}, and {\tt NoUnderline}.  The {\tt Thickness} rule
specifies the thickness of the line, {\tt Thin} or {\tt Thick}.

As with font family and style, only identical inheritance is allowed:
the box has the same underlining type as the box from which it
inherits the value.  This is indicated by an equals sign after the
kinship specification.

\begin{verbatim}
                   'Underline' ':' UnderLineInherit /
                   'Thickness' ':' ThicknessInherit /

UnderLineInherit = Kinship '=' / 'NoUnderline' /
                   'Underlined' / 
                   'Overlined' / 'CrossedOut' .
ThicknessInherit = Kinship '=' / 'Thick' / 'Thin' .
\end{verbatim}

\subsection{Stacking order}

The {\tt Depth} rule is used to define the stacking order of terminal
boxes when multiple boxes at least partially overlap.  This rule
defines how the depth parameter, which is zero or a positive integer,
is calculated.  The depth parameter has a value for all boxes.  For
terminal boxes in the structure and for presentation boxes, the depth
value is used during display and printing: the boxes with the lowest
value overlap those with higher depths.  For non-terminal boxes, the
depth is not interpreted during display, but it is used to calculate
the depth of terminal boxes by inheritance.

Like most other rules, the depth rule is defined in the default rules
of each presentation schema (see page~\pageref{reglesdefaut}).  Thus,
there is always a depth value, even when it is not necessary because
there is no overlapping.  To avoid useless operations, a zero value
can be given to the depth parameter, which signifies that overlapping
is never a problem.

The depth rule has the same form as the visibility rule (see
section~\ref{visib}).  It can be defined by inheritance or by a
constant numeric value.  When the rule is attached to a numeric
attribute, it can take the value of that attribute.

\begin{verbatim}
                'Depth' ':' NumberInherit
\end{verbatim}

\begin{example}
For a purely textual document, in which overlapping never poses a
problem, a single default {\tt Depth} rule in the presentation schema
is sufficient:
\begin{verbatim}
DEFAULT
    Depth : 0;
    ...
\end{verbatim}

To make the text of examples appear on a light blue background, a
presentation box is defined:
\begin{verbatim}
BOXES
   BlueBG :
      BEGIN
      Content : Graphics 'R';
      Background : LightBlue3;
      FillPattern: backgroundcolor;
      Depth : 2;
      ...
      END;
\end{verbatim}
and is created by the {\tt Example} element, which has the rules:
\begin{verbatim}
RULES
   Example :
      BEGIN
      CreateFirst (BlueBG);
      Depth : 1;
      ...
      END;
\end{verbatim}
In this way, the text of an example (if it inherits its depth from its
ancestor) will be superimposed on a light blue background, and not the
reverse).
\end{example}

\subsection{Line Style}
\label{styletrait}

The {\tt LineStyle} rule determines the style of line which should be
used to draw all the elements contained in the box.  The line style
can be indicated by a name ({\tt Solid}, {\tt Dashed}, {\tt Dotted})
or it can be inherited.  Only elements of the graphic base type are
affected by this rule, but it can be attached to any box and
transmitted by inheritance to the graphic elements.

Only identical inheritance is allowed: the box takes the same line
style as the box from which it inherits.  This is indicated by an
equals sign after the kinship specification.

\begin{verbatim}
                      'LineStyle' ':' LineStyleInherit
     LineStyleInherit = Kinship '=' /
                      'Solid' / 'Dashed' / 'Dotted' .
\end{verbatim}

\begin{example}
To specify that, in Figures, the graphical parts should be
drawn in solid lines, the Figure element is given a rule using the
{\tt Solid} name:

\begin{verbatim}
Figure :
   LineStyle : Solid;
\end{verbatim}
and the elements composing figures are given an inheritance rule:
\begin{verbatim}
   LineStyle : Enclosing =;
\end{verbatim}
\end{example}

\subsection{Line thickness}

The {\tt LineWeight} rule determines the thickness of the lines  of
all graphical elements which appear in the box, no matter what their
line style.  Line thickness can be specified by a constant value or by
inheritance.  A constant value is a positive number followed by an
optional unit specification (which is absent when using relative
units).  All the distance units given on page~\pageref{unites} can be
used.  Line thickness is expressed in the same way as interline
spacing (see section~\ref{interligne}).

\begin{verbatim}
                 'LineWeight' ':' DistOrInherit
\end{verbatim}

Only elements of the graphic base type are affected by this rule, but
it can be attached to any box and transmitted by inheritance to the
graphic elements.

\begin{example}
To specify that, in Figures, the graphical parts should be drawn with
lines 0.3 pt thick, the Figure element is given this rule:
\begin{verbatim}
Figure :
   LineWeight : 0.3 pt;
\end{verbatim}
and the elements composing figures are given an inheritance rule:
\begin{verbatim}
   LineWeight : Enclosing =;
\end{verbatim}
\end{example}

\subsection{Fill pattern}
\label{remplissage}

The {\tt FillPattern} rule determines the pattern used to fill closed
graphical elements (circles, rectangles, etc.) which appear in the
box.  This pattern can be indicated by a named constant or by
inheritance.  The named constant identifies one of the patterns
available in Thot.  The names of the available patterns are:
nopattern, foregroundcolor, backgroundcolor, gray1, gray2, gray3,
gray4, gray5, gray6, gray7, horiz1, horiz2, horiz3, vert1, vert2,
vert3, left1, left2, left3, right1, right2, right3, square1, square2,
square3, lozenge, brick, tile, sea, basket.

Like the other rules peculiar to graphics, {\tt LineStyle} and {\tt
LineWeight}, only elements of the graphic base type are affected by
the {\tt FillPattern} rule, but the rule can be attached to any box
and transmitted by inheritance to the graphic elements.  As with the
other rules specific to graphics, only identical inheritance is
allowed.

The {\tt FillPattern} rule can also be used to determine whether or
not text characters, symbols and bitmaps should be colored.  For these
element types (test, symbols, and images), the only valid values are
nopattern, foregroundcolor, and backgroundcolor.  When {\tt
FillPattern} has the value {\tt backgroundcolor}, text characters,
symbols, and bitmaps are given the color specified by the {\tt
Background} rule (see section~\ref{couleurs}) which applies to these
elements.  When {\tt FillPattern} has the value {\tt foregroundcolor},
these same elements are given the color specified by the {\tt
Foreground} rule (see section~\ref{couleurs}) which applies to these
elements.  In all other case, text characters are not colored.

\begin{verbatim}
                 'FillPattern' ':' NameInherit
\end{verbatim}

\begin{example}
To specify that, in Figures, the closed graphical elements should be
filled with a pattern resembling a brick wall, the Figure element is
given this rule:

\begin{verbatim}
Figure :
   FillPattern : brick;
\end{verbatim}
and the elements composing figures are given an inheritance rule:
\begin{verbatim}
   FillPattern : Enclosing =;
\end{verbatim}
\end{example}

\subsection{Colors}
\label{couleurs}

The {\tt Foreground} and {\tt Background} rules determine the
foreground and background colors of the base elements which appear in
the box.  These colors cancan be specified with a named constant or by
inheritance.  The named constants specify one of the available colors
in Thot\footnote{The available color names can be found in the file
{\tt thot.color}.}. 

In contrast to the preceding rules, the color rules affect all base
elements the same way, no matter what their type (text, graphics,
images, symbols), but they only affect base elements.  The color rules
can nevertheless be associated with any box and can be transmitted to
the base elements by inheritance.  Like the preceding rules, only
inheritance of the same value is allowed.

\begin{verbatim}
                 'Foreground' ':' NameInherit
                 'Background' ':' NameInherit
\end{verbatim}

Note: text colors only appear for text elements whose fill pattern
does not prevent the use of color (see section~\ref{remplissage}).

\begin{example}
To specify that, in Figures, everything must be drawn in blue on a
background of yellow, the Figure element is
given these rules:

\begin{verbatim}
Figure :
   BEGIN
   Foreground : Blue;
   Background : Yellow;
   Fillpattern : backgroundcolor;
   END;
\end{verbatim}
and the elements composing figures are given inheritance rules:
\begin{verbatim}
   Foreground : Enclosing =;
   Background : Enclosing =;
   FillPattern : Enclosing =;
\end{verbatim}
\end{example}

\subsection{Presentation box content}
\label{content}

The {\tt Content} rule applies to presentation boxes.  It indicates
the content given to a box.  This content is either a variable's value
or a constant value.  In the special case of header or footer boxes
(see page~\pageref{page}), the content can also be a structured
element type.

If the content is a constant, it can be specified, as in a variable
declaration, either by the name of a constant declared in the {\tt
CONST} section or by direct specification of the type and value of the
box's content.

Similarly, if it is a variable, the name of a variable declared in
{\tt VAR} section can be given or the variable may be defined within
parentheses.  The content inside the parentheses has the same syntax
as a variable declaration (see page~\pageref{variables}).

When the content is a structured element type, the name of the element
type is given after the colon.  In this case,  the box's content is
all elements of the named type which are designated by references
which are part of the page on which the header or footer with this
{\tt Content} rule appears.  Only associated elements can appear in a
{\tt Content} rule and the structure must provide references to these
elements.  Moreover, the box whose content they are must be a header
or footer box generated by a page box of the primary view.

\begin{verbatim}
               'Content' ':' VarConst
     VarConst = ConstID / ConstType ConstValue /
                VarID / '(' FunctionSeq ')' /
                ElemID .
\end{verbatim}

A presentation box can have only one {\tt Content} rule, which means
that the content of a presentation box cannot vary from view to view.
However, such an effect can be achieved by creating several
presentation boxes, each with different content and visible in
different views.

The {\tt Content} rule also applies to elements defined as references
in the structure schema.  In this case, the content defined by the
rule  must be a constant.  It is this content which appears on the
screen or paper to represent references of the type to which the rule
applies.  A reference can have a {\tt Content} rule or a {\tt Copy}
rule (see section~\ref{regleCopy} for each view.  If neither of these
rules appears, the reference is displayed as {\tt [*]}, which is
equivalent to the rule:

\begin{verbatim}
     Content: Text '[*]';
\end{verbatim}

\begin{example}
The content of the presentation box created to make the chapter
number and section number appear before each section title can be
defined by:

\begin{verbatim}
BOXES
     SectionNumBox :
          BEGIN
          Content : NumSection;
          ...
          END;
\end{verbatim}
if the {\tt NumSection} variable has been defined in the variable
definition section of the presentation schema.  Otherwise the {\tt
Content} would be written:

\begin{verbatim}
BOXES
     SectionNumBox :
          BEGIN
          Content : (VALUE (ChapterCtr, Roman) TEXT '.'
                     VALUE (SectionCtr, Arabic));
          ...
          END;
\end{verbatim}

To specify that a page footer should contain all elements of the {\tt
Note} type are referred to in the page, the following rule is written:
\begin{verbatim}
BOXES
     NotesFooterBox :
          BEGIN
          Content : Note;
          ...
          END;
\end{verbatim}
{\tt Note} is defined as an associated element in the structure schema
and NotesFooterBox is created by a page box of the primary view.
\end{example}

\subsection{Presentation box creation}
\label{creation}

A creation rule specifies that a presentation box should be created
when an element of the type to which the rule is attached appears in
the document.

A keyword specifies the position, relative to the
creating box, at which the created box will be placed in the
structure:

\begin{description}
\item[ {\tt CreateFirst} ]specifies that the box should be created as
the first box of the next lower level, before any already existing
boxes, and only if the beginning of the creating element is visible;

\item[ {\tt CreateLast} ]specifies that the box should be created as
the last box of the next lower level, after any existing boxes, and
only if the end of the creating element is visible;

\item[ {\tt CreateBefore} ]specifies that the box should be created
before the creating box, on the same level as the creating box, and
only if the beginning of the creating element is visible;

\item[ {\tt CreateAfter} ]specifies that the box should be created
after the creating box, on the same level as the creating box, and
only if the beginning of the creating element is visible;

\item[ {\tt CreateEnclosing} ]specifies that the box should be created
at the upper level relatively to the creating box, and that it must
contain that creating box and all presentation boxes created by the
same creating box.

\end{description}

This keyword can be followed by the {\tt Repeated} keyword to indicate
that the box must be created for each part of the creating element.
These parts result from the division of the element by page breaks or
column changes.  If the {\tt Repeated} keyword is missing, the box is
only created for the first part of the creating element ({\tt
CreateFirst} and {\tt CreateBefore} rules) or for the last part ({\tt
CreateLast} and {\tt CreateAfter} rules).

The type of presentation to be created is specified at the end of the
rule between parentheses.

Creation rules cannot appear in the default presentation rules (see
page~\pageref{reglesdefaut}).  The boxes being created should have a
{\tt Content} rule which indicates their content (see
page~\pageref{content}).

Creation rules can only appear in the block of rules for the primary
view; creation is provoked by a document element for all views.
However, for each view, the presentation box is only created if the
creating element is itself a box in the view. Moreover, the visibility
parameter of the presentation box can be adjusted to control the
creation of the box on a view-by-view basis.

\begin{verbatim}
                      Creation '(' BoxID ')'
     Creation      = Create [ 'Repeated' ] .
     Create        ='CreateFirst' / 'CreateLast' /
                    'CreateBefore' / 'CreateAfter' /
                    'CreateEnclosing' .
\end{verbatim}

\begin{example}
Let us define an object type, called Table, which is composed of a
sequence of columns, all having the same fixed width, where the
columns are separated by vertical lines.  There is a line to the left
of the first column and one to the right of the last.  Each column has
a variable number of cells, placed one on top of the other and
separated by horizontal lines.  There are no horizontal lines above
the first cell or below the last cell.  The text contained in each
cell is  broken into lines and these lines are centered horizontally
in the cell. The logical structure of this object is defined by:

\begin{verbatim}
Table   = LIST OF (Column);
Column  = LIST OF (Cell = Text);
\end{verbatim}

\begin{figure}
\begin{center}
\setlength{\unitlength}{1 mm}
\begin{picture}(90,45)
\put(0,0){\line(0,1){45}}
\put(30,0){\line(0,1){45}}
\put(60,0){\line(0,1){45}}
\put(90,0){\line(0,1){45}}
\put(0,15){\line(1,0){30}}
\put(30,20){\line(1,0){30}}
\put(0,25){\line(1,0){30}}
\put(60,30){\line(1,0){30}}
\put(30,40){\line(1,0){30}}
\put(1,1){\shortstack{xxx xxxx xx x\\xx xxxxx xxxx x\\xx xxx xxxxx x\\xxx x
 xxxxx xx\\xxxx xx}}
\put(31,3){\shortstack{xxx xxxxx xx\\xx xxx x xxxxxx\\xx xxxx x xxxx\\
 xxxxx xx xxxxx\\xxxx xx xxxxx\\xxxxxx}}
\put(61,4){\shortstack{xx xxxx xxxxx\\xxx xxxxx xx\\xx xxxx xxxxxx\\xxxx x xxx
\\xxxxx xx xxx xx\\x xxx xxxx xx\\xxxx xx xxxx xx\\xxxxx xx xxxx\\xxxx xx}}
\put(2,18){\shortstack{xxxx xxx x xxx\\xxx x xxxx xx}}
\put(32,21){\shortstack{xx xx xxxx xx\\x xxx xxx xxxx\\xx xxxxx x x\\xxxxx xx
xxx\\xxxxx xx xxx\\x xxx xxxx xx\\xxxxx x}}
\put(61,31){\shortstack{x xxx xxx xxx\\xxx xxx x xxxxx\\xxxx x xx xxx\\
x xx xxxx x\\xxx}}
\put(1,28){\shortstack{xx xxxxx xxxx\\xxxx xxx x xxxx\\xxx xx x xxx\\xx xxxxx
xx x\\xxx xx xxx x\\xxx}}
\put(31,42){\shortstack{x xxx xxx xxxxx}}
\end{picture}
\end{center}
\caption{The design of a table}
\label{table}
\end{figure}

The presentation of the table should resemble the design of
Figure~\ref{table}.  It is defined by the following presentation
schema fragment:

\begin{verbatim}
BOXES
     VertLine : BEGIN
                 Width : 0.3 cm;
                 Height : Enclosing . Height;
                 VertPos : Top = Enclosing . Top;
                 HorizPos : Left = Previous . Right;
                 Content : Graphics 'v';
                 END;

     HorizLine: BEGIN
                 Width : Enclosing . Width;
                 Height : 0.3 cm;
                 VertPos : Top = Previous . Bottom;
                 HorizPos : Left = Enclosing . Left;
                 Content : Graphics 'h';
                 END;

RULES
     Column  : BEGIN
               CreateBefore (VertLine);
               IF LAST CreateAfter (VertLine);
               Width : 2.8 cm;
               Height : Enclosed . Height;
               VertPos : Top = Enclosing . Top;
               HorizPos : Left = Previous . Right;
               END;

     Cell    : BEGIN
               IF NOT FIRST CreateBefore (HorizLine);
               Width : Enclosing . Width;
               Height : Enclosed . Height;
               VertPos : Top = Previous . Bottom;
               HorizPos : Left = Enclosing . Left;
               Line;
               Adjust : VMiddle;
               END;
\end{verbatim}
It is useful to note that the horizontal position rule of the first
vertical line will not be applied, since there is no preceding box.
In this case, the box is simply placed on the left side of the
enclosing box.

\end{example}

\subsection{Page layout}
\label{page}

The page models specified in the {\tt Page} rule are defined by boxes
declared in the {\tt BOXES} section of the presentation schema.  Pages
are not described as frames which will be filled by the document's
text, but as element are inserted in the flow of the document and which
mark the page breaks.  Each of these page break elements contains
presentation boxes which represent the footer boxes of a page followed
by header boxes of the next page.  The page box itself is the simple
line which separates two pages on the screen.  Both the footer and
header boxes placed themselves with respect to this page box, with the
footer being placed above it and the header boxes being placed above
it.

The boxes created by a page box are headers and footers and can only
place themselves vertically with respect to the page box itself (which
is in fact the separation between two pages).  Besides, it is their
vertical position rule  which determines whether they are header or
footer boxes.  Header and footer boxes must have an explicit vertical
position rule (they must not use the default rule).

Footer boxes must have an absolute height or inherit the height of
their contents:
\begin{verbatim}
Height : Enclosed . Height;
\end{verbatim}

A page box must have height and width rules and these two rules must
be specified with constant values, expressed in centimeters, inches,
or typographer's points.  These two rules are interpreted in a special
way for page boxes:  they determine the width of the page and the
vertical distance between two page separators, which is the height of
the page and its header and footer together.

A page box should also have vertical and horizontal position rules and
these two rules should specify the position on the sheet of paper of
the rectangle enclosing the page's contents.  These two rules must
position the upper left corner of the enclosing rectangle in relation
to the upper left corner of the sheet of paper, considered to be the
enclosing element.  In both rules, distances must be expressed in
fixed units: centimeters ({\tt cm}), inches ({\tt in}), or
typographer's points ({\tt pt}).  Thus, rules similar to the following
should be found in the rules for a page box:

\begin{verbatim}
BOXES
   ThePage :
      BEGIN
      VertPos : Top = Enclosing . Top + 3 cm;
      HorizPos : Left = Enclosing . Left + 2.5 cm;
      Width : 16 cm;
      Height : 22.5 cm;
      END;
\end{verbatim}

When a document must be page broken, the page models to be constructed
are defined in the {\tt BOXES} section of the presentation schema by
declaring page boxes and header and footer boxes.  Also, the {\tt
Page} rule is used to specify to which parts of the document and to
which views each model should be applied.

The {\tt Page} rule has only one parameter, given between parentheses
after the {\tt Page} keyword.  This parameter is the name of the box
which must serve as the model for page construction.  When a {\tt
Page} rule is attached to an element type, each time such an element
appears in a document, a page break takes place and the page model
indicated in the rule is applied to all following pages, until
reaching the next element which has a {\tt Page} rule.

The {\tt Page} rule applies to only one view; if it appears in the
primary view's block of rules, a {\tt Page} rule applies only to that
view.  Thus, different page models
can be defined for the full document and for its table of contents,
which is another view of the same document.
Some views can be specified with pages, and other views of the same document
can be specified without pages.

\begin{verbatim}
                   'Page' '(' BoxID ')'
\end{verbatim}

\subsection{Box copies}
\label{regleCopy}

The {\tt Copy} rule can be used for an element which is defined as a
reference in the structure schema.  In this case, the rule specified,
between parenthesis, thee name of the box (declared in the {\tt BOXES}
section) which must be produced when this reference appears in the
structure of a document.  The box produced is a copy (same contents,
but possible different presentation) of the box type indicated by the
parameter between parentheses, and which is in the element designated
by the reference.  The name of a box can be replaced by type name.
Then what is copied is the contents of the element of this type which
is inside the referenced element.

Whether a box name or type name is given, it may be followed by the
name of a structure schema between parentheses.  This signifies that
the box or type is defined in the indicated structure schema and not
in the structure schema with which the rule's presentation schema is
associated.

The {\tt Copy} rule can also be applied to a presentation box.  If the
presentation box was created by a reference attribute, the rule is
applied as in the case of a reference element: the contents of the box
having the {\tt Copy} rule are based on the element designated by the
reference attribute.  For other presentation boxes, the {\tt Copy} rule takes a
type name parameter which can be followed, between parentheses, by the
name of the structure schema in which the type is defined, if it is
not defined in the same schema.  The contents of the box which has
this rule are a copy of the element of this type which is in the
element creating the presentation box, or by default, the box of this
type which precedes the presentation box.  This last facility is used,
for example, to define the running titles in headers or footers.

\begin{verbatim}
               'Copy' '(' BoxTypeToCopy ')' .
  BoxTypeToCopy = BoxID [ ExtStruct ] /
                     ElemID [ ExtStruct ] .
  ExtStruct        = '(' ElemID ')' .
\end{verbatim}

Like the creation rules, the {\tt Copy} rule cannot appear in the
default presentation rules (see~\pageref{reglesdefaut}).  Moreover,
this rule can only appear in the primary view's block of rules; the
copy rule is applied to all views.

\begin{example}
If the following definitions are in the structure schema:

\begin{verbatim}
Body = LIST OF (Chapter =
                     BEGIN
                     ChapterTitle = Text;
                     ChapterBody = SectionSeq;
                     END);
RefChapter = REFERENCE (Chapter);
\end{verbatim}
then the following presentation rules (among many other rules in the
presentation schema) can be specified:

\begin{verbatim}
COUNTERS
   ChapterCtr : RANK OF Chapter;
BOXES
   ChapterNumber :
      BEGIN
      Content : (VALUE (ChapterCtr, URoman));
      ...
      END;
RULES
   Chapter :
      BEGIN
      CreateFirst (ChapterNumber);
      ...
      END;
   RefChapter :
      BEGIN
      Copy (ChapterNumber);
      ...
      END;
\end{verbatim}
which makes the number of the chapter designated by the reference
appear in uppercase roman numerals, in place of the reference to a
chapter itself.  Alternatively, the chapter title can be made to
appear in place of the reference by writing this {\tt Copy} rule:

\begin{verbatim}
      Copy (ChapterTitle);
\end{verbatim}

To define a header box, named {\tt RunningTitle}, which contains the
title of the current chapter, the box's contents are defined in this way:
\begin{verbatim}
BOXES
   RunningTitle :
      Copy (ChapterTitle);
\end{verbatim}

\end{example}

\chapter{The T language}

\section{Document translation}

Because of its document model, Thot can produce documents in a
high-level abstract form.  This form, called the {\em canonical
form}\footnote{{\em forme pivot} in French, where {\em pivot} means
linchpin} is specific to Thot; it is well suited to the editor's
manipulations, but it does not necessarily suit other operations which
might be applied to documents.  Because of this, the Thot editor
offers the choice of saving documents in its own form (the canonical
form) or a format defined by the user.  In the latter case, the Thot
document is transformed by the translation program.  This facility can
also be used to export documents from Thot to systems using other
formalisms.

\subsection{Translation principles}

Document translation allows the export of documents to other systems
which do not accept Thot's canonical form.  Translation can be used to
export document to source-based formatters like {\TeX}, {\LaTeX},
Scribe, and {\tt troff}.  It can also be used to translate documents
into interchange formats like SGML or HTML.  To allow the widest range of
possible exports, Thot does not limit the choice of translations, but
rather allows the user to define the formalisms into which documents
can be translated.

For each document or object class, a set of translation rules can be
defined, specifying how the canonical form should be transformed into
a given formalism.  These translation rules are grouped into
{\em translation schemas}, each schema containing the rules necessary
to translate a generic logical structure (document or object
structure) into a particular formalism.  The same generic logical
structure can have several different translation schemas, each
defining translation rules for a different formalism.

Like presentation schemas, translation schemas are generic.  Thus,
they apply to an entire object or document class and permit
translation of all documents or objects of that class.

\subsection{Translation procedure}

The translator works on the specific logical structure of the document
being translated.  It traverses the primary tree of this logical
structure in pre-order and, at each node encountered, it applies the
corresponding translation rules defined in the translation schema.
Translation can be associated:
\begin{itemize}
\item with element types defined in the structure schema,

\item with global or local attributes defined in the structure schema,

\item with specific presentation rules,

\item with the content of the leaves of  the structure (characters,
symbols and graphical elements)

\end{itemize}

Thus, for each node, the translator applies all rules associated with
the element type, all rules associated with each attribute (local or
global) carried by the element, and if the element is a leaf of the
tree, it also applies Translation rules for characters, symbols, or
graphical elements, depending on the type of the leaf.

Rules associated with the content of leaves are different from all
other rules: they specify only how to translate character strings,
symbols, and graphical elements.  All other rules, whether associated
with element types, with specific presentation rules or with
attributes, are treated similarly.  These rules primarily allow:
\begin{itemize}
\item generation of a text constant or variable before or after
the contents of an element,

\item modification of the order in which elements appear after
translation,

\item removal of an element in the translated document,

\item and writing messages on the user's terminal during translation.
\end{itemize}

\section{Translation definition language}

Translation schemas are written in a custom language, called T, which
is described in the rest of this chapter.  The grammar of T is
specified using the same meta-language as was used for the S and P
languages (see page~\pageref{metalang}) and the translation schemas
are written using the same conventions as the structure and
presentation schemas.  In particular, the keywords of the T language
(the stings between apostrophes in the following syntax rules) can be
written in any combination of upper-case and lower-case letters, but
identifiers created by the programmer must always be written in the
same way.

\subsection{Organization of a translation schema}

A translation schema is begun by the {\tt TRANSLATION} keyword and
is terminated by the {\tt END} keyword.  The {\tt TRANSLATION} keyword
is followed by the name of the generic structure for which a
translation is being defined and a semicolon.  This name must be
identical to the name which appears after the {\tt STRUCTURE} keyword
in the corresponding structure schema.

After this declaration of the structure, the following material
appears in order:
\begin{itemize}
\item the length of lines produced by the translation,

\item the character delimiting the end of the line,

\item the character string which the translator will insert if it must
line-break the translated text,

\item declarations of
\begin{itemize}
  \item buffers,
  \item counters,
  \item constants,
  \item variables,
\end{itemize}

\item translation rules associated with element types,

\item translation rules associated with attributes,

\item translation rules associated with specific presentation rules,

\item translation rules associated with characters strings, symbols and graphical
elements.
\end{itemize}

Each of these sections is introduced by a keyword followed by a
sequence of declarations.  All of these sections are optional, expect
for the translation rules associated with element types.  Many {\tt TEXTTRANSLATE}
sections can appear, each defining the rules for translating character
strings of a particular alphabet.

\begin{verbatim}
     TransSchema ='TRANSLATION' ElemID ';'
                [ 'LINELENGTH' LineLength ';' ]
                [ 'LINEEND' CHARACTER ';' ]
                [ 'LINEENDINSERT' STRING ';' ]
                [ 'BUFFERS' BufferSeq ]
                [ 'COUNTERS' CounterSeq ]
                [ 'CONST' ConstSeq ]
                [ 'VAR' VariableSeq ]
                  'RULES' ElemSeq
                [ 'ATTRIBUTES' AttrSeq ]
                [ 'PRESENTATION' PresSeq ]
                < 'TEXTTRANSLATE' TextTransSeq >
                [ 'SYMBTRANSLATE' TransSeq ]
                [ 'GRAPHTRANSLATE' TransSeq ]
                  'END' .
\end{verbatim}

\subsection{Line length}
\label{linelength}

If a {\tt LINELENGTH} instruction is present after the structure
declaration, the translator divides the text it produces into lines,
each line having a length less than or equal to the integer which
follows the {\tt LINELENGTH} keyword.  This maximum line length is
expressed as a number of characters.  The end of the line is marked by
the character defined by the {\tt LINEEND} instruction.  When the
translator breaks the lines on a space character in generated text,
this space will be replaced by the character string defined by the
{\tt LINEENDINSERT} instruction.

If the {\tt LINEEND} instruction is not defined then the linefeed
character (octal code 12) is used as the default line end character.
If the {\tt LINEENDINSERT} instruction is not defined, the linefeed
character is inserted at the end of the produced lines.  If there is
no {\tt LINELENGTH} instruction, the translated text is not divided into
lines.  Otherwise, if the translation rules generate line end marks,
these marks remain in the translated text, but the length of the lines
is not controlled by the translator.

\begin{verbatim}
     LineLength = NUMBER .
\end{verbatim}

\begin{example}
To limit the lines produced by the translator to a length of 80
characters, the following rule is written at the beginning of the
translation schema.
\begin{verbatim}
LineLength 80;
\end{verbatim}
\end{example}

\subsection{Buffers}

A buffer is a  unit of memory managed by the translator, which can
either contain text read from the terminal during the translation (see
the {\tt Read} rule, p.~\pageref{readrule}), or the name of the last image (bit-map)
encountered by the translator in its traversal of the document.
Remember the images are stored in files that are separate for the
document files and that the canonical form contains only the names of
the files in which the images are found.

Thus, there are two types of buffers:  buffers for reading from the
terminal (filled by the {\tt Read} rule) and the buffer of image
names (containing the name of the last image encountered).  A
translation schema can use either type, one or several read buffers
and one (and only one) image name buffer.

If any buffers are used, the {\tt BUFFERS} keyword must be present,
followed by declarations of every buffer used in the translation
schema.  Each buffer declaration  is composed only of the name of the
buffer, chosen freely by the programmer.  The image name buffer is
identified by the {\tt Picture} keyword, between parentheses,
following the buffer name.  The {\tt Picture} keyword may only appear
once.  Each buffer declaration is terminated by a semicolon.

\begin{verbatim}
     BufferSeq = Buffer < Buffer > .
     Buffer    = BufferID [ '(' 'Picture' ')' ] ';' .
     BufferID  = NAME .
\end{verbatim}

\begin{example}
The following buffer declarations create an image name buffer named
{\tt ImageName} and a read buffer named {\tt DestName}:
\begin{verbatim}
BUFFERS
     ImageName (Picture); DestName;
\end{verbatim}
\label{nomdest}
\end{example}

\subsection{Counters}
\label{counters}

Certain translation rules generate text that varies according to the
context of the element to which the rules apply.  Variable text is
defined either in the {\tt VAR} section of the translation schema (see
section~\ref{sectvar}) or in the rule itself (see the {\tt Create} and
{\tt Write} rules).  Both types of definition rely on counters for the
calculation of variable material.

There are two types of counter: counters whose value is explicitely
computed by applying {\tt Set} and {\tt Add} rules (see
section~\ref{setandadd}), and counters whose value is computed by a
function associated with the counter.  Those functions allow the same
calculations as can be used in presentation schemas.
As in a presentation schema, counters must
be defined in the {\tt COUNTERS} section of the translation schema
before they are used.

When counters are used in a translation schema, the {\tt COUNTERS}
keyword is followed by the declarations of every counter used.  Each
declaration is composed of the counter's name possibly followed by a
colon and the counting function to be used for the counter.  The
declaration is terminated by a semi-colon. If the counter is explicitely
computed by {\tt Set} and {\tt Add} rules, no counting function is
indicated. If a counting function is indicated, {\tt Set} and {\tt Add}
rules cannot be applied to that counter.

The counting function indicates how the counter's value will be
computed.  Three functions are available: {\tt Rank}, {\tt Rlevel},
and {\tt Set}.

\begin{itemize}
\item
{\tt Rank of ElemID} indicates that the counter's value is the rank of
the element of type {\tt ElemID} which encloses the element for which
the counter is being evaluated.  For the purposes of this function, an
element of type {\tt ElemID} is considered to enclose itself.  This
function is primarily used  when the element of type {\tt ElemID} is
part of an aggregate or list, in which case the counter's value is the
element's rank in its list or aggregate.  Note that, unlike the {\tt
Rank} function for presentation schemas, the {\tt Page} keyword cannot
be used in place of the {\tt ElemID}.

The type name {\tt ElemID} can be followed by an integer.  That number
represents the relative level, among the ancestors of the concerned element,
of the element whose rank is asked.  If that relative level $n$ is
unsigned, the $n^th$ element of type {\tt ElemID} encountered when
travelling the logical structure from the root to the concerned element
is taken into account.  If the relative level is negative, the logical
structure is travelled in the other direction, from the concerned element
to the root.

\item
{\tt Rlevel of ElemID} indicates that the counter's values is the
relative level in the tree of the element for which the counter is
being evaluated.  The counter counts the number of elements of type
{\tt ElemID} which are found on the path between the root of the
document's logical structure tree and the element (inclusive).

\item
{\tt Set n on Type1 Add m on Type2} indicates that the counter's value
is calculated as follows:  in traversing the document from the
beginning to the element for which the counter is being evaluated, the
counter is set to the value {\tt n} each time a {\tt Type1} element is
encountered and is incremented by the amount {\tt m} each time a {\tt
Type2} element is encountered.  The initial value {\tt n} and the
increment {\tt m} are integers.
\end{itemize}

As in a presentation schema, the {\tt Rank} and {\tt Set} functions
can be modified by a numeric attribute which changes their initial
value.  This is indicated by the {\tt Init} keyword followed by the
numeric attribute's name.  The {\tt Set} function takes the value of
the attribute instead of the {\tt InitValue} ({\tt n}).  For the {\tt Rank}
function, the value of the attribute is considered to be the rank of
the first element of the list (rather than the normal value of 1).
Subsequent items in the list have their ranks shifted accordingly.  In
both cases, the attribute must be numeric and must be a local
attribute of the root of the document itself.

\begin{verbatim}
     CounterSeq  = Counter < Counter > .
     Counter     = CounterID [ ':' CounterFunc ] ';' .
     CounterID   = NAME .
     CounterFunc = 'Rank' 'of' ElemID [ SLevelAsc ]
                   [ 'Init' AttrID ] /
                   'Rlevel' 'of' ElemID /
                   'Set' InitValue 'On' ElemID
                         'Add' Increment 'On' ElemID
                         [ 'Init' AttrID ] .
     SLevelAsc   = [ '-' ] LevelAsc .
     LevelAsc    =  NUMBER .
     InitValue   = NUMBER .
     Increment   = NUMBER .
     ElemID      = NAME .
     AttrID      = NAME .
\end{verbatim}

\begin{example}
If the body of a chapter is defined in the structure schema by:
\begin{verbatim}
Chapter_Body = LIST OF
         (Section = BEGIN
                    Section_Title = Text;
                    Section_Body  = BEGIN
                                    Paragraphs;
                                    Section;
                                    END;
                    END
         );
\end{verbatim}
(sections are defined recursively), a counter can be defined giving
the number of a section within its level in the hierarchy:
\begin{verbatim}
COUNTERS
   SectionNumber : Rank of Section;
\end{verbatim}
\label{numsect}
A counter holding the hierarchic level of a section:
\begin{verbatim}
   SectionLevel : Rlevel of Section;
\end{verbatim}
A counter which sequentially numbers all the document's sections,
whatever their hierarchic level:
\begin{verbatim}
   UniqueSectNum : Set 0 on Document Add 1 on Section;
\end{verbatim}
\label{numunique}
\end{example}

\subsection{Constants}

A common feature of translation rules is the generation of constant
text.  This text can be defined in the rule that generates it (see for
example the {\tt Create} and {\tt Write} rules, pp. \pageref{create}
and \pageref{writerule}); but it can also be
defined once in the constant declaration section and used many times
in different rules.  The latter option is preferable when the same
text is used in several rules or several variables (see
section~\ref{sectvar}).

The {\tt CONST} keyword begins the constant declaration section of the
translation schema.  It must be omitted if no constants are declared.
Each constant declaration is composed of the constant name, an equals
sign, and the constant's value, which is a character string between
apostrophes.  A constant declaration is terminated by a semicolon.

\begin{verbatim}
     ConstSeq   = Const < Const > .
     Const      = ConstID '=' ConstValue ';' .
     ConstID    = NAME .
     ConstValue = STRING .
\end{verbatim}

\label{txtniveau}
\begin{example}
The following rule assigns the name {\tt TxtLevel} to the character
string ``Level'':
\begin{verbatim}
CONST
     TxtLevel = 'Level';
\end{verbatim}
\end{example}

\subsection{Variables}
\label{sectvar}

Variables allow to define file names or variable text which is
generated by
the {\tt Create} and {\tt Write} rules.  Variables can be defined
either in the {\tt VAR} section of the translation schema or directly
in the rules which use them.  Variables that define fine names must
be declared in the {\tt VAR} section, and when the same variable is
used several times in the translation schema, it makes sense to define
it globally in the {\tt VAR} section.  This section is only present
if at least one variable is defined globally.

After the {\tt VAR} keyword, each global variable is defined by its
name, a colon separator and a sequence of functions (at least one
function).  Each variable definition is terminated by a semicolon.
Functions determine the different parts which together give the value
of the variable.  The value is obtained by concatenating the strings
produced by each of the functions.  Seven types of functions are
available.  Each variable definition may use any number of functions
of each type.

\begin{itemize}
\item
The function {\tt Value(Counter)} returns a string representing the
value taken by the counter when it is evaluated for the element in
whose rule the variable is used.  The counter must have been declared
in the {\tt COUNTERS} section of the translation schema.  When the
counter is expressed in arabic numerals, the counter
name can be followed by a colon and an integer indicating a minimum
length (number of characters) for the string; if the counter's value
is normally expressed with fewer characters than the required minimum,
zeroes are added to the front of the string to achieve the minimum
length.

By default, the counter value is written in arabic digits.  If another
representation of that value is needed, the counter name must be followed
by a comma and one of the following keywords:
  \begin{itemize}
  \item
  {\tt Arabic}: arabic numerals (default value),
  \item
  {\tt LRoman}: lower-case roman numerals,
  \item
  {\tt URoman}: upper-case roman numerals,
  \item
  {\tt Uppercase}: upper-case letter,
  \item
  {\tt Lowercase}: lower-case letter.
  \end{itemize}

\item
The function {\tt FileDir}, without parameter, returns a string
representing the name of the directory of the output file that has been
given as a parameter to the translation program. The string includes
a character '/' at the end.

\item
The function {\tt FileName}, without parameter, returns a string
representing the name of the output file that has been given as a
parameter to the translation program. The file extension (the character
string that terminate the file name, after a dot) is not part of
that string.

\item
The function {\tt Extension}, without parameter, returns a string
representing the extension of the file name. That string is empty
if the file name that has been given as a parameter to the translation
program has no extension. If there is an extension, its first
character is a dot.

\item
The function {\tt DocumentName}, without parameter, returns a string
representing the name of the document being translated.

\item
The function {\tt DocumentDir}, without parameter, returns a string
representing the directory containing the document being translated.

\item
The function formed by the name of a constant returns that
constant's value.

\item
The function formed by a character string between apostrophes returns
that string.

\item
The function formed by the name of a buffer returns the contents of
that buffer.  If the named buffer is the image buffer, then the name
of the last image encountered is returned.  Otherwise, the buffer is a read buffer
and the value returned is text previously read from the terminal.  If
the buffer is empty (no image has been encountered or the {\tt Read}
rule has not been executed for the buffer), then the empty string is
returned.

\item
The function formed by an attribute name takes the value of the
indicated attribute for the element to which the variable applies.  If
the element does not have that attribute, then the element's ancestor
are searched toward the root of the tree.  If one of the ancestors
does have the attribute then its value is used.  If no ancestors have
the attribute, then the value of the function is the empty string.
\end{itemize}

\begin{verbatim}
     VariableSeq = Variable < Variable > .
     Variable    = VarID ':' Function < Function > ';' .
     VarID       = NAME .
     Function    ='Value' '(' CounterID [ ':' Length ]
                            [ ',' CounterStyle ] ')' /
                  'FileDir' / 'FileName' / 'Extension' /
                  'DocumentName' / 'DocumentDir' /
                   ConstID / CharString / 
                   BufferID / AttrID .
     Length      = NUMBER .
     CounterStyle= 'Arabic' / 'LRoman' / 'URoman' /
                   'Uppercase' / 'Lowercase' .
     CharString  = STRING .
\end{verbatim}

\begin{example}
To create, at the beginning of each section of the translated
document, text composed of the string ``Section'' followed by the
section number, the following variable definition might be used:
\begin{verbatim}
VAR
     SectionVar : 'Section' Value(SectionNumber);
\end{verbatim}
\label{varsect}
(see the definition of {\tt SectionNumber} on page~\pageref{numsect})

The following variable definition can be used to create, at the
beginning of each section, the text ``Level'' followed by the
hierarchical level of the section.  It used the constant defined
above.
\begin{verbatim}
     LevelVar : TxtLevel Value(SectionLevel);
\end{verbatim}
(see the definitions of {\tt SectionLevel} on page~\pageref{numsect}
and of {\tt TxtLevel} on page~\pageref{txtniveau}).

To generate the translation of each section in a different file,
(see rule {\tt ChangeMainFile}, p.~\pageref{changemainfile}), the
name of these files might be defined by the following variable:
\begin{verbatim}
     VarOutpuFile : FileName Value(SectionNumber)
                    Extension;
\end{verbatim}
\label{varoutputfile}
If {\tt output.txt} is the name of the output file specified when
starting the translation program, translated sections are written
in files {\tt output1.txt}, {\tt output2.txt}, etc.
\end{example}

\subsection{Translating structure elements}

The {\tt RULES} keyword introduces the translation rules which will be
applied to the various structured element types.  Translation rules
can be specified for each element type defined in the structure
schema, including the base types defined implicitly, whose names are
{\tt TEXT\_UNIT}, {\tt PICTURE\_UNIT}, {\tt SYMBOL\_UNIT}, {\tt
GRAPHIC\_UNIT} and {\tt PAGE\_UNIT} in the English version of Thot or
{\tt TEXTE}, {\tt IMAGE}, {\tt SYMBOLE}, {\tt GRAPHIQUE} and {\tt
PAGE} in the French version.  But it is not necessary to specify rules
for every defined type.

If there are no translation rules for an element type, the elements
that it contains (and which may have rules themselves) will still be
translated, but the translator will produce nothing for the element
itself.  To make the translator completely ignore the content of an
element the {\tt Remove} rule must be used (see
page~\pageref{remove}).

The translation rules for an element type defined in the structure
schema are written using the name of the type followed by a colon and
the list of applicable rules.  When the element  type is a mark pair
(see section~\ref{paires}), but only in this case, the type name must
be preceded by the {\tt First} or {\tt Second} keyword.  This keyword
indicates whether the rules that follow apply to the first or second
mark of the pair.

The list of rules can take several forms.  It may be a simple
non-conditional rule.  It can also be formed by a condition followed
by one or more simple rules.  Or it can be a block of rules beginning
with the {\tt BEGIN} keyword and ending with the {\tt END} keyword and
a semicolon.  This block of rules can contain one or more simple rules
and/or one or more conditions, each followed by one or more simple
rules.

\begin{verbatim}
     ElemSeq        = TransType < TransType > .
     TransType      = [ FirstSec ] ElemID ':' RuleSeq .
     FirstSec       = 'First' / 'Second' .
     RuleSeq        = Rule / 'BEGIN' < Rule > 'END' ';' .
     Rule           = SimpleRule / ConditionBlock .
     ConditionBlock = 'IF' ConditionSeq SimpleRuleSeq .
     SimpleRuleSeq  = 'BEGIN' < SimpleRule > 'END' ';' / 
                      SimpleRule .
\end{verbatim}

\subsection{Conditional rules}
\label{tradcond}

In a translation schema, the translation rules are either associated
with element types or with attribute values or with a specific
presentation.  They are applied by the translator each time an element
of the corresponding type is encountered in the translated document or
each time the attribute value is carried by an element or also, each
time the specific translation is attached to an element.  This
systematic application of the rules can be relaxed: it is possible to
add a condition to one or more rules, so that these rules are only
applied when the condition is true.

A condition begins with the keyword {\tt IF}, followed by a sequence
of elementary conditions.  Elementary conditions are separated from
each other by the {\tt AND} keyword.  If there is only one elementary
condition, this keyword is absent.  The rules are only applied if all
the elementary conditions are true.  The elementary condition can be
negative; it is then preceded by the {\tt NOT} keyword.

When the translation rule(s) controlled by the condition apply to a
reference element or a reference attribute, an elementary condition
can also apply to element referred by this reference.  The {\tt Target}
keyword is used for that purpose.  It must appear before the keyword
defining the condition type.

Depending on their type, some conditions may apply either to the element
with which they are associated, or to one of its ancestor.  In the case
od an ancestor, the key word {\tt Ancestor} must be used, followed by
\begin{itemize}
\item either an integer which represents the number of levels in the
   tree between the element and the ancestor of interest,
\item or the type name of the ancestor of interest.  If that type is
   defined in a separate structure schema, the name of that schema must
   follow between parentheses.
\end{itemize}
There is a special case for the parent element, which can be simply
written {\tt Parent} instead of {\tt Ancestor 1}.

Only conditions {\tt First}, {tt Last}, {tt Referred}, {tt Within},
{tt Attributes}, {tt Presentation}, {tt Comment} and those concerning
an attribute or a specific presentation can apply to an ancestor.
Conditions {\tt Defined}, {\tt FirstRef},
{\tt LastRef}, {\tt ExternalRef}, {\tt Alphabet}, {\tt FirstAttr},
{\tt LastAttr}, {\tt ComputedPage}, {\tt StartPage}, {\tt UserPage},
{\tt ReminderPage}, {\tt Empty} cannot be preceded by keywords
{\tt Parent} or {\tt Ancestor}.

In condition {\tt Referred} and in the condition that applies to
a named attribute, a symbol {\tt *} can indicate that the condition
is related only to the element itself. If this symbol is not present,
not only the element is considered, but also its ancestor, at any level.

The form of an elementary condition varies according to the type of
condition.

\subsubsection{Conditions based on the logical position of the element}

The condition can be on the position of the element in the document's
logical structure tree.  It is possible to test whether the element is
the first ({\tt First}) or last ({\tt Last}) among its siblings or if
it is not the first ({\tt NOT First}) or not the last ({\tt NOT
Last}).

It is also possible to test if the element is contained in an
element of a given type ({\tt Within}) or if it is not ({\tt NOT Within}).
If that element type is defined in a structure schema which is
not the one which corresponds to the translation schema, the type name
of this element must be followed, between parentheses, by the name of
the structure schema which defines it.

If the keyword
{\tt Within} is preceded by {\tt Immediately}, the condition is
satisfied only if the {\em parent} element has the type indicated.
If the word {\tt Immediately} is missing, the condition is satisfied
if any {\em ancestor} has the type indicated.

An integer $n$ can appear between the keyword {\tt Within} and the
type.  It specifies the number of ancestors of the indicated type that must
be present for the condition to be satisfied.  If the keyword
{\tt Immediately} is also present, the $n$ immediate ancestors of the
element must have the indicated type.  The integer $n$ must be positive
or zero.  It can be preceded by {\tt <} or {\tt >} to indicate a maximum
or minimum number of ancestors.  If these symbols are missing, the
condition is satisfied only if it exists exactly $n$ ancestors.  When
this number is missing, it is equivalent to > 0.

\subsubsection{Conditions on references}

References may be taken into account in conditions, which can be based
on the fact that the element, or one of its ancestors (unless symbol
{\\tt *} is present), is designated
by a at least one reference ({\tt Referred}) or by none ({\tt NOT
Referred}).  If the element or attribute to which the condition is
attached is a reference, the condition can be based on the fact that
it acts as the first reference to the designated element ({\tt FirstRef}),
or as the last ({\tt LastRef}), or as a reference to an element located in
another document ({\tt ExternalRef}).  Like all conditions, conditions
on references can be inverted by the {\tt NOT} keyword.

\subsubsection{Conditions on the parameters}

Elements which are parameters (see page~\pageref{param}) can be given
a particular condition which is based on whether or not the parameter
is given a value in the document ({\tt Defined} or {\tt NOT Defined},
respectively). 

\subsubsection{Conditions on the alphabets}

The character string base type (and only this type) can use the
condition {\tt Alphabet = a} which indicates that the translation
rule(s) should only apply if the alphabet of the character string is
the one whose name appears after the equals sign (or is not, if there
is a preceding {\tt NOT} keyword).  This condition cannot be applied
to translation rules of an attribute.

In the current implementation of Thot, the available alphabets are the
{\tt Latin} alphabet and the {\tt Greek} alphabet.

\subsubsection{Conditions on page breaks}

The page break base type (and only this type) can use the following
conditions:\\
{\tt ComputedPage}, {\tt StartPage}, {\tt UserPage}, and
{\tt ReminderPage}.  The {\tt ComputedPage} condition indicates that
the translation rule(s) should apply if the page break was created
automatically by Thot;  the {\tt StartPage} condition is true if the
page break is generated before the element by the {\tt Page} rule of
the P language; the {\tt UserPage} condition applies if the page break
was inserted by the user; and the {\tt ReminderPage} is applied if the
page break is a reminder of page breaking.

\subsubsection{Conditions on the element's content}

The condition can be based on whether or not the element is empty.  An
element which has no children or whose leaves are all empty is
considered to be empty itself.  This condition is expressed by the
{\tt Empty} keyword, optionally preceded by the {\tt NOT} keyword.

\subsubsection{Conditions on the presence of comments}

The condition can be based on the presence or absence of comments
associated with the translated element.  This condition is expressed
by the keyword {\tt Comment}, optionally preceded by the keyword {\tt
NOT}.

\subsubsection{Conditions on the presence of specific presentation rules}

The condition can be based on the presence or absence of specific
presentation rules associated with the translated element, whatever
the rules, their value or their number.  This condition is expressed
by the keyword {\tt Presentation}, optionally preceded by the {\tt
NOT} keyword.

\subsubsection{Conditions on the presence of logical attributes}

In the same way, the condition can be based on the presence or absence
of attributes associated with the translated elements, no matter what
the attributes or their values.  The {\tt Attributes} keyword
expresses this condition.

\subsubsection{Conditions on logical attributes}

If the condition appears in the translation rules of an attribute, the
{\tt FirstAttr} and {\tt LastAttr} keywords can be used to indicate
that the rules must only be applied if this attribute is the first
attribute for the translated element or if it is the last
(respectively).  These conditions can also be inverted by the {\tt
NOT} keyword.

Another type of condition can only be applied to the translation rules
when the element being processed (or one of its ancestors if symbol
{\tt *} is missing) has a certain
attribute, perhaps with a certain value or, in contrast, when the
element does not have this attribute with this value.  The condition
is specified by writing the name of the attribute after the keyword
{\tt IF} or {\tt AND}.  The {\tt NOT} keyword can be used to invert
the condition.  If the translation rules must be applied to any
element which has this attribute (or does not have it, if the
condition is inverted) no matter what the attribute's value, the
condition is complete.  If, in contrast, the condition applies to one
or more values of the attribute, these are indicated after the name of
the attribute, except for reference attributes which do not have
values.

\label{relattr}
The representation of the values of an attribute in a condition depends
on the attribute's type.  For attributes with enumerated or textual
types, the value (a name or character string between apostrophes,
respectively) is simply preceded by an equals sign.  For numeric
attributes, the condition can be based on a single value or on a range
of values.  In the case of a unique value, this value (an integer) is
simply preceded by an equals sign.  Conditions based on ranges of
values have several forms:
\begin{itemize}
\item all values less than a given value (the value is preceded by a
``less than'' sign).

\item all values greater than a given value (the value is preceded by a
``greater than'' sign).

\item all values falling in an interval, bounds included.  The range
of values is then specified {\tt IN [}Minimum {\tt ..} Maximum{\tt]},
where Minimum and Maximum are integers.

\end{itemize}
All numeric values may be negative.  The integer is simply preceded by
a minus sign.

Both local and global attributes can be used in conditions.

\subsubsection{Conditions on specific presentation rules}

It is possible to apply translation rules only when the element being
processed has or does not have a specific presentation rule, possibly
with a certain value.  The condition is specified by writing the name
of the presentation rule after the keyword {\tt IF} or {\tt AND}.  The
{\tt NOT} keyword can be used to invert the condition.  If the
translation rules must be applied to any element which has this
presentation rule (or does not have it, if the condition is inverted) no
matter what the rule's value, the condition is complete.  If, in
contrast, the condition applies to one or more values of the
rule, these are indicated after the name of the attribute.

The representation of presentation rule values in a condition is
similar to that for attribute values.  The representation of these
values depend on the type of the presentation rule. There are three
categories of presentation rules:
\begin{itemize}
\item those taking numeric values ({\tt Size, Indent, LineSpacing,
LineWeight}),

\item those with values taken from a predefined list ({\tt Adjust,
Justify, Hyphenate, Style, Font, UnderLine, Thickness, LineStyle}),

\item those whose value is a name ({\tt FillPattern, Background,
Foreground}).

\end{itemize}

For presentation rules which take numeric values, the condition can
take a unique value or a range of values.  In the case of a unique
value, this value (an integer) is simply preceded by an equals sign.
Conditions based on ranges of values have several forms:
\begin{itemize}
\item all values less than a given value (the value is preceded by a
``less than'' sign).

\item all values greater than a given value (the value is preceded by a
``greater than'' sign).

\item all values falling in an interval, bounds included.  The range
of values is then specified {\tt IN [}Minimum {\tt ..} Maximum{\tt]},
where Minimum and Maximum are integers.

\end{itemize}
Values for the {\tt Indent} rule may be negative.  The integer is then
simply preceded by a minus sign and represents how far the first line
starts to the left of the other lines.

For presentation rules whose values are taken from predefined lists,
the value which satisfies the condition is indicated by an equals sign
followed by the name of the value.

For presentation rule whose values are names, the value which
satisfies the condition is indicated by the equals sign followed by
the value's name.\footnote{The names of fill patterns (the {\tt
FillPattern} rule) and of colors (the {\tt Foreground} and {\tt
Background} rules) known to Thot are the same as in the P language.}

The syntax of conditions based on the specific presentation is the
same as the syntax used to express the translation of specific
presentation rules (see page~\pageref{prestrans}).

When a condition has only one rule, the condition is simply followed
by that rule.  If it has several rules, they are placed after the
condition between the keywords {\tt BEGIN} and {\tt END}.

\begin{verbatim}
   ConditionSeq = Condition [ 'AND' Condition ] .
   Condition    = [ 'NOT' ] [ 'Target' ] Cond .
   Cond         = CondElem / CondAscend .
   CondElem     ='FirstRef' / 'LastRef' /
                 'ExternalRef' /
                 'Defined' /
                 'Alphabet' '=' Alphabet /
                 'ComputedPage' / 'StartPage' / 
                 'UserPage' / 'ReminderPage' /
                 'Empty' /
                 'FirstAttr' / 'LastAttr' .
   CondAscend   = [ Ascend ] CondOnAscend .
   Ascend       = '*' / 'Parent' / 'Ancestor' LevelOrType .
   LevelOrType  = CondRelLevel / ElemID [ ExtStruct ] .
   CondRelLevel = NUMBER .
   CondOnAscend ='First' / 'Last' /
                 'Referred' / 
                  [ 'Immediately' ] 'Within' [ NumParent ]
                                    ElemID [ ExtStruct ] /
                 'Attributes' /
                  AttrID [ RelatAttr ] /
                 'Presentation' /
                  PresRule /
                 'Comment' .		  
   NumParent    = [ GreaterLess ] NParent .
   GreaterLess  = '>' / '<' .
   NParent      = NUMBER.
   ExtStruct    = '(' ElemID ')' .
   Alphabet     = NAME .
   RelatAttr    ='=' Value /
                 '>' [ '-' ] Minimum /
                 '<' [ '-' ] Maximum /
                 'IN' '[' [ '-' ] MinInterval '..'
                          [ '-' ] MaxInterval ']' .
   Value        = [ '-' ] IntegerVal / TextVal / AttrValue .
   Minimum      = NUMBER .
   Maximum      = NUMBER .
   MinInterval  = NUMBER .
   MaxInterval  = NUMBER .
   IntegerVal   = NUMBER .
   TextVal      = STRING .
   AttrValue    = NAME .
\end{verbatim}

\begin{example}
Suppose that after each element of type Section\_Title it is useful to produce the text {\tt
$\backslash$label\{SectX\} } where {\tt X} represents the section
number, but only if the section is designated by one or more
references in the document.  The following conditional rule produces
this effect:

\begin{verbatim}
RULES
  Section_Title :
    IF Referred
      Create ('\label{Sect' Value(UniqueSectNum)
              '}\12') After;
\end{verbatim}
(see page~\pageref{numunique} for the declaration of the {\tt
UniqueSectNum} counter).  The string {\tt $\backslash$12} represents a
line break.
\end{example}

\begin{example}
Suppose that for elements of the Elmnt type it would be useful to
produce a character indicating the value of the numeric attribute
Level associated with the element: an  ``A'' for all values of Level
less than 3, a ``B'' for values between 3 and 10 and a ``C'' for
values greater than 10.  This can be achieved by writing the following
rules for the Elmnt type:
\begin{verbatim}
RULES
  Elmnt :
    BEGIN
    IF Level < 3
      Create 'A';
    IF Level IN [3..10]
      Create 'B';
    IF Level > 10
      Create 'C';
    END;
\end{verbatim}
\end{example}

\subsection{Translation rules}

Thirteen types of translation rules can be associated with element types
and attribute values.  They are the {\tt Create}, {\tt Write}, {\tt
Read}, {\tt Include}, {\tt Get}, {\tt Copy}, {\tt Use}, {\tt Remove},
{\tt NoTranslation}, {\tt NoLineBreak}, {\tt ChangeMainFile}, {\tt Set},
{\tt Add} rules.
Each rule has its own syntax, although they are all based on very
similar models.

\begin{verbatim}
     SimpleRule = 'Create' [ 'IN' VarID ] Object
                        [ Position ] ';' /
                  'Write' Object [ Position ] ';' /
                  'Read' BufferID [ Position ] ';' /
                  'Include' File [ Position ] ';' /
                  'Get' [ RelPosition ] ElemID 
                        [ ExtStruct ] 
                        [ Position ] ';' /
                  'Copy' [ RelPosition ] ElemID 
                        [ ExtStruct ] 
                        [ Position ] ';' /
                  'Use' TrSchema [ 'For' ElemID ] ';' /
                  'Remove' ';' /
                  'NoTranslation' ';' /
		  'NoLineBreak' ';' /
                  'ChangeMainFile' VarID [ Position ] ';' /
                  'Set' CounterID InitValue
                        [ Position ] ';' /
                  'Add' CounterID Increment
                        [ Position ] ';' .
\end{verbatim}

\subsection{The {\tt Create} rule}
\label{create}

The most frequently used rule is undoubtedly the {\tt Create} rule,
which generates fixed or variable text (called an {\em object}) in the
output file.  The generated text can be made to appear either
before or after the content of the element to which the rule applies.
The rule begins with the {\tt Create} keyword, followed by a specifier
for the object and a keyword ({\tt Before} or {\tt After}) indicating
the position of the generated text (before or after the element's
content (see section~\ref{order})).  If the position is not indicated,
the object will be generated before the element's content.  This rule,
like all translation rules, is terminated by a semicolon.

The {\tt Create} keyword can be followed by the {\tt IN} keyword and
by the name of a variable.  This means that the text generated by the
rule must not be written in the main output file, but in the file whose
name is specified by the variable.

This allows the translation program to generate text in different files
during the same run. These files do not need to be explicetely declared or
opened. They do not need to be closed either. As soon as the translation
program executes a {\tt Create} rule for a file that is not yet open,
it opens the file. These files are closed when the translation is finished.

\begin{verbatim}
               'Create' [ 'IN' VarID ] Object
                        [ Position ] ';'
     Object   = ConstID / CharString /
                BufferID /
                VarID /
               '(' Function < Function > ')' /
                AttrID /
               'Value' /
               'Content' /
               'Comment' / 
               'Attributes' /
               'Presentation' /
               'RefId' /
               'PairId' /
	       'FileDir' /
	       'FileName' /
	       'Extension' /
               'DocumentName' /
               'DocumentDir' /
                [ 'Referred' ] ReferredObject .
     Position ='After' / 'Before' .

     ReferredObject = VarID /
                ElemID [ ExtStruct ] /
               'RefId' /
               'DocumentName' /
               'DocumentDir' .
\end{verbatim}

The object to be generated can be:
\begin{itemize}
\item a constant string, specified by its name if it is declared in
the schema's {\tt CONST} section, or given directly as a value between
apostrophes;

\item the contents of a buffer, designated by the name of the buffer;

\item a variable, designated by its name if it is declared in the
translation schema's {\tt VAR} section, or given directly between
parentheses.  The text generated is the value of that variable
evaluated for the element to which the rule applies.

\item the value of an attribute, if the element being translated has this
attribute.  The attribute is specified by its name;

\item the value of a specific presentation rule.  This object can only
be generated if the translation rule is for a specific presentation
rule (see page~\pageref{valpres}).  It is specified by the {\tt Value}
keyword;

\item the element's content.  That is, the content of the leaves of
the subtree of the translated element.  This is specified by the {\tt
Content} keyword;

\item the comment attached to the element.  When the element doesn't
have a comment, nothing is generated.  This is indicated by the {\tt
Comment} keyword;

\item the translation of all attributes of the element (which is
primarily used to apply the attribute translation rules before those
of the element type (see section~\ref{order})).  This is
specified by the {\tt Attributes} keyword.

\item the translation of all of the element's specific presentation
rules (which is primarily used to apply the translation rules for the
specific presentation rules before those of the element or its
attributes (see section~\ref{order})).  This option is specified
by the {\tt Presentation} keyword;

\item  The value of the reference's identifier. \\
Thot associates a unique identifier with each element in a document.  This
identifier (called {\it reference's identifier} or {\it label}) is a
character string containing the letter `L' followed by digits.  Thot uses it
in references for identifying the referred element. \\
The {\tt RefId} keyword produces the reference's identifier
of the element to which the translation rule is applied, or the reference's
identifier of its first ancestor that is referred by a reference
or that can be referred by a reference.

\item the value of a mark pair's unique identifier.  This may only be used
for mark pairs (see section~\ref{paires}) and is indicated by the {\tt
PairId} keyword.

\item the directory containing the file being generated (this string
includes an ending '/', if it is not empty).  This is indicated by the
{\tt FileDir} keyword.

\item the name of the file being generated (only the name, without the
directory and without the extension).  This is indicated by the {\tt FileName}
keyword.

\item the extension of the file being generated (this string starts with a
dot, if it is not empty).  This is indicated by the {\tt Extension} keyword.

\item the name of the document being translated.  This is indicated by
the {\tt DocumentName} keyword.

\item the directory containing the document being translated.  This is
indicated by the {\tt DocumentDir} keyword.

\end{itemize}

When the rule applies to a reference (an element or an attribute defined
as a reference in the structure schema), it can generate a text related
to the element referred by that reference.  The rule name is then followed
the {\tt Referred} keyword and a specification of the object to be
generated for the referred element.  This specification can be:

\begin{itemize}
\item the name of a variable.  The rule generates the value of that
variable, computed for the referred element.

\item an element type.  The rule generates the translation of the element
of that type, which is in the subtree of the referred element.  If this
element is not defined in the structure schema which corresponds to the
translation schema (that is, an object defined in another schema), the
element's type name must be followed by the name of its structure
schema between parentheses.

\item the {\tt RefId} keyword.  The rule generates the reference's
identifier of the referred element.

\item the {\tt DocumentName} keyword.  The rule generates the name of
the document to which the referred element belongs.

\item the {\tt DocumentDir} keyword.  The rule generates the name of
the directory that contains the document of the referred element.
\end{itemize}

\subsection{The {\tt Write} rule}
\label{writerule}

The {\tt Write} has the same syntax as the {\tt Create} rule.  It also
produces the same effect, but the generated text is displayed on the
user's terminal during the translation of the document, instead of
being produced in the translated document.  This is useful for helping
the user keep track of the progress of the translation and for
prompting the user on the terminal for input required by the {\tt
Read} rule.

\begin{verbatim}
               'Write' Object [ Position ] ';'
\end{verbatim}

Notice: if the translator is launched by the editor (by the ``Save as''
command), messages produced by the {\tt Write} rule are not displayed.

\begin{example}
To make the translator display the number of each section being
translated on the user's terminal, the following rule is specified for
the {\tt Section} element type:
\begin{verbatim}
Section : BEGIN
          Write VarSection;
          ...
          END;
\end{verbatim}
(see page~\pageref{varsect} for the definition of the {\tt VarSection}
variable).

To display text on the terminal before issuing a read operation with
the {\tt Read} rule, the following rule is used:
\begin{verbatim}
BEGIN
Write 'Enter the name of the destination: ';
...
END;
\end{verbatim}
\end{example}

\subsection{The {\tt Read} rule}
\label{readrule}

The {\tt Read} rule reads text from the terminal during the
translation of the document and saves the text read in one of the
buffers declared in the {\tt BUFFERS} section of the schema.  The
buffer to be used is indicated by its name, after the {\tt READ}
keyword.  This name can be followed, as in the {\tt Create} and {\tt
Write} rules, by a keyword indicating if the read operation must be
performed {\tt Before} or {\tt After} the translation of the element's
content.  If this keyword is absent, the read operation is done
beforehand.  The text is read into the buffer and remains there until
a rule using the same buffer --- possibly the same rule --- is
applied.

\begin{verbatim}
               'Read' BufferID [ Position ] ';'
\end{verbatim}

\begin{example}
The following set of rules tells the user that the translator is
waiting for the entry of some text, reads this text into a buffer and
copies the text into the translated document.
\begin{verbatim}
BEGIN
Write 'Enter the name of the destination: ';
Read DestName;
Create DestName;
...
END;
\end{verbatim}
(see the definition of {\tt DestName} on page~\pageref{nomdest}).
\end{example}

\subsection{The {\tt Include} rule}

The {\tt Include} rule, like the {\tt Create} rule, is used to produce
text in the translated document.  It inserts constant text which is
not defined in the translation schema, but is instead taken from a
file.  The file's name  is specified after the {\tt Include} keyword,
either directly as a character string between apostrophes or as the
name of one of the buffers declared in the {\tt BUFFERS} section of
the schema.  In the latter case, the buffer is assumed to contain the
file's name.  This can be used when the included file's name is known only at
the moment of translation.  This only requires that the {\tt Include}
rule is preceded by a {\tt Read} rule which puts the name of the
file desired by the user into the buffer.

Like the other rules, it is possible to specify whether the inclusion
will occur before or after the element's content, with the default
being before.  The file inclusion is only done at the moment of
translation, not during the compilation of the translation schema.
Thus, the file to be included need not exist during the compilation,
but it must be accessible at the time of translation.  Its contents
can also be modified between two translations, thus producing
different results, even if neither the document or the translation
schema are modified.

During translation, the file to be included is searched for along the
schema directory path (indicated by the environment variable {\tt
THOTSCH}).  The file name is normally only composed of a simple name,
without specification of a complete file path.

\begin{verbatim}
                 'Include' File [ Position ] ';'
     File      = FileName / BufferID .
     FileName  = STRING .
\end{verbatim}

\begin{example}
Suppose that it is desirable to print documents of the Article class
with a formatter which requires a number of declarations and
definitions at the beginning of the file.  The {\tt Include} rule can
be used to achieve this.  All the declarations and definitions are
placed in a file called {\tt DeclarArt} and then the {\tt Article}
element type is given the following rule:
\begin{verbatim}
Article : BEGIN
          Include 'DeclarArt' Before;
          ...
          END;
\end{verbatim}
\end{example}

\subsection{The {\tt Get} rule}

The {\tt Get} rule is used to change the order in which the elements
appear in the translated document.  More precisely, it produces  the
translation of a specified element before or after the translation of
the content of the element to which the rule applies.  The {\tt
Before} and {\tt After} keywords are placed at the end of the rule to
specify whether the operation should be performed before or after
translation of the rule's element (the default is before). The type of
the element to be moved must be specified after the {\tt Get} keyword,
optionally preceded by a keyword indicating where the element will be
found in the logical structure of the document:
\begin{description}
\item[ {\tt Included}: ]
The element to be moved is the first element of the indicated type
which is found inside the element to which the rule applies.

\item[ {\tt Referred}: ]
This keyword can only be used if the rule applies to a reference
element.  The element to be moved is either the element designated by
the reference (if that element is of the specified type), or the first
element of the desired type contained within the element designated by
the reference.
\item[ no keyword: ]
If the element to be moved is an associated element, defined in the
{\tt ASSOC} section of the structure schema (see
page~\pageref{elemassoc}), all associated elements of this type which
have not been translated yet are then translated.  Certain elements
may in fact have already been translated by a {\tt Get Referred} rule.

If the element to be moved is not an associated element, the
translator takes the first element of the indicated type from among
the siblings of the rule's element.  This is primarily used to change
the order of the components of an aggregate.
\end{description}

If the element to be moved is defined in a structure schema which is
not the one which corresponds to the translation schema (in the case
of an included object with a different schema), the type name of this
element must be followed, between parentheses, by the name of the
structure schema which defines it.

\begin{verbatim}
                  'Get' [ RelPosition ] ElemID 
                        [ ExtStruct ]
                        [ Position ] ';' /
     RelPosition ='Included' / 'Referred' .
     ExtStruct   = '(' ElemID ')' .
\end{verbatim}

The {\tt Get} rule has no effect if the element which it is supposed
to move has already been translated.  Thus, the element will not be
duplicated.  It is generally best to associate the rule with the first
element which will be encountered by the translator in its traversal
of the document.  Suppose an aggregate has two elements {\tt A} and
{\tt B}, with {\tt A} appearing first in the logical structure.  To
permute these two elements, a {\tt Get B before} rule should be
associated with the {\tt A} element type, not the inverse.  Similarly,
a rule of the form {\tt Get Included X After}, even though
syntactically correct, makes no sense since, by the time it will be
applied, after the translation of the contents of the element to which
it is attached, the {\tt X} element will already have been translated.

The {\tt Get} rule is the only way to obtain the translation of the
associated elements.  In fact, the translator only traverses the
primary tree of the document and thus does not translate the
associated elements, except when the translation is explicitly
required by a {\tt Get Referred Type} or {\tt Get Type} rule where {\tt
Type} is an associated element type.


\begin{example}
The structure schema defined figures as associated element which are
composed of some content and a caption.  Moreover, it is possible to make
references to figures, using elements of the {\tt RefFigure} type:
\begin{verbatim}
     ...
     RefFigure = REFERENCE(Figure);
ASSOC
     Figure = BEGIN
              Content = NATURE;
              Caption = Text;
              END;
     ...
\end{verbatim}
Suppose it would be useful to make a figure appear in the translated
document at the place in the text where the first reference to the
figure is made.  If some figures are not referenced, then they would
appear at the end of the document.  Also, each figure's caption should
appear before the content.  The following rules in the translation
schema will produce this result:
\begin{verbatim}
Article :   BEGIN
            ...
            Get Figures After;
            END;
RefFigure : BEGIN
            If FirstRef Get Referred Figure;
            ...
            END;
Content :   BEGIN
            Get Caption Before;
            ...
            END;
\end{verbatim}
\end{example}


\subsection{The {\tt Copy} rule}

Like the {\tt Get} rule, the {\tt Copy} rule generates the translation
of a specified element, but it acts even if the element has already been
translated and it allows to copy it or to translate it later.
Both rules have the same syntax.

\begin{verbatim}
              'Copy' [ RelPosition ] ElemID 
                     [ ExtStruct ] [ Position ] ';'
\end{verbatim}

\subsection{The {\tt Use} rule}

The {\tt Use} rule specifies the translation schema to be applied to
objects of a certain class that are part of the document.  This rule
only appears in the rules for the root element of the document (the
first type defined after the {\tt STRUCT} keyword in the structure
schema) or the rules of an element defined by an external structure
(by another structure schema).  Also, the {\tt Use} rule cannot be
conditional.

If the rule is applied to an element defined by an external structure,
the {\tt Use} keyword is simply followed by the name of the
translation schema to be used for element constructed  according to
that external structure.  If the rule is applied to the document's
root element, it is formed by the {\tt Use} keyword followed by the
translation schema's name, the {\tt For} keyword and the name of the
external structure to which the indicated translation schema should be
applied.

\begin{verbatim}
               'Use' TrSchema [ 'For' ElemID ] ';'
     TrSchema = NAME .
\end{verbatim}

If no {\tt Use} rule defines the translation schema to be used for an
external structure which appears in a document, the translator asks
the user, during the translation process, which schema should be used.
Thus, it is not necessary to give the translation schema a {\tt Use}
rule for every external structure used, especially when the choice of
translation schemas is to be left to the user.

Notice: if the translator is launched by the editor (by the ``Save as''
command), prompts are not displayed.

\begin{example}
The {\tt Article} structure schema uses the {\tt Formula} external
structure, defined by another structure schema, for mathematical formulas:
\begin{verbatim}
STRUCTURE Article;
   ...
STRUCT
   Article = ...
   ...
   Formula_in_text  = Formula;
   Isolated_formula = Formula;
   ...
END
\end{verbatim}
Suppose that it would be useful to use the {\tt FormulaT} translation
schema for the formulas of an article.  This can be expressed in two
different ways in the {\tt Article} class translation schema, using
the rules:
\begin{verbatim}
RULES
    Article :
       Use FormulaT for Formula;
\end{verbatim}
or:
\begin{verbatim}
RULES
    ...
    Formula :
       Use FormulaT;
\end{verbatim}
\end{example}

\subsection{The {\tt Remove} rule}
\label{remove}

The {\tt Remove} rule indicates that nothing should be generated, in
the translated document, for the content of the element to which the
rule applies.  The content of that element is simply ignored by the
translator.  This does not prevent the generation of text for the
element itself, using the {\tt Create} or {\tt Include} rules, for
example.

The {\tt Remove} rule is simply written with the {\tt Remove} keyword.
It is terminated, like all rules, by a semicolon.

\begin{verbatim}
               'Remove' ';'
\end{verbatim}

\subsection{The {\tt NoTranslation} rule}
\label{notrans}

The {\tt NoTranslation} rule indicates to the translator that it must
not translate the content of the leaves of the element to which it
applies.  In contrast to the {\tt Remove} rule, it does not suppress
the content of the element, but it inhibits the translation of
character strings, symbols, and graphical elements contained in the
element.  These are retrieved so that after the translation of the
document, the rules of the {\tt TEXTTRANSLATE}, {\tt SYMBTRANSLATE}
and {\tt GRAPHTRANSLATE} sections (see page~\pageref{texttrans}) will
not be applied to them.

The {\tt NoTranslation} rule is
written with the {\tt NoTranslation} keyword followed by a semicolon.

\begin{verbatim}
               'NoTranslation' ';'
\end{verbatim}

\subsection{The {\tt NoLineBreak} rule}
\label{nolinebreak}

The {\tt NoLineBreak} rule indicates to the translator that it must
not generate additional line breaks in the output produced for the element
to which it applies.  This is as if it was an instruction {\tt LINELENGTH 0;}
at the beginning of the translation schema, but only for the current
element (see page~\pageref{linelength}).

The {\tt NoLineBreak} rule is
written with the {\tt NoLineBreak} keyword followed by a semicolon.

\begin{verbatim}
               'NoLineBreak' ';'
\end{verbatim}

\subsection{The {\tt ChangeMainFile} rule}
\label{changemainfile}

When the translation program starts, it opens a main output file,
whose name is given as a parameter of the program. All {\tt Create}
rules without explicit indication of the output file (see p.~\pageref{create})
write sequentially in this file. When a {\tt ChangeMainFile} rule is
executed, the main output file is closed and it is replaced by a new
one, whose name is specified in the {\tt ChangeMainFile} rule. The
{\tt Create} rules without indication of the output file that are
then executed write in this new file. Several {\tt ChangeMainFile}
rules can be executed during the same translation, for dividing the
main output into several files.

This rule is written with the {\tt ChangeMainFile} keyword followed by
the name of a variable that specify the name of the new main file. The
keyword {\tt Before} or {\tt After} can be placed at the end of the rule
to specify whether the operation should be performed before or after
translation of the rule's element (the default is before). This rule,
like all translation rules, is terminated by a semicolon.

\begin{verbatim}
               'ChangeMainFile' VarID [ Position ] ';'
\end{verbatim}

\begin{example}
To generate the translation of each section in a different file,
the following rule can be associated with type {\tt Section}.
That rule uses the {\tt VarOutpuFile} variable defined
p.~\pageref{varoutputfile}.
\begin{verbatim}
     Section:
         ChangeMainFile VarOutpuFile Before;
\end{verbatim}
If {\tt output.txt} is the name of the output file specified when
starting the translation program, translated sections are written
in files {\tt output1.txt}, {\tt output2.txt}, etc.
\end{example}

\subsection{The {\tt Set} and {\tt Add} rules}
\label{setandadd}

The {\tt Set} and {\tt Add} rules are used for modifying the value
of counters that have no counting function (see p.~\pageref{counters}).
Only this type of counter can be used in the {\tt Set} and {\tt Add}
rules.

Both rules have the same syntax: after the keyword {\tt Set} or {\tt Add}
appear the counter name and the value to assign to the counter
({\tt Set} rule) or the value to be added to the counter
({\tt Add} rule). The keyword {\tt Before} or {\tt After} can follo9w
that value to indicate when the rule must be applied: before or after
the element's content is translated. By default, {\tt Before} is assumed.
A semicolon terminates the rule.

\begin{verbatim}
               'Set' CounterID InitValue [ Position ] ';' /
               'Add' CounterID Increment [ Position ] ';'
\end{verbatim}

\subsection{Rule application order}
\label{order}

The translator translates the elements which comprise the document in
the order induced by the tree structure, except when the {\tt Get}
rule is used to change the order of translation.  For each element,
the translator first applies the rules specified for the element's
type that must be applied before translation of the element's content
(rules ending with the {\tt Before} keyword or which have no position
keyword).  If several rules meet these criteria, the translator
applies them in the order in where they appear in the translation
schema.

It then applies all rules for the attributes which the element has
(see section~\ref{tradattr}) and which must be applied before the
translation of the element's content (rules ending with the {\tt
Before} keyword or which have no position keyword).  For one attribute
value, the translator applies the rules in the order in which they are
defined in the translation schema.

The same procedure is followed with translation rules for specific
presentations.

Next, the element's content is translated, as long as a {\tt Remove}
rule does not apply.

In the next step, the translator applies rules for the specific
presentation of the element that are to be applied after translation
of the content (rules which end with the {\tt After} keyword).  The
rules for each type of presentation rule or each value are applied in
the order in which the translation appear in the schema.

Then, the same procedure is followed for translation rules for
attributes of the element.

Finally, the translator applies rules for the element which must be
applied after translation of the element's content.  These rules are
applied in the order that they appear in the translation schema.  When
the translation of an element is done, the translator procedes to
translate the following element.

This order can be changed with the {\tt Attributes} and {\tt
Presentation} options of the {\tt Create} rule (see
section~\ref{create}).

\subsection{Translation of logical attributes}
\label{tradattr}

After the rules for the element types, the translation schema defines
rules for attribute values.  This section begins with the {\tt
ATTRIBUTES} keyword and is composed of a sequence of rule blocks each
preceded by an attribute name and an optional value or value range.

If the attribute's name appears alone before the rule block, the rule
are applied to all element which have the attribute, no matter what
value the attribute has.  In this case, the attribute name is followed
by a colon before the beginning of the rule block.

The attribute's name can be followed by the name of an element type
between parentheses.  This says, as in presentation schemas, that the
rule block which follows applies not to the element which has the
attribute, but to its descendants of the type indicated between the
parentheses.

If values are given after the attribute name (or after the name of
the element type), the rules are applied only when the attribute has
the indicated values.  The same attribute can appear several times,
with different values and different translation rules.  Attribute
values are indicated in the same way as in conditions (see
section~\ref{tradcond}) and are followed by a colon before the block
of rules.

The rule block associated with an attribute is either a simple rule or a
sequence of rules delimited by the {\tt BEGIN} and {\tt END} keywords.
Note that rules associated with attribute values cannot be
conditional.

Translation rules are not required for all attributes (or their
values) defined in a structure schema.  Only those attributes for
which a particular action must be performed by the translator must have such rules.  The
rules that can be used are those described in sections~\ref{create}
to~\ref{notrans}.

\begin{verbatim}
     AttrSeq       = TransAttr < TransAttr > .
     TransAttr     = AttrID [ '(' ElemID ')' ] 
                     [ RelatAttr ] ':' RuleSeq .
     AttrID        = NAME .
     ElemID        = NAME .
\end{verbatim}

\begin{example}
The structure defined the ``Language'' attribute which can take the
values ``French'' and ``English''.  To have the French parts of the
original document removed and prevent the translation of the leaves of the English
parts, the following rules would be used:
\begin{verbatim}
ATTRIBUTES
   Language=French :
      Remove;
   Language=English :
      NoTranslation;
\end{verbatim}
\end{example}

\subsection{Translation of specific presentations}
\label{prestrans}


After the rules for attributes, the translation schema defines rules
for the specific presentation.  This section begins with the {\tt
PRESENTATION} keyword and is composed of a sequence of translation
rule blocks each preceded by a presentation rule name, optionally
accompanied by a part which depends on the particular presentation
rule.

Each of these translation rule blocks is applied when the translator
operates on an element which has a specific presentation rule of the
type indicated at the head of the block.  Depending on the type of the
specific presentation rule, it is possible to specify values of the
presentation rule for which the translation rule block should be
applied.

There are three categories of the presentation rules:
\begin{itemize}
\item
rules taking numeric values ({\tt Size, Indent, LineSpacing, LineWeight}),
\item
rules whose values are taken from a predefined list (i.e. whose type
is an enumeration) ({\tt Adjust,
Justify, Hyphenate, Style, Font, UnderLine,\\ Thickness, LineStyle}),
\item
rules whose value is a name ({\tt FillPattern, Background, Foreground}).
\end{itemize}

For presentation rules of the first category, the values which provoke
application of the translation rules are indicated in the same manner
as for numeric attributes (see page~\pageref{relattr}).  This can be
either a unique value or range of values.  For a unique value, the
value (an integer) is simply preceded by an equals sign.  Value ranges
can be specified in one of three ways:
\begin{itemize}

\item all values less than a given value (this value is preceded by a
``less than'' sign ({\tt <}),

\item all values greater than a given value (this value is preceded by a
``greater than'' sign ({\tt >}),

\item all values falling in an interval, bounds included.  The range
of values is then specified {\tt IN [}Minimum {\tt ..} Maximum{\tt]},
where Minimum and Maximum are integers.

\end{itemize}
All numeric values can be negative, in which case the integer is
preceded by a minus sign.  All values must be given in typographers
points.

For presentation rules whose values are taken from a predefined list,
the value which provokes application of the translation rules is
simply indicated by the equals sign followed by the name of the value.

For presentation rules whose values are names, the value which
provokes the application of translation rules is simply indicated by
the equals sign followed by the name of the value.\footnote{The names
of the fill patterns (the {\tt FillPattern} rule) and of the colors
(the {\tt Foreground} and {\tt Background} rules) used in Thot are the
same as in the P language.}

\begin{verbatim}
     PresSeq        = PresTrans < PresTrans > .
     PresTrans      = PresRule ':' RuleSeq .
     PresRule       = 'Size' [ PresRelation ] /
                      'Indent' [ PresRelation ] /
                      'LineSpacing' [ PresRelation ] /
                      'Adjust' [ '=' AdjustVal ] /
                      'Justify' [ '=' BoolVal ] /
                      'Hyphenate' [ '=' BoolVal ] /
                      'Style' [ '=' StyleVal ] /
                      'Font' [ '=' FontVal ] /
                      'UnderLine' [ '=' UnderLineVal ] /
                      'Thickness' [ '=' ThicknessVal ] /
                      'LineStyle' [ '=' LineStyleVal ] /
                      'LineWeight' [ PresRelation ] /
                      'FillPattern' [ '=' Pattern ] /
                      'Background' [ '=' Color ] /
                      'Foreground' [ '=' Color ] .

     PresRelation   = '=' PresValue /
                      '>' [ '-' ] PresMinimum /
                      '<' [ '-' ] PresMaximum /
                      'IN' '[' [ '-' ] PresIntervalMin '..'
                              [ '-' ] PresIntervalMax ']' .
     AdjustVal      = 'Left' / 'Right' / 'VMiddle' / 
                      'LeftWithDots' .
     BoolVal        = 'Yes' / 'No' .
     StyleVal       = 'Bold' / 'Italics' / 'Roman' /
                      'BoldItalics' / 'Oblique' /
                      'BoldOblique' .
     FontVal        = 'Times' / 'Helvetica' / 'Courier' .
     UnderLineVal   = 'NoUnderline' / 'UnderLined' /
                      'OverLined' / 'CrossedOut' .
     ThicknessVal   = 'Thick' / 'Thin' .
     LineStyleVal   =  'Solid' / 'Dashed' / 'Dotted' .
     Pattern        = NAME .
     Color          = NAME .
     PresMinimum    = NUMBER .
     PresMaximum    = NUMBER .
     PresIntervalMin= NUMBER .
     PresIntervalMax= NUMBER .
     PresValue      = [ '-' ] PresVal .
     PresVal        = NUMBER .
\end{verbatim}

\label{valpres}
The translation rules associated with specific presentation rules can
use the value of the specific presentation rule that causes them to be
applied.  This behavior is designated by the keyword {\tt Value}
(see p.~\pageref{create}).
For numerically-valued presentation rules, the numeric value is
produced.  For other presentation rules, the name of the value is
produced.

It should be noted that modifications to the layout of the document's
elements that are made using the combination of the control key and a
mouse button will have no effect on the translation of the document.

\begin{example}
Suppose that it is desirable to use the same font sizes as in the
specific presentation, but the font size must be between 10 and 18
typographer's points.  If font size is set in the translated document
by the string {\tt pointsize=n} where {\tt n} is the font size in
typographer's points then the following rules will suffice:

\begin{verbatim}
PRESENTATION
   Size < 10 :
        Create 'pointsize=10';
   Size in [10..18] :
        BEGIN
        Create 'pointsize=';
        Create Value;
        END;
   Size > 18 :
        Create 'pointsize=18';
\end{verbatim}
\end{example}

\subsection{Recoding of characters, symbols and graphics}
\label{texttrans}

The coding of characters, graphical elements and symbols as defined in
Thot does not necessarily correspond to what is required by an
application to which a Thot document must be exported.  Because of
this the translator can recode these terminal elements of the
documents structure.  The last sections of a translation schema are
intended for this purpose, each specifying the recoding rules for
one type of terminal element.

The recoding rules for character strings are grouped by alphabets.
There is a group of rules for each alphabet of the Thot document that
must be translated.  Each such group of rules begins with the {\tt
TEXTTRANSLATE} keyword, followed by the specification of the alphabet
to translate and the recoding rules, between the {\tt BEGIN} and {\tt
END} keywords unless there is only one recoding rule for the alphabet.
The specification of the alphabet is not required: by default it is
assumed to the Latin alphabet (the ISO Latin-1 character set).

Each recoding rule is formed by a source string between apostrophes
and a target string, also between apostrophes, the two strings being
separated by the arrow symbol ({\tt ->}), formed by the ``minus'' and
``greater than'' characters.  The rule is terminated by a semi-colon.

\begin{verbatim}
     TextTransSeq = [ Alphabet ] TransSeq .
     Alphabet     = NAME .
     TransSeq     ='BEGIN' < Translation > 'END' ';' /
                    Translation .
     Translation  = Source [ '->' Target ] ';' .
     Source       = STRING .
     Target       = STRING .
\end{verbatim}

One such rule signifies that when the source string appears in a text
leaf of the document being translated, the translator must replace it,
in the translated document, with the target string.  The source string
and the target string can have different lengths and the target string
can be empty.  In this last case, the translator simply suppresses
every occurrence of the source string in the translated document.

For a given alphabet, the order of the rules is not important and has
no significance because the T language compiler reorders the rules in
ways that speed up the translator's work.  The total number of
recoding rules is limited by the compiler as is the maximum length
of the source and target strings.

The recoding rules for symbols and graphical elements are written in
the same manner as the recoding rules for character strings.  They are
preceded, respectively, by the {\tt SYMBTRANSLATE} and {\tt
GRAPHTRANSLATE} and so not require a specification of the alphabet.
Their source string is limited to one character, since, in Thot, each
symbol and each graphical element is represented by a single
character.  The symbol and graphical element codes are defined on
page~\pageref{codage}, along with the non-standard character codes.

\begin{example}

In a translation schema producing documents destined for use with the
{\LaTeX} formatter, the Latin characters ``\'{e}'' (octal code 351 in
Thot) and ``\`{e}'' (octal code 350 in Thot)  must be converted to
their representation in {\LaTeX}:
\begin{verbatim}
TEXTTRANSLATE	Latin
     BEGIN
     '\350' -> '\`{e}';    { e grave }
     '\351' -> '\''{e}';   { e acute }
     END;
\end{verbatim}
\end{example}

\chapter{Language grammars}

This chapter gives the complete grammars of the languages of Thot.
The grammars were presented and described in the preceding chapters,
which also specify the semantics of the languages.  This section gives
only the syntax.

\section{The M meta-language}

The language grammars are all expressed in the same formalism, the M
meta-language, which is defined in this section.

\begin{verbatim}
{ Any text between braces is a comment. }
Grammar      = Rule < Rule > 'END' .
               { The < and > signs indicate zero }
               { or more repetitions. }
               { END marks the end of the grammar. }
Rule         = Ident '=' RightPart '.' .
               { The period indicates the end of a rule }
RightPart    = RtTerminal / RtIntermed .
               { The slash indicates a choice }
RtTerminal   ='NAME' / 'STRING' / 'NUMBER' .
               { Right part of a terminal rule }
RtIntermed   = Possibility < '/' Possibility > .
               { Right part of an intermediate rule }
Possibility  = ElemOpt < ElemOpt > .
ElemOpt      = Element / '[' Element < Element > ']' /
              '<' Element < Element > '>'  .
               { Brackets delimit optional parts }
Element      = Ident / KeyWord .
Ident        = NAME .
               { Identifier, sequence of characters
KeyWord      = STRING .
               { Character string delimited by apostrophes }
END
\end{verbatim}

\section{The S language}

The S language is used to write structure schemas, which contain the
generic logical structures of document and object classes.  It is
described here in the M meta-language.

\begin{verbatim}
StructSchema   = 'STRUCTURE' [ 'EXTENSION' ] ElemID ';'
                 'DEFPRES' PresID ';'
               [ 'ATTR' AttrSeq ]
               [ 'PARAM' RulesSeq ]
               [ 'STRUCT' RulesSeq ]
               [ 'EXTENS' ExtensRuleSeq ]
               [ 'ASSOC' RulesSeq ]
               [ 'UNITS' RulesSeq ]
               [ 'EXPORT' SkeletonSeq ]
               [ 'EXCEPT' ExceptSeq ]
                 'END' .

ElemID         = NAME .
PresID         = NAME .

AttrSeq        = Attribute < Attribute > .
Attribute      = AttrID '=' AttrType ';' .
AttrType       = 'INTEGER' / 'TEXT' /
                 'REFERENCE' '(' RefType ')' /
                 ValueSeq .
RefType        = 'ANY' /
                 [ FirstSec ] ElemID [ ExtStruct ] .
ValueSeq       = AttrVal < ',' AttrVal > .
AttrID         = NAME .
FirstSec       = 'First' / 'Second' .
ExtStruct      = '(' ElemID ')' .
AttrVal        = NAME .

RulesSeq       = Rule < Rule > .
Rule           = ElemID [ LocAttrSeq ] '='
                 DefWithAttr ';' .
LocAttrSeq     = '(' 'ATTR' LocalAttr
                      < ';' LocalAttr > ')' .
LocalAttr      = [ '!' ] AttrID [ '=' AttrType ] .
DefWithAttr    = Definition
                 [ '+' '(' ExtensionSeq ')' ]
                 [ '-' '(' RestrictSeq ')' ]
                 [ 'WITH' FixedAttrSeq ] .
ExtensionSeq   = ExtensionElem < ',' ExtensionElem > .
ExtensionElem  = ElemID / 'TEXT' / 'GRAPHICS' /
                 'SYMBOL' / 'PICTURE' .
RestrictSeq    = RestrictElem < ',' RestrictElem > .
RestrictElem   = ElemID / 'TEXT' / 'GRAPHICS' /
                 'SYMBOL' / 'PICTURE' .
FixedAttrSeq   = FixedAttr < ',' FixedAttr > .
FixedAttr      = AttrID [ FixedOrModifVal ] .
FixedOrModifVal= [ '?' ] '=' FixedValue .
FixedValue     = [ '-' ] NumValue / TextValue / AttrVal .
NumValue       = NUMBER .
TextValue      = STRING .

Definition     = BaseType [ LocAttrSeq ] / Constr /
                 Element .
BaseType       = 'TEXT' / 'GRAPHICS' / 'SYMBOL' /
                 'PICTURE' / 'UNIT' / 'NATURE' .
Element        = ElemID [ ExtOrDef ] .
ExtOrDef       = 'EXTERN' / 'INCLUDED' /
                 [ LocAttrSeq ] '=' Definition .

Constr         = 'LIST' [ '[' min '..' max ']' ] 'OF'
                        '(' DefWithAttr ')' /
                 'BEGIN' DefOptSeq 'END' /
                 'AGGREGATE' DefOptSeq 'END' /
                 'CASE' 'OF' DefSeq 'END' /
                 'REFERENCE' '(' RefType ')' /
                 'PAIR' .

min            = Integer / '*' .
max            = Integer / '*' .
Integer        = NUMBER .

DefOptSeq      = DefOpt ';' < DefOpt ';' > .
DefOpt         = [ '?' ] DefWithAttr .

DefSeq         = DefWithAttr ';' < DefWithAttr ';' > .

SkeletonSeq    = SkeletonElem < ',' SkeletonElem > ';' .
SkeletonElem   = ElemID [ 'WITH' Contents ] .
Contents       = 'Nothing' / ElemID [ ExtStruct ] .

ExceptSeq      = Except ';' < Except ';' > .
Except         = [ 'EXTERN' ] [ FirstSec ] ExcTypeOrAttr ':'
                 ExcValSeq .
ExcTypeOrAttr  = ElemID / AttrID .
ExcValSeq      = ExcValue < ',' ExcValue > .
ExcValue       = 'NoCut' / 'NoCreate' /
                 'NoHMove' / 'NoVMove' / 'NoMove' /
                 'NoHResize' / 'NoVResize' / 'NoResize' /
                 'NewWidth' / 'NewHeight' /
                 'NewHPos' / 'NewVPos' /
                 'Invisible' / 'NoSelect' /
                 'Hidden' / 'ActiveRef' /
		 'ImportLine' / 'ImportParagraph' /
		 'NoPaginate' / 'HighlightChildren' /
		 'ExtendedSelection' .

ExtensRuleSeq  = ExtensRule ';' < ExtensRule ';' > .
ExtensRule     = RootOrElem [ LocAttrSeq ]
                 [ '+' '(' ExtensionSeq ')' ]
                 [ '-' '(' RestrictSeq ')' ]
                 [ 'WITH' FixedAttrSeq ] .
RootOrElem     = 'Root' / ElemID .

END
\end{verbatim}

\section{The P language}

The P language is used to write presentation schemas, which the
defined the graphical presentation rules to be applied to different
classes of documents and objects.  It is described here in the M
meta-language.
\begin{verbatim}
PresSchema      = 'PRESENTATION' ElemID ';'
                [ 'VIEWS' ViewSeq ]
                [ 'PRINT' PrintViewSeq ]
                [ 'COUNTERS' CounterSeq ]
                [ 'CONST' ConstSeq ]
                [ 'VAR' VarSeq ]
                [ 'DEFAULT' ViewRuleSeq ]
                [ 'BOXES' BoxSeq ]
                [ 'RULES' PresentSeq ]
                [ 'ATTRIBUTES' PresAttrSeq ]
                [ 'TRANSMIT' TransmitSeq ]
                  'END' .

ElemID          = NAME .

ViewSeq         = ViewDeclaration
                  < ',' ViewDeclaration > ';' .
ViewDeclaration = ViewID [ 'EXPORT' ] .
ViewID          = NAME .

PrintViewSeq    = PrintView < ',' PrintView > ';' .
PrintView       = ViewID / ElemID .

CounterSeq      = Counter < Counter > .
Counter         = CounterID ':' CounterFunc ';' .
CounterID       = NAME .
CounterFunc     = 'RANK' 'OF' TypeOrPage [ SLevelAsc ]
                  [ 'INIT' AttrID ] [ 'REINIT' AttrID ] /
                  SetFunction < SetFunction >
                  AddFunction < AddFunction >
                  [ 'INIT' AttrID ] /
                  'RLEVEL' 'OF' ElemID .
SLevelAsc       = [ '-' ] LevelAsc .
LevelAsc        = NUMBER .
SetFunction     = 'SET' CounterValue 'ON' TypeOrPage .
AddFunction     = 'ADD' CounterValue 'ON' TypeOrPage .
TypeOrPage      = 'Page' [ '(' ViewID ')' ] /
                  [ '*' ] ElemID .
CounterValue    = NUMBER .

ConstSeq        = Const < Const > .
Const           = ConstID '=' ConstType ConstValue ';' .
ConstID         = NAME .
ConstType       = 'Text' [ Alphabet ] / 'Symbol' /
                  'Graphics' / 'Picture' .
ConstValue      = STRING .
Alphabet        = NAME .

VarSeq          = Variable < Variable > .
Variable        = VarID ':' FunctionSeq ';' .
VarID           = NAME .
FunctionSeq     = Function < Function > .
Function        = 'DATE' / 'FDATE' /
                  'DocName' / 'DirName' /
		  'ElemName' / 'AttributeName' /
                  ConstID / ConstType ConstValue /
                  AttrID /
                  'VALUE' '(' PageAttrCtr ','
                  CounterStyle ')' .
PageAttrCtr     = 'PageNumber' [ '(' ViewID ')' ] /
                  [ MinMax ] CounterID / AttrID .
CounterStyle    = 'Arabic' / 'LRoman' / 'URoman' /
                  'Uppercase' / 'Lowercase' .
MinMax          = 'MaxRangeVal' / 'MinRangeVal' .

BoxSeq          = Box < Box > .
Box             = 'FORWARD' BoxID ';' /
                  BoxID ':' ViewRuleSeq .
BoxID           = NAME .

PresentSeq      = Present < Present > .
Present         = [ '*' ] [ FirstSec ] ElemID ':'
                  ViewRuleSeq .
FirstSec        = 'First' / 'Second' .

PresAttrSeq     = PresAttr < PresAttr > .
PresAttr        = AttrID [ '(' [ FirstSec ] ElemID ')' ] 
                  [ AttrRelation ] ':' ViewRuleSeq .
AttrID          = NAME .
AttrRelation    = '=' AttrVal /
                  '>' [ '-' ] MinValue /
                  '<' [ '-' ] MaxValue /
                  'IN' '[' [ '-' ] LowerBound '..' 
                  [ '-' ] UpperBound ']' /
                  'GREATER' AttrID /
                  'EQUAL' AttrID /
                  'LESS' AttrID .
AttrVal         = [ '-' ] EqualNum / EqualText / AttrValue .
MinValue        = NUMBER .
MaxValue        = NUMBER .
LowerBound      = NUMBER .
UpperBound      = NUMBER.
EqualNum        = NUMBER .
EqualText       = STRING .
AttrValue       = NAME .

ViewRuleSeq     = 'BEGIN' < RulesAndCond > < ViewRules >
                  'END' ';' /
                  ViewRules / CondRules / Rule .
RulesAndCond    = CondRules / Rule .
ViewRules       = 'IN' ViewID CondRuleSeq .
CondRuleSeq     = 'BEGIN' < RulesAndCond > 'END' ';' /
                  CondRules / Rule .
CondRules       = CondRule < CondRule >
                  [ 'Otherwise' RuleSeq ] .
CondRule        = 'IF' ConditionSeq RuleSeq .
RulesSeq        = 'BEGIN' Rule < Rule > 'END' ';' / Rule .

ConditionSeq   = Condition < 'AND' Condition > .
Condition      = [ 'NOT' ] [ 'Target' ] ConditionElem .
ConditionElem  = 'First' / 'Last' /
                 [ 'Immediately' ] 'Within' [ NumParent ]
                                    ElemID [ ExtStruct ] /
                  ElemID /
                 'Referred' / 'FirstRef' / 'LastRef' /
                 'ExternalRef' / 'InternalRef' / 'CopyRef' /
                 'AnyAttributes' / 'FirstAttr' / 'LastAttr' /
                 'UserPage' / 'StartPage' / 'ComputedPage' /
                 'Empty' /
                 '(' [ MinMax ] CounterName CounterCond ')' /
                 CondPage '(' CounterID ')' .
NumParent      = [ GreaterLess ] NParent .
GreaterLess    = '>' / '<' .
NParent        = NUMBER.
CounterCond    = '<' MaxCtrVal / '>' MinCtrVal /
                 '=' EqCtrVal / 
                 'IN' '[' ['-'] MinCtrBound '..' 
                 ['-'] MaxCtrBound ']' .
PageCond       = 'Even' / 'Odd' / 'One' .
MaxCtrVal      = NUMBER .
MinCtrVal      = NUMBER .
EqCtrVal       = NUMBER .
MaxCtrBound    = NUMBER .
MinCtrBound    = NUMBER .

Rule            = PresParam ';' / PresFunc ';' .
PresParam       = 'VertRef' ':' HorizPosition /
                  'HorizRef' ':' VertPosition /
                  'VertPos' ':' VPos /
                  'HorizPos' ':' HPos /
                  'Height' ':' Extent /
                  'Width' ':' Extent /
                  'LineSpacing' ':' DistOrInherit /
                  'Indent' ':' DistOrInherit /
                  'Adjust' ':' AlignOrInherit /
                  'Justify' ':' BoolInherit /
                  'Hyphenate' ':' BoolInherit /
                  'PageBreak' ':' Boolean /
                  'LineBreak' ':' Boolean /
                  'InLine' ':' Boolean /
                  'NoBreak1' ':' AbsDist /
                  'NoBreak2' ':' AbsDist /
                  'Gather' ':' Boolean /
                  'Visibility' ':' NumberInherit /
                  'Size'  ':' SizeInherit /
                  'Font' ':' NameInherit /
                  'Style' ':' StyleInherit /
                  'Underline' ':' UnderLineInherit /
                  'Thickness' ':' ThicknessInherit /
                  'Depth' ':' NumberInherit /
                  'LineStyle' ':' LineStyleInherit /
                  'LineWeight' ':' DistOrInherit /
                  'FillPattern' ':' NameInherit /
                  'Background' ':' NameInherit /
                  'Foreground' ':' NameInherit .
                  'Content' ':' VarConst .
PresFunc        = Creation '(' BoxID ')' /
                  'Line' /
                  'NoLine' /
                  'Page' '(' BoxID ')' /
                  'Copy' '(' BoxTypeToCopy ')' .

BoxTypeToCopy   = BoxID [ ExtStruct ] /
                   ElemID [ ExtStruct ] .
ExtStruct       = '(' ElemID ')' .

Distance        = [ Sign ] AbsDist .
Sign            = '+' / '-' .
AbsDist         = IntegerOrAttr [ '.' DecimalPart ]
                  [ Unit ] .
IntegerOrAttr   = IntegerPart / AttrID .
IntegerPart     = NUMBER .
DecimalPart     = NUMBER .
Unit            = 'em' / 'ex' / 'cm' / 'mm' / 'in' / 'pt' /
                  'pc' / 'px' / '%' .

HPos            = 'nil' / VertAxis '=' HorizPosition 
                  [ 'UserSpecified' ] .
VPos            = 'nil' / HorizAxis '=' VertPosition 
                  [ 'UserSpecified' ] .
VertAxis        = 'Left' / 'VMiddle' / 'VRef' / 'Right' .
HorizAxis       = 'Top' / 'HMiddle' / 'HRef' / 'Bottom' .

VertPosition   = Reference '.' HorizAxis [ Distance ] .
HorizPosition  = Reference '.' VertAxis [ Distance ] .
Reference      = 'Enclosing' [ BoxTypeNot ] /
                 'Enclosed' [ BoxTypeNot ] /
                 'Previous' [ BoxTypeNot ] /
                 'Next' [ BoxTypeNot ] /
                 'Referred' [ BoxTypeNot ] /
                 'Creator' /
                 'Root' /
                 '*' /
                 BoxOrType .
BoxOrType      = BoxID /
                 [ '*' ] [ FirstSec ] ElemID /
                 'AnyElem' / 'AnyBox' .
BoxTypeNot     = [ 'NOT' ] BoxOrType .

Extent         = Reference '.' HeightWidth
                 [ Relation ] [ 'Min' ] /
                 AbsDist [ 'UserSpecified' ] [ 'Min' ] /
                 HPos / VPos .
HeightWidth    = 'Height' / 'Width' .
Relation       = '*' ExtentAttr '%' / Distance .
ExtentAttr     = ExtentVal / AttrID .
ExtentVal      = NUMBER .

Inheritance    = Kinship  InheritedValue .
Kinship        = 'Enclosing' / 'GrandFather'/ 'Enclosed' /
                 'Previous' / 'Creator' .
InheritedValue = '+' PosIntAttr [ 'Max' maximumA ] /
                 '-' NegIntAttr [ 'Min' minimumA ] /
                 '=' .
PosIntAttr     = PosInt / AttrID .
PosInt         = NUMBER .
NegIntAttr     = NegInt / AttrID .
NegInt         = NUMBER .
maximumA       = maximum / AttrID .
maximum        = NUMBER .
minimumA       = minimum / AttrID .
minimum        = NUMBER .

AlignOrInherit = Kinship '=' / Alignment .
Alignment      = 'Left' / 'Right' / 'VMiddle' /
                 'LeftWithDots' .

DistOrInherit  = Kinship InheritedDist / Distance .
InheritedDist  = '=' / '+' AbsDist / '-' AbsDist .

BoolInherit    = Boolean / Kinship '=' .
Boolean        = 'Yes' / 'No' .

NumberInherit  = Integer / AttrID / Inheritance .
Integer        = NUMBER .

LineStyleInherit= Kinship '=' / 'Solid' / 'Dashed' /
                  'Dotted' .

SizeInherit    = SizeAttr [ 'pt' ] / Kinship InheritedSize .
InheritedSize  = '+' SizeAttr [ 'pt' ]
                     [ 'Max' MaxSizeAttr ] /
                 '-' SizeAttr [ 'pt' ]
                     [ 'Min' MinSizeAttr ] /
                 '=' .
SizeAttr       = Size / AttrID .
Size           = NUMBER .
MaxSizeAttr    = MaxSize / AttrID .
MaxSize        = NUMBER .
MinSizeAttr    = MinSize / AttrID .
MinSize        = NUMBER .

NameInherit    = Kinship '=' / FontName .
FontName       = NAME .
StyleInherit   = Kinship '=' /
                 'Roman' / 'Bold' / 'Italics' / 
                 'BoldItalics' / 'Oblique' / 'BoldOblique' .
UnderLineInherit= Kinship '=' /
                 'NoUnderline' / 'Underlined' / 
                 'Overlined' / 'CrossedOut' .
ThicknessInherit= Kinship '=' / 'Thick' / 'Thin' .

VarConst       = ConstID / ConstType ConstValue /
                 VarID / '(' FunctionSeq ')' /
                 ElemID .

Creation       = Create [ 'Repeated' ] .
Create         = 'CreateFirst' / 'CreateLast' /
                 'CreateBefore' / 'CreateAfter' /
                 'CreateEnclosing' .

TransmitSeq    = Transmit < Transmit > .
Transmit       = TypeOrCounter 'To' ExternAttr
                 '(' ElemID ')' ';' .
TypeOrCounter  = CounterID / ElemID .
ExternAttr     = NAME .

END
\end{verbatim}

\section{The T language}

\begin{verbatim}
TransSchema   = 'TRANSLATION' ElemID ';'
              [ 'LINELENGTH' LineLength ';' ]
              [ 'LINEEND' CHARACTER ';' ]
              [ 'LINEENDINSERT' STRING ';' ]
              [ 'BUFFERS' BufferSeq ]
              [ 'COUNTERS' CounterSeq ]
              [ 'CONST' ConstSeq ]
              [ 'VAR' VariableSeq ]
                'RULES' ElemSeq
              [ 'ATTRIBUTES' AttrSeq ]
              [ 'PRESENTATION' PresSeq ]
              < 'TEXTTRANSLATE' TextTransSeq >
              [ 'SYMBTRANSLATE' TransSeq ]
              [ 'GRAPHTRANSLATE' TransSeq ]
                'END' .

LineLength    = NUMBER .

BufferSeq     = Buffer < Buffer > .
Buffer        = BufferID [ '(' 'Picture' ')' ] ';' .
BufferID      = NAME .

CounterSeq    = Counter < Counter > .
Counter       = CounterID [ ':' CounterFunc ] ';' .
CounterID     = NAME .
CounterFunc   = 'Rank' 'of' ElemID [ SLevelAsc ]
                [ 'Init' AttrID ] /
                'Rlevel' 'of' ElemID /
                'Set' InitValue 'On' ElemID
                      'Add' Increment 'On' ElemID
                      [ 'Init' AttrID ] .
SLevelAsc     = [ '-' ] LevelAsc .
LevelAsc      =  NUMBER .
InitValue     = NUMBER .
Increment     = NUMBER .
ElemID        = NAME .
AttrID        = NAME .

ConstSeq      = Const < Const > .
Const         = ConstID '=' ConstValue ';' .
ConstID       = NAME .
ConstValue    = STRING .

VariableSeq   = Variable < Variable > .
Variable      = VarID ':' Function < Function > ';' .
VarID         = NAME .
Function      = 'Value' '(' CounterID [ ':' Length ]
                          [ ',' CounterStyle ]  ')' /
                'FileDir' / 'FileName' / 'Extension' /
		'DocumentName' / 'DocumentDir' /
                ConstID / CharString / 
                BufferID / AttrID .
Length        = NUMBER .
CounterStyle=   'Arabic' / 'LRoman' / 'URoman' /
                'Uppercase' / 'Lowercase' .
CharString    = STRING .

ElemSeq       = TransType < TransType > .
TransType     = [ FirstSec ] ElemID ':' RuleSeq .
FirstSec      = 'First' / 'Second' .
RuleSeq       = Rule / 'BEGIN' < Rule > 'END' ';' .
Rule          = SimpleRule / ConditionBlock .
ConditionBlock= 'IF' ConditionSeq SimpleRuleSeq .
SimpleRuleSeq = 'BEGIN' < SimpleRule > 'END' ';' / 
                SimpleRule .

ConditionSeq  = Condition [ 'AND' Condition ] .
Condition     = [ 'NOT' ] [ 'Target' ] Cond .
Cond          = CondElem / CondAscend .
CondElem      = 'FirstRef' / 'LastRef' /
                'ExternalRef' /
                'Defined' /
                'Alphabet' '=' Alphabet /
                'ComputedPage' / 'StartPage' / 
                'UserPage' / 'ReminderPage' /
                'Empty' /
                'FirstAttr' / 'LastAttr' .
CondAscend    = [ Ascend ] CondOnAscend .
Ascend        = '*' / 'Parent' / 'Ancestor' LevelOrType .
LevelOrType   = CondRelLevel / ElemID [ ExtStruct ] .
CondRelLevel  = NUMBER .
CondOnAscend  = 'First' / 'Last' /
                'Referred' / 
                [ 'Immediately' ] 'Within' [ NumParent ]
                                  ElemID [ ExtStruct ] /
                'Attributes' /
                AttrID [ RelatAttr ] /
                'Presentation' /
                PresRule /
                'Comment' .		  
NumParent     = [ GreaterLess ] NParent .
GreaterLess   = '>' / '<' .
NParent       = NUMBER.
Alphabet      = NAME .
RelatAttr     = '=' Value /
                 '>' [ '-' ] Minimum /
                 '<' [ '-' ] Maximum /
                 'IN' '[' [ '-' ] MinInterval '..'
                          [ '-' ] MaxInterval ']' .
Value         = [ '-' ] IntegerVal / TextVal / AttrValue .
Minimum       = NUMBER .
Maximum       = NUMBER .
MinInterval   = NUMBER .
MaxInterval   = NUMBER .
IntegerVal    = NUMBER .
TextVal       = STRING .
AttrValue     = NAME .

SimpleRule    = 'Create' [ 'IN' VarID ] Object
                       [ Position ] ';' /
                'Write' Object [ Position ] ';' /
                'Read' BufferID [ Position ] ';' /
                'Include' File [ Position ] ';' /
                'Get' [ RelPosition ] ElemID 
                      [ ExtStruct ] 
                      [ Position ] ';' /
                'Copy' [ RelPosition ] ElemID 
                       [ ExtStruct ] 
                       [ Position ] ';' /
                'Use' TrSchema [ 'For' ElemID ] ';' /
                'Remove' ';' /
                'NoTranslation' ';' /
		'NoLineBreak' ';' /
                'ChangeMainFile' VarID [ Position ] ';' /
                'Set' CounterID InitValue
                      [ Position ] ';' /
                'Add' CounterID Increment
                      [ Position ] ';' .

Object        = ConstID / CharString /
                BufferID /
                VarID /
                '(' Function < Function > ')' /
                 AttrID /
                'Value' /
                'Content' /
                'Comment' / 
                'Attributes' /
                'Presentation' /
                'RefId' /
                'PairId' /
		'FileDir' / 'FileName' / 'Extension' /
                'DocumentName' / 'DocumentDir' /
                [ 'Referred' ] ReferredObject .
Position      = 'After' / 'Before' .

ReferredObject = VarID /
                 ElemID [ ExtStruct ] /
                 'RefId' /
                 'DocumentName' / 'DocumentDir' .                

File           = FileName / BufferID .
FileName       = STRING .

RelPosition    = 'Included' / 'Referred' .
ExtStruct      = '(' ElemID ')' .

TrSchema       = NAME .

AttrSeq        = TransAttr < TransAttr > .
TransAttr      = AttrID [ '(' ElemID ')' ] 
                 [ RelatAttr ] ':' RuleSeq .

PresSeq        = PresTrans < PresTrans > .
PresTrans      = PresRule ':' RuleSeq .
PresRule       = 'Size' [ PresRelation ] /
                 'Indent' [ PresRelation ] /
                 'LineSpacing' [ PresRelation ] /
                 'Adjust' [ '=' AdjustVal ] /
                 'Justify' [ '=' BoolVal ] /
                 'Hyphenate' [ '=' BoolVal ] /
                 'Style' [ '=' StyleVal ] /
                 'Font' [ '=' FontVal ] /
                 'UnderLine' [ '=' UnderLineVal ] /
                 'Thickness' [ '=' ThicknessVal ] /
                 'LineStyle' [ '=' LineStyleVal ] /
                 'LineWeight' [ PresRelation ] /
                 'FillPattern' [ '=' Pattern ] /
                 'Background' [ '=' Color ] /
                 'Foreground' [ '=' Color ] .

PresRelation   = '=' PresValue /
                 '>' [ '-' ] PresMinimum /
                 '<' [ '-' ] PresMaximum /
                 'IN' '[' [ '-' ] PresIntervalMin '..'
                          [ '-' ] PresIntervalMax ']' .
AdjustVal      = 'Left' / 'Right' / 'VMiddle' / 
                 'LeftWithDots' .
BoolVal        = 'Yes' / 'No' .
StyleVal       = 'Bold' / 'Italics' / 'Roman' /
                 'BoldItalics' / 'Oblique' /
                 'BoldOblique' .
FontVal        = 'Times' / 'Helvetica' / 'Courier' .
UnderLineVal   = 'NoUnderline' / 'UnderLined' /
                 'OverLined' / 'CrossedOut' .
ThicknessVal   = 'Thick' / 'Thin' .
LineStyleVal   = 'Solid' / 'Dashed' / 'Dotted' .
Pattern        = NAME .
Color          = NAME .
PresMinimum    = NUMBER .
PresMaximum    = NUMBER .
PresIntervalMin= NUMBER .
PresIntervalMax= NUMBER .
PresValue      = [ '-' ] PresVal .
PresVal        = NUMBER .

TextTransSeq   = [ Alphabet ] TransSeq .
Alphabet       = NAME .
TransSeq       = 'BEGIN' < Translation > 'END' ';' /
                 Translation .
Translation    = Source [ '->' Target ] ';' .
Source         = STRING .
Target         = STRING .
\end{verbatim}

\chapter{Character coding}
\label{codage}

\section{Characters}

The characters of the Latin alphabet follow the encoding defined in
the ISO 8859-1 (ISO Latin 1) standard.  The characters of the Greek
alphabet follow the encoding defined by Adobe for its Symbol font
(Adobe FontSpecific).

Characters whose octal code is greater than 0200 are written in the
form of their octal code preceded by a backslash character
(``$\backslash$'').  For example, the word R\'{e}sum\'{e} is written
{\tt R$\backslash$351sum$\backslash$351}.

To the ISO 8859-1 encoding six characters with the following codes
have been added: \\
{\tt 212} : line break\\
{\tt 240} : sticky space\\
{\tt 201} : thin space\\
{\tt 202} : en space\\
{\tt 230} : $\div$\\
{\tt 231} : $\times$\\
{\tt 367} : \oe\\
{\tt 327} : \OE 

The {\tt 212} character is a ``line break'' character which forces a
line break.  The {\tt 240} character is a ``sticky space'', which
cannot be replaced by a line break.

\section{Symbols}
\label{codesymbole}

The table below gives the codes for the symbols of Thot.  Symbols can
be used in presentation schemas constants and in transcoding rules of
translation schemas.  Each symbol is represented by a single
character.

\begin{description}
\item{ {\tt r }}: a radical $\surd$
\item{ {\tt i }}: a simple integral $\int$
\item{ {\tt c }}: a curvilinear integral $\oint$
\item{ {\tt d }}: a double integral $\int\int$
\item{ {\tt t }}: a triple integral $\int\int\int$
\item{ {\tt S }}: the summation symbol $\sum$
\item{ {\tt P }}: the product symbol $\prod$
\item{ {\tt U }}: the union symbol $\cup$
\item{ {\tt I }}: the intersection symbol $\cap$
\item{ {\tt > }}: a right arrow $\rightarrow$
\item{ {\tt < }}: a left arrow $\leftarrow$
\item{ {\tt \^{ } }}:  an up arrow $\uparrow$
\item{ {\tt V }}: a down arrow $\downarrow$
\item{ {\tt ( }}: an opening parenthesis (
\item{ {\tt ) }}: a closing parenthesis )
\item{ {\tt \{ }}: an opening brace \{
\item{ {\tt \} }}: a closing brace \}
\item{ {\tt [ }}: an opening bracket [
\item{ {\tt ] }}: a closing bracket ]
\end{description}

\section{Graphical elements}

The table below gives the codes for the graphical elements of Thot.
These elements can be used in presentation schemas constants and in
transcoding rules of translation schemas.  Each graphical element is
represented by a single character.

\begin{description}
\item{ {\tt c }}: an ellipse inscribed in the box $\bigcirc$
\item{ {\tt R }}: a rectangle which is the shape of the box
\item{ {\tt C }}: a rectangle with rounded corners
\item{ {\tt t }}: a horizontal line along the upper side of the box ---
\item{ {\tt h }}: a horizontal line as wide as the box and placed in
its middle ---
\item{ {\tt b }}: a horizontal line along the lower side of the box \_\_

\item{ {\tt > }}: a right arrow as long as the box's width and in its
middle $\rightarrow$

\item{ {\tt > }}: a left arrow as long as the box's width and in its
middle $\leftarrow$

\item{ {\tt l }}: a vertical line on the left side of the box $\mid$
\item{ {\tt v }}: a vertical line as tall as the box and placed in its
middle $\mid$
\item{ {\tt r }}: a vertical line on the right side of the box $\mid$
\item{ {\tt \^{ } }}: an up arrow as tall as the box and in its middle
$\uparrow$
\item{ {\tt V }}: a down arrow as tall as the box and in its middle
$\downarrow$
\item{ {\tt / }}: The southwest/northeast diagonal of the box /
\item{ $\backslash$ }: the northwest/southeast diagonal of the box $\backslash$
\item{ {\tt O }}: The northwest/southeast diagonal of the box with an
arrowhead at the top $\nwarrow$
\item{ {\tt e }}: The northwest/southeast diagonal of the box with an
arrowhead at the bottom $\searrow$
\item{ {\tt E }}: The southwest/northeast diagonal of the box with an
arrowhead at the top $\nearrow$
\item{ {\tt o }}: The southwest/northeast diagonal of the box with an
arrowhead at the bottom $\swarrow$
\item{ {\tt space }}: a transparent element
\item{ {\tt P }}: a rectangle with round corners and a horizontal bar
at the top
\item{ {\tt Q }}: an ellipse with a horizontal bar at the top: $\bigcirc$
\item{ {\tt L }}: a lozenge
\item{ {\tt W }}: the upper right corner: $\rceil$
\item{ {\tt X }}: the lower right corner: $\rfloor$
\item{ {\tt Y }}: the lower left corner: $\lfloor$
\item{ {\tt Z }}: the upper left corner: $\lceil$
\item{ {\tt p }}: a polygon
\item{ {\tt S }}: an open broken line
\item{ {\tt N }}: an open broken line with an arrow head at start
\item{ {\tt U }}: an open broken line with an arrow head at the end
\item{ {\tt M }}: an open broken line with two arrow heads
\item{ {\tt s }}: a closed curve
\item{ {\tt B }}: an open curve
\item{ {\tt F }}: an open curve with an arrow head at start
\item{ {\tt A }}: an open curve with an arrow head at the end
\item{ {\tt D }}: an open curve with two arrow heads
\end{description}

\end{document}
